\section{Expectativas Iniciales}

Inicialmente, se esperaba que la arquitectura Delta proporcionara una mayor flexibilidad y escalabilidad en comparación con la arquitectura Kappa.
Sobre todo porque a nivel publicitario, en los últimos tiempos la industria de ingeniería de datos se ha encargado de promover la arquitectura Delta, 
en particular los Data Lakehouse, como la solución definitiva para el procesamiento de datos en tiempo real y batch.

Por lo que a priori, se esperaba que Delta fuera más eficiente en términos de rendimiento y costos, y en simplicidad de implementación y mantenimiento.

Por otro lado, dado que Delta utiliza de forma extensiva el Object Storage, se esperaba una pérdida de rendimiento, y en particular más latencia, que su contraparte. 
Sin embargo, no se esperaba una diferencia demasiado grande ya que muchas veces se propone utilizar un Data Lakehouse como un sistema de almacenamiento de datos en tiempo real.

Por último, dado el apogeo de los Data Lakehouse se suponía una facilidad en el ensamblado y funcionamiento conjunto de los distintos componentes de la arquitectura,
asumiendo que hubieran soluciones estandarizadas y abiertas que permitieran la integración de los distintos componentes de forma sencilla y rápida. 

\newpage