\section{Costos}

\subsection{Suposiciones}
Se calcularán los costos con base en el despliegue de cada arquitectura en AWS.

La arquitectura desplegada en AWS consiste en:

\begin{itemize}
    \item Cluster de EKS (Elastic Kubernetes Service) distribuido en 4 Availability Zones (a, b, c, d)
    \item AZ-a: Doris Backend 1, Task Manager 1, Kafka Broker 1, Zookeeper 1
    \item AZ-b: Doris Backend 2, Task Manager 2, Kafka Broker 2, Zookeeper 2
    \item AZ-c: Doris Backend 3, Task Manager 3, Kafka Broker 3, Zookeeper 3
    \item AZ-d: Doris Frontend, Job Manager, Task Manager 4
    \item Almacenamiento S3 nativo (sustituyendo MinIO)
    \item Interconexión entre zonas de disponibilidad
\end{itemize}

El costo de EKS en AWS es de \$0.10 por hora por cada nodo del cluster.
Por lo que el costo mensual base es de U\$S 72.00. \newline

Se asume además que todos los nodos experimentan una carga constante igual a la mayor carga registrada en las pruebas. 
Esto asegura que se mantenga la disponibilidad del sistema.\newline

Por otro lado, Zookeeper no fue incluido en la comparación de costos operativos ya que no hace una diferencia. 
Sin embargo, será incluido en el costo total de ambas arquitecturas. \newline

Se asume que para todos los nodos, salvo Job Manager y Doris Frontend hay un 25\% de uso de red en la misma zona de disponibilidad y un 75\% de uso de red entre zonas de disponibilidad.
El Job Manager, por el despliegue que definimos, tendrá un 100\% de uso de red en zonas de disponibilidad distintas.
A su vez Doris Frontend tendrá un 100\% de uso de red hacia internet. \newline


\newpage
\subsection{Costos de la Arquitectura Kappa}

Para el caso de Kappa, se asumirá que tanto el almacenamiento como el uso de red medidos corresponden al 61\% del valor real si se hubiera podido completar la carga de datos completa.\newline

De esta forma, se consideran instancias que cumplan estos requisitos. 

El caso más complejo es el de Kafka, que requiere un muy buen acceso a disco y una buena red. Y por otro lado, necesita que si se cae un nodo no se pierdan datos, ya que es nuestra fuente de verdad. \newline

Para cumplir los patrones de almacenamiento vistos en la prueba se propone para Kafka el uso de una instancia r6i.xlarge con 4vCPU y 32GB de RAM,
 que permite asegurar que los nodos no estén sobrecargados. Además se le deberá agregar un volumen persistente EBS de 1TB para asegurar que el nodo pueda almacenar los datos necesarios.\newline

Para el caso de Doris, se utilizará una instancia c5d.xlarge para el frontend (que viene con un disco) y una c5.x2large para el backend ya que requieren un buen uso de CPU y de red. 
Además, el backend debe tener un volumen persistente de al menos 50 GB para asegurar que pueda almacenar los datos necesarios y estos no se pierdan. \newline

Para el caso de Flink, se utilizará una instancia r5.xlarge para los nodos Task Manager  ya que proveen un muy buen rendimiento de memoria y una c5.large para el Job Manager. \newline

Para el caso de Zookeeper, se utilizará una instancia m5.large que asegura un buen funcionamiento de este sistema.  

\newpage

\begin{longtable}{|p{3cm}|p{2cm}|p{2cm}|p{2cm}|p{3cm}|}
    \hline
    \textbf{Componente} & \textbf{Cantidad} & \textbf{Instancia} & \textbf{Costo} & \textbf{Total} \\
    \hline
    Job Manager & 1 & c5d.large & \$69,12 & \$69,12 \\
    \hline
    Task Manager & 4 & r5d.large & \$181,44 & \$725,76 \\
    \hline
    Kafka Broker & 3 & r6i.xlarge & \$144,72 & \$434,16 \\
    \hline
    Zookeeper & 3 & m5.large & \$69,12 & \$207,36 \\
    \hline
    Doris Frontend & 1 & c5d.xlarge & \$144,72 & \$144,72 \\
    \hline
    Doris Backend & 3 & c5.2xlarge & \$244,8 & \$734,4 \\
    \hline
    \caption{Costo de Kappa en instancias} \\
\end{longtable}

\begin{longtable}{|p{3cm}|p{2cm}|p{2cm}|p{2cm}|p{2cm}|p{2cm}|}
    \hline
    \textbf{Componente} & \textbf{Cantidad} & \textbf{Servicio} & \textbf{Tamaño} & \textbf{Costo} & \textbf{Total} \\
    \hline
    Job Manager & 1 & Incluido & 50 GB & \$0 & \$0 \\
    \hline
    Task Manager & 4 & Incluido & 50 GB & \$0 & \$0 \\
    \hline
    Kafka Broker & 3 & EBS gp3 & 1 TB & \$398,72 & \$1196,16 \\
    \hline
    Zookeeper & 3 & Incluido & 50 GB & \$0 & \$0 \\
    \hline
    Doris Frontend & 1 & Incluido & 50 GB & \$0 & \$0 \\
    \hline
    Doris Backend & 3 &  EBS gp3 & 50 GB & \$4 & \$12 \\
    \hline
    S3 & N/A & S3 & 1TB & \$23,55 & \$23,55 \\
    \hline
    \caption{Costo de Kappa en almacenamiento} \\
\end{longtable}

\begin{longtable}{|p{3cm}|p{2cm}|p{2cm}|p{2cm}|p{2cm}|p{2cm}|}
    \hline
    \textbf{Componente} & \textbf{Cantidad} & \textbf{Salida} & \textbf{Entre AZ} & \textbf{Internet} & \textbf{Total} \\
    \hline
    Job Manager & 1 & 90 GB    & \$0.90 & \$0 & \$0.9 \\
    \hline
    Task Manager & 4 & 420 GB  & \$12,6 & \$0 & \$12,6 \\
    \hline
    Kafka Broker & 3 & 420 GB  & \$9,45 & \$0 & \$9,45 \\
    \hline
    Doris Frontend & 1 & 1 GB  & \$0 & \$0.09 & \$0.09 \\
    \hline
    Doris Backend & 3 & 0,2 GB & \$0.0045 & \$0.00 & \$0.0045 \\
    \hline
    S3 & N/A & 700 GB & N/A & N/A & \$7 \\
    \hline
    \caption{Costo de Kappa en uso de red} \\
\end{longtable}

\begin{longtable}{|p{3cm}|>{\raggedleft\arraybackslash}p{11cm}|}
    \hline
    \textbf{Componente} & \textbf{Costo} \\
    \hline
    Instancia & \$2315,52 \\
    \hline
    Almacenamiento & \$1231,71 \\
    \hline
    Red & \$30,04 \\
    \hline
    Total & \textbf{\$3577,27} \\
    \hline
    \caption{Costo de Kappa} \\
\end{longtable}

\newpage


\subsection{Costos de la Arquitectura Delta}

Para el caso de Delta, las instancias son similares a Kappa, aunque su menor uso de Kafka permite reducir el costo en el tipo de instancia.
Por otro lado, como la fuente de verdad es el almacenamiento en S3, tampoco se necesitan tener volúmenes persistentes ni para Kafka ni para Doris.\newline

Para cumplir con estos requisitos se propone el uso de una instancia c5d.xlarge para Kafka que cuenta con 4 vCPU, 8 GB RAM y 100 GB de disco efímero.\newline

Para el resto de las instancias se utilizarán las misma que para Kappa. 
S3 es un caso especial, ya que gracias a la optimización de Paimon con el uso de Parquet se necesita menos espacio para almacenar los datos y 
se pueden reducir los requisitos.

\newpage

\begin{longtable}{|p{3cm}|p{2cm}|p{2cm}|p{2cm}|p{3cm}|}
    \hline
    \textbf{Componente} & \textbf{Cantidad} & \textbf{Instancia} & \textbf{Costo} & \textbf{Total} \\
    \hline
    Job Manager & 1 & c5d.large & \$69,12 & \$69,12 \\
    \hline
    Task Manager & 4 & r5d.large & \$181,44 & \$725,76 \\
    \hline
    Kafka Broker & 3 & c5d.xlarge & \$138,24 & \$414,72 \\
    \hline
    Zookeeper & 3 & m5.large & \$69,12 & \$207,36 \\
    \hline
    Doris Frontend & 1 & c5d.xlarge & \$144,72 & \$144,72 \\
    \hline
    Doris Backend & 3 & c5.2xlarge & \$244,8 & \$734,4 \\
    \hline
    \caption{Costo de Delta en instancias} \\
\end{longtable}

\begin{longtable}{|p{3cm}|p{2cm}|p{2cm}|p{2cm}|p{2cm}|p{2cm}|}
    \hline
    \textbf{Componente} & \textbf{Cantidad} & \textbf{Servicio} & \textbf{Tamaño} & \textbf{Costo} & \textbf{Total} \\
    \hline
    Job Manager & 1 & Incluido & 50 GB & \$0 & \$0 \\
    \hline
    Task Manager & 4 & Incluido & 50 GB & \$0 & \$0 \\
    \hline
    Kafka Broker & 3 & Incluido & 50 GB & \$0 & \$0 \\
    \hline
    Zookeeper & 3 & Incluido & 50 GB & \$0 & \$0 \\
    \hline
    Doris Frontend & 1 & Incluido & 50 GB & \$0 & \$0 \\
    \hline
    Doris Backend & 3 &  Incluido & 50 GB & \$0 & \$0 \\
    \hline
    S3 & N/A & S3 & 50 GB & \$1,15 & \$1,15 \\
    \hline
    \caption{Costo de Delta en almacenamiento} \\
\end{longtable}

\begin{longtable}{|p{3cm}|p{2cm}|p{2cm}|p{2cm}|p{2cm}|p{2cm}|}
    \hline
    \textbf{Componente} & \textbf{Cantidad} & \textbf{Salida} & \textbf{Entre AZ} & \textbf{Internet} & \textbf{Total} \\
    \hline
    Job Manager & 1 & 6 GB    & \$0.06 & \$0 & \$0.06 \\
    \hline
    Task Manager & 4 & 86 GB  & \$2,58 & \$0 & \$2,58 \\
    \hline
    Kafka Broker & 3 & 102 GB  & \$2,295 & \$0 & \$2,295 \\
    \hline
    Doris Frontend & 1 & 1,1 GB  & \$0 & \$0.099 & \$0.099 \\
    \hline
    Doris Backend & 3 & 0,2 GB & \$0.0045 & \$0.00 & \$0.0045 \\
    \hline
    S3 & N/A & 374 GB & N/A & N/A & \$3,74 \\
    \hline
    \caption{Costo de Delta en uso de red} \\
\end{longtable}

\begin{longtable}{|p{3cm}|>{\raggedleft\arraybackslash}p{11cm}|}
    \hline
    \textbf{Componente} & \textbf{Costo} \\
    \hline
    Instancia & \$2288,16 \\
    \hline
    Almacenamiento & \$1,15 \\
    \hline
    Red & \$8,78 \\
    \hline
    Total & \textbf{\$2298,09} \\
    \hline
    \caption{Costo de Delta} \\
\end{longtable}

\newpage

\subsection{Resultados}

El costo de delta es un \textbf{64\%} del costo de Kappa. 
Esto se debe sobre todo a la necesidad que tiene Kappa de que Kafka sea su fuente de verdad.
Y si bien el almacenamiento en capas que ofrece Kafka permite reducir costos al almacenar sus segmentos en Object Storage,
aún así necesita de mucha cantidad de disco para funcionar correctamente en un entorno de alta disponibilidad.\newline

Por su lado, Delta hace un uso extensivo de Object Storage en combinación con el formato Parquet, lo que reduce 
tanto la necesidad de almacenamiento como el tráfico de red; volviéndose una solución mucho más económica.\newline

En cuanto al costo de las instancias, no existen diferencias significativas entre las arquitecturas. 
\newpage