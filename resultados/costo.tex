\section{Costos}

\subsection{Suposiciones}
Se calcularán los costos basados en el despliegue de cada arquitectura en AWS.

La arquitectura desplegada en AWS consiste en:

\begin{itemize}
    \item Cluster de EKS (Elastic Kubernetes Service) distribuido en 4 Availability Zones (a, b, c, d)
    \item AZ-a: Doris Backend 1, Task Manager 1, Kafka Broker 1, Zookeeper 1
    \item AZ-b: Doris Backend 2, Task Manager 2, Kafka Broker 2, Zookeeper 2
    \item AZ-c: Doris Backend 3, Task Manager 3, Kafka Broker 3, Zookeeper 3
    \item AZ-d: Doris Frontend, Job Manager, Task Manager 4
    \item Almacenamiento S3 nativo (sustituyendo MinIO)
    \item Interconexión entre zonas de disponibilidad
\end{itemize}

El costo de EKS en AWS es de \$0.10 por hora por cada nodo del cluster.
Por lo que el costo mensual base es de U\$S 72.00.

Se asume además que todos los nodos experimentan una carga constante igual a la mayor carga registrada en las pruebas. 
Esto asegura que se mantenga la disponibilidad del sistema.

Por otro lado, Zookeeper no fué incluido en la comparación de costos operativos ya que no hace una diferencia. 
Sin embargo, será incluido en el costo total de ambas arquitecturas.

Se asume que el trafico de red entre los nodos de la misma zona de disponibilidad es gratuito. 
Sin embargo, el tráfico entre zonas de disponibilidad tiene un costo de \$0.01 por GB.
Y habiendo 4 zonas de disponibilidad que practicamente simétricas, se asume que se deistribuye el tráfico de forma equitativa entre todas ellas.

Se le dará a cada nodo una instancia de AWS que permita de forma óptima el procesamiento de los datos. 
Además, se asignará un volumen EBS de mínimo 50 GB a cada nodo.

\newpage
\subsection{Costos de la Arquitectura Kappa}

\begin{longtable}{|p{3cm}|c|c|c|c|c|}
    \hline
    \textbf{Componente} & \textbf{Instancia} & \textbf{Base} & \textbf{Almacenamiento} & \textbf{Red} & \textbf{Total} \\
    \hline
    Job Manager & c5.large & \$47 & \$5 & \$0.4 & \$52.4 \\
    \hline
    Task Manager & r5.xlarge & \$137 & \$5 & \$1.7 & \$143.7 \\
    \hline
    Kafka Broker & c5.x2large & \$188 & \$23.4 & \$5.74 & \$217.14 \\
    \hline
    Zookeeper & m5.large & \$69.12 & \$5 & \$1.9 & \$76.02 \\
    \hline
    Doris Frontend & c5.xlarge & \$94 & \$5 & \$0.01 & \$99.01 \\
    \hline
    Doris Backend & c5.x2large & \$188 & \$5 & \$0.05 & \$193.05 \\
    \hline
    Almacenamiento S3 & - & \$0 & \$38.4 & \$2.9 & \$41.3 \\
    \hline
\end{longtable}

El costo total de la arquitectura Kappa es de: 
\newline
\newline
\text{\small\$52.4 + 4 * \$143.7 + 3 * \$217,14 + 3 * \$76.02 + \$99.01 + 3 * \$193.05 + \$41.3 + \$72.00 = 2298.11\\}
\newpage


\subsection{Costos de la Arquitectura Delta}

\begin{longtable}{|p{3cm}|c|c|c|c|c|}
    \hline
    \textbf{Componente} & \textbf{Instancia} & \textbf{Base} & \textbf{Almacenamiento} & \textbf{Red} & \textbf{Total} \\
    \hline
    Job Manager & c5.large & \$47 & \$5 & \$0.4 & \$52.4 \\
    \hline
    Task Manager & r5.xlarge & \$137 & \$5 & \$0.65 & \$142.65 \\
    \hline
    Kafka Broker & c5.xlarge & \$94 & \$3.8 & \$2.30 & \$100,1 \\
    \hline
    Zookeeper & m5.large & \$69.12 & \$5 & \$1.9 & \$76.02 \\
    \hline
    Doris Frontend & c5.xlarge & \$94 & \$5 & \$0.01 & \$99.01 \\
    \hline
    Doris Backend & c5.x2large & \$188 & \$5 & \$0.05 & \$193.05 \\
    \hline
    Almacenamiento S3 & - & \$0 & \$0.84 & \$2.8 & \$3.64 \\
    \hline
\end{longtable}

El costo total de la arquitectura Delta es de:
\newline
\newline
\text{\small\$52.4 + 4 * \$142.65 + 3 * \$100.1 + 3 * \$76.02 + \$99.01 + 3 * \$193.05 + \$3.64 + \$72.00 = 1905.16\\}
\newpage

\subsection{Resultados}

Se puede ver que la arquitectura Delta es más económica que la arquitectura Kappa.
Además, a medida que crecen los datos este diferencial se hace más grande.
La diferencia más grande se da en el uso extensivo de Kafka por parte de Kappa para el ruteo de mensajes. 
Esto se traduce no sólo en un mayor costo de almacenamiento (aún usando almacenamiento en capas) sino también en un mayor costo de procesamiento.