\section{Aspectos Operativos}

Se realizó una prueba de carga de ambas arquitecturas con datos sintéticos de 32 pacientes a lo largo de un año, con un total de 110.122.654 registros; 
que implica un archivo CSV de aproximadamente 7GB.
El hardware donde se realizó la prueba cuenta con 64GB de RAM y 24 núcleos de CPU, y se utilizó Docker Compose para la ejecución de las pruebas.

Variaron no solo los tiempos de carga del total del conjunto de datos, sino también el uso de recursos.

\subsection{Throughput}
El throughput se refiere a la cantidad de datos procesados por unidad de tiempo.
En este caso, se midió en base a los mismos límites en el uso de recursos, la cantidad de datos que podían ser ingestados por el sistema hasta completar la carga total.

Para \textbf{Delta} se pudieron ingestar en un promedio de 1300 registros por segundo; y la carga completa llevó 88438 segundos.
Para \textbf{Kappa} se pudieron ingestar en un promedio de 792 registros por segundo; y la carga del 61\% de los datos llevó 84879 segundos.

La carga de datos en Kappa no pudo completarse completamente ya que en todas las instancias en que se realizó, 
el hardware usado para el despliegue de la arquitectura consumió toda la memoria de su disco duro.  

Se entiende entonces que a nivel de gestión (al menos de disco), Delta es más eficiente que Kappa.
A su vez, a nivel de cantidad de mensajes que se pueden consumir por segundo, Delta fué dos veces más eficiente que Kappa.

\newpage

\subsection{Latencia}

La latencia se refiere al tiempo que tarda un dato en ser procesado por el sistema.
En este caso, se midió en base a los mismos límites en el uso de recursos, la cantidad de tiempo que tardó un dato en pasar por todo el flujo de datos.

La latencia se midió en segundos y los resultados fueron los siguientes:

\begin{longtable}{|p{3cm}|c|c|c|}
    \hline
    \textbf{Arquitectura} & \textbf{Mínima} & \textbf{Máxima} & \textbf{Promedio} \\
    \hline
    Kappa & -1 & 322 & 0.18 \\
    \hline
    Delta & 91 & 610 & 180 \\
    \hline
\end{longtable}

\begin{longtable}{|l|r|r|}
    \hline
    \textbf{Rango (segundos)} & \textbf{Registros} & \textbf{Porcentaje (\%)} \\
    \hline
    \endhead
    90-120 & 12247607 & 64,1047 \\
    \hline
    120-150 & 13210708 & 24,1907 \\
    \hline
    150-180 & 7582878 & 47,3081 \\
    \hline
    180-210 & 7892041 & 49,2370 \\
    \hline
    210-240 & 299165 & 1,8664 \\
    \hline
    240-270 & 16,24 & 8,1E-06 \\
    \hline
    270-300 & 116,86 & 6,9E-05 \\
    \hline
    330-360 & 95,61 & 9,3E-05 \\
    \hline
    390-420 & 31,87 & 6,4E-05 \\
    \hline
    420-450 & 16,24 & 8,1E-06 \\
    \hline
    600-630 & 16,24 & 8,1E-06 \\
    \hline
    \caption{Distribución de Latencia en Delta} \\
\end{longtable}

\begin{longtable}{|l|r|r|}
    \hline
    \textbf{Rango (segundos)} & \textbf{Registros} & \textbf{Porcentaje (\%)} \\
    \hline
    \endhead
    <1 & 9962080 & 81,48 \\
    \hline
    1-10 & 2264616 & 18,52 \\
    \hline
    10-60 & 60 & 0,0005 \\
    \hline
    60-180 & 17 & 0,00014 \\
    \hline
    180-300 & 17 & 0,00014 \\
    \hline
    >300 & 2 & 1,63575E-05 \\
    \hline
    \caption{Distribución de Latencia en Kappa} \\
\end{longtable}

\newpage

Es interesante ver los resultados. Ya que la latencia mínima de Kappa es negativa. 
Esto se debe a que el conector de kafka que se usa, no actualiza el campo \textbf{aggregation\_timestamp} cuando se actualiza el registro porque llegan nuevos datos.
Esto muestra que también muestra un problema de diseño del trabajo de inserción al calcular los datos pero no se encontró una solución al mismo. 
Por otro lado, se registroó el tiempo de latencia de los registros que se insertaron en el sistema; sin embargo, debido al conector de Flink con Doris, 
este proceso se realiza en batch, por lo que nuevamente es posible que el registro de tiempo de inserción sea anterior al de procesamiento. 

Por otro lado, Delta muestra una latencia de varias magnitudes superior a Kafka. Esto es esperable debido al uso extensivo de Object Storage;
ya que las lecturas en disco en conjunto con la transferencia por red suelen ser mucho más lentas que las lecturas en memoria.
Sin embargo, es importante destacar que la latencia máxima de Delta es de 610 segundos, lo que implica que en el peor de los casos,
el sistema tardó 10 minutos en procesar un dato.
Esto es un tiempo aceptable para la mayoría de los casos de uso en los que se implementan flujos de datos; 
aunque no se puede decir que lo sea para un sistema de procesamiento de datos en tiempo real. 

\newpage
\subsection{Uso de Recursos}

El uso de recursos se refiere a la cantidad de recursos que utiliza el sistema para procesar los datos.
En este caso, se midió en base a los mismos límites en el uso de recursos, la cantidad máxima de recursos que utilizó el sistema para procesar los datos.

Para \textbf{Kappa} se presentan los siguientes resultados:

\begin{longtable}{|p{3cm}|c|c|c|c|}
    \hline
    \textbf{Componente} & \textbf{Uso de CPU} & \textbf{Memoria} & \textbf{Almacenamiento} & \textbf{Transferencia} \\
    \hline
    Job Manager & 134\% & 1.8 GB & - & 54 GB \\
    \hline
    Task Manager 1 & 260\% & 6.2 GB & - & 230 GB \\
    \hline
    Task Manager 2 & 244\% & 6.3 GB & - & 252 GB \\
    \hline
    Task Manager 3 & 272\% & 6.2 GB & - & 243 GB \\
    \hline
    Task Manager 4 & 256\% & 6.4 GB & - & 240 GB \\
    \hline
    Kafka 1 & 250\% & 8.2 GB & 234 GB & 255 GB \\
    \hline
    Kafka 2 & 282\% & 8.4 GB & 234 GB & 255 GB \\
    \hline
    Kafka 3 & 268\% & 8.5 GB & 234 GB & 255 GB \\
    \hline
    Doris Frontend & 370\% & 2.4 GB & - & 0.55 GB \\
    \hline
    Doris Backend 1 & 374\% & 4.2 GB & 0.234 GB & 0.1 GB \\
    \hline
    Doris Backend 2 & 345\% & 4.2 GB & 0.234 GB & 0.1 GB \\
    \hline
    Doris Backend 3 & 356\% & 4.3 GB & 0.234 GB & 0.1 GB \\
    \hline
    MinIO & 127\% & 4.2 GB & 384 GB & 386 GB \\
    \hline
\end{longtable}

Para \textbf{Delta} se presentan los siguientes resultados:

\begin{longtable}{|p{3cm}|c|c|c|c|}
    \hline
    \textbf{Componente} & \textbf{Uso de CPU} & \textbf{Memoria} & \textbf{Almacenamiento} & \textbf{Transferencia} \\
    \hline
    Job Manager & 250\% & 3.5 GB & - & 5.9 GB \\
    \hline
    Task Manager 1 & 208\% & 9.7 GB & - & 86.1 GB \\
    \hline
    Task Manager 2 & 166\% & 7.6 GB & - & 45.2 GB \\
    \hline
    Task Manager 3 & 194\% & 8.5 GB & - & 40.2 GB \\
    \hline
    Task Manager 4 & 206\% & 7.5 GB & - & 10.9 GB \\
    \hline
    Kafka 1 & 234\% & 2.7 GB & 38 GB & 101.8 GB \\
    \hline
    Kafka 2 & 274\% & 2.7 GB & 38 GB & 101.9 GB \\
    \hline
    Kafka 3 & 176\% & 2.7 GB & 38 GB & 102.1 GB \\
    \hline
    Doris Frontend & 365\% & 3.6 GB & - & 1.1 GB \\
    \hline
    Doris Backend 1 & 331\% & 4.1 GB & 0 GB & 0.2 GB \\
    \hline
    Doris Backend 2 & 351\% & 4.1 GB & 0 GB & 0.2 GB \\
    \hline
    Doris Backend 3 & 340\% & 4.1 GB & 0 GB & 0.2 GB \\
    \hline 
    MinIO & 127\% & 3.8 GB & 8.4 GB & 374 GB \\
    \hline
\end{longtable}

\newpage

Acorde con lo esperado, Kappa requiere un mayor uso de recursos para los nodos de Kafka que Delta.
Por su lado, Delta requiere un mayor uso de recursos para los nodos de procesamiento. También para el nodo de frontend de Doris. 
Aunque es notablemente menor el uso de memoria que requiere de los nodos Kafka.

El uso de CPU es similar en las dos arquitecturas, aunque Delta tiene más diferencias entre nodos del mismo tipo. 
Probablemente sea debido a un desbalanceo en el paralelismo de los trabajos de procesamiento. 

La sorpresa más grande es el uso de almacenamiento y de transferencia de red. 
Kappa requiere un uso de almacenamiento mucho mayor que Delta, y también un uso de transferencia de red mucho mayor.
Esto se debe a que Delta guarda sus datos en Parquet, lo que lo hace muchísimo más eficiente que el formato de Kafka.
Incluso, en el caso de Delta, todas las transformaciones de los datos están guardadas en Object Storage 
y aún así el uso de almacenamiento es notablementemenor que en Kappa. 

\newpage

\subsection{Resultados}

A nivel de throughput, Delta es más eficiente que Kappa. No solo eso sino que es más fácil de operar y consume menos recursos.
A nivel de latencia, Kappa es más eficiente que Delta. Sin embargo, la latencia máxima de Delta es aceptable para la mayoría de los casos de uso.
En cuanto a uso de memoria y procesamiento, ambas son similares. Aunque Delta es muchísimo mas eficiente en el uso de almacenamiento y transferencia de red. 

\newpage