\section{Aspectos Operativos}

Se realizó una prueba de carga de ambas arquitecturas con datos sintéticos de 32 pacientes a lo largo de un año, con un total de 110.122.654 registros; 
que implica un archivo CSV de aproximadamente 7GB.
El hardware donde se realizó la prueba cuenta con 64GB de RAM y 24 núcleos de CPU, y se utilizó Docker Compose para la ejecución de las pruebas.

Variaron no solo los tiempos de carga del total del conjunto de datos, sino también el uso de recursos.


\subsection{Throughput}
El throughput se refiere a la cantidad de datos procesados por unidad de tiempo.
En este caso, se midió en base a los mismos límites en el uso de recursos, la cantidad de datos que podían ser ingestados por el sistema hasta completar la carga total.

Para \textbf{Delta} se pudieron ingestar en un promedio de 1300 registros por segundo; y la carga completa llevó alrededor de 23 horas y media.
Para \textbf{Kappa} se pudieron ingestar en un promedio de X registros por segundo; y la carga completa llevó alrededor de Y horas.

Es importante destacar que la carga de Kappa debió generó problemas de rendimiento en el sistema ya que hubieron episodios en los que el disco fué totalmente llenado. 
Si bien esto ocurre por ser un sistema compartido con otras aplicaciones, esto no fué el caso para Delta, que no tuvo problemas de rendimiento.
Se entiende entonces que a nivel de gestión de disco al menos, Delta es más eficiente que Kappa. 

\subsection{Latencia}

La latencia se refiere al tiempo que tarda un dato en ser procesado por el sistema.
En este caso, se midió en base a los mismos límites en el uso de recursos, la cantidad de tiempo que tardó un dato en pasar por todo el flujo de datos.

Para \textbf{Delta} se presentan los siguientes resultados: 

\begin{itemize}
    \item \textbf{Latencia Mínima:} 91 segundos
    \item \textbf{Latencia Máxima:} 610 segundos
    \item \textbf{Latencia Promedio:} 180 segundos
    \item \textbf{Distribución de la Latencia:} en el 96.7\% de los casos la latencia se encontró entre 150 y 210 segundos. 
    Menos de un 2\% de registros mostraron latencia entre 210 y 240 segundos. 
    Sólo 26 registros tuvieron latencia por encima de 240 segundos y nunca mayores a 630 segundos 
\end{itemize}


Para \textbf{Kappa} se presentan los siguientes resultados: 

\begin{itemize}
    \item \textbf{Latencia Mínima:} 10 milisegundos
    \item \textbf{Latencia Máxima:} 800 milisegundos
    \item \textbf{Latencia Promedio:} 200 milisegundos
    \item \textbf{Distribución de la Latencia:} 
\end{itemize}

En términos de Latencia hay una clara diferencia entre ambos sistemas, siendo Kappa mucho más rápido que Delta.
Esto se debe en mayor medida, a las características que tiene Delta en el uso de Object Storage; mientras que Kappa está diseñado para trabajar con datos en tiempo real.
De esta manera queda claro que si el escenario que se maneja requiere latencias menores a un minuto Kappa es la mejor opción. 
Y si las necesidades son de latencias menores al segundo Kappa es la únicas opción posible.

De hecho, con estos resultados no se puede decir que Delta sea una opción válida para procesamiento de datos en tiempo real. 
Sin embargo, es importante destacar que la mayoría de los casos de uso en los que se implementan flujos de datos si bien requieren procesamiento contínuo, 
menos de cinco minutos de latencia son una característica más que aceptable.

\newpage
\subsection{Uso de Recursos}

El uso de recursos se refiere a la cantidad de recursos que utiliza el sistema para procesar los datos.
En este caso, se midió en base a los mismos límites en el uso de recursos, la cantidad máxima de recursos que utilizó el sistema para procesar los datos.

Para \textbf{Delta} se presentan los siguientes resultados:

\begin{longtable}{|p{3cm}|c|c|c|c|}
    \hline
    \textbf{Componente} & \textbf{Uso de CPU} & \textbf{Memoria} & \textbf{Almacenamiento} & \textbf{Transferencia} \\
    \hline
    Job Manager & 250\% & 3.5 GB & - & 5.9 GB \\
    \hline
    Task Manager 1 & 208\% & 9.7 GB & - & 86.1 GB \\
    \hline
    Task Manager 2 & 166\% & 7.6 GB & - & 45.2 GB \\
    \hline
    Task Manager 3 & 194\% & 8.5 GB & - & 40.2 GB \\
    \hline
    Task Manager 4 & 206\% & 7.5 GB & - & 10.9 GB \\
    \hline
    Kafka 1 & 234\% & 2.7 GB & 38 GB & 101.8 GB \\
    \hline
    Kafka 2 & 274\% & 2.7 GB & 38 GB & 101.9 GB \\
    \hline
    Kafka 3 & 176\% & 2.7 GB & 38 GB & 102.1 GB \\
    \hline
    Doris Frontend & 365\% & 3.6 GB & - & 1.1 GB \\
    \hline
    Doris Backend 1 & 331\% & 4.1 GB & 0 GB & 0.2 GB \\
    \hline
    Doris Backend 2 & 351\% & 4.1 GB & 0 GB & 0.2 GB \\
    \hline
    Doris Backend 3 & 340\% & 4.1 GB & 0 GB & 0.2 GB \\
    \hline
\end{longtable}


Por su lado para \textbf{Kappa} se presentan los siguientes resultados:

\begin{longtable}{|p{3cm}|c|c|c|c|}
    \hline
    \textbf{Componente} & \textbf{Uso de CPU} & \textbf{Memoria} & \textbf{Almacenamiento} & \textbf{Transferencia} \\
    \hline
    Job Manager & X\% & X GB & - & X GB \\
    \hline
    Task Manager 1 & X\% & X GB & - & X GB \\
    \hline
    Task Manager 2 & X\% & X GB & - & X GB \\
    \hline
    Task Manager 3 & X\% & X GB & - & X GB \\
    \hline
    Task Manager 4 & X\% & X GB & - & X GB \\
    \hline
    Kafka 1 & 234\% & X GB & X GB & X GB \\
    \hline
    Kafka 2 & 274\% & X GB & X GB & X GB \\
    \hline
    Kafka 3 & 176\% & X GB & X GB & X GB \\
    \hline
    Doris Frontend & X\% & X GB & - & X GB \\
    \hline
    Doris Backend 1 & X\% & X GB & X GB & X GB \\
    \hline
    Doris Backend 2 & X\% & X GB & X GB & X GB \\
    \hline
    Doris Backend 3 & X\% & X GB & X GB & X GB \\
    \hline
\end{longtable}

Es claro que Kafka tiene una gran incidencia en el consumo de recursos. Teniendo para ambos casos muchísimo uso de CPU, RAM y uso de red. 
El uso de disco tampoco es un elemento despreciable a considerar, en especial para Kappa que necesita tener sus nodos de Kafka siempre a máxima potencia 
y son estos los que presentan el mayor desafío para la escalabilidad del sistema debido a que si no se configuran las politicas de retención apropiadamente, 
los costos pueden crecer desmesuradamente.



\newpage