\section{Desafíos en Arquitecturas de Streaming}
Diseñar e implementar sistemas de procesamiento de datos en tiempo real conlleva varios desafíos técnicos y operativos. A continuación se presentan los más relevantes para arquitecturas de streaming como Kappa y Delta.

\subsection{Procesamiento Fuera de Orden}
En sistemas distribuidos, los eventos pueden llegar desordenados debido a retrasos de red o procesamiento. Para mitigar este problema, se utilizan:
\begin{itemize}
    \item \textbf{Ventanas con tolerancia al desorden} (\textit{allowed lateness})
    \item \textbf{Algoritmos tolerantes al orden de llegada}
\end{itemize}
Ambas soluciones incrementan la latencia del sistema, por lo que se debe alcanzar un equilibrio entre precisión y velocidad.

\subsection{Manejo de Archivos Pequeños}
El procesamiento continuo genera muchos archivos pequeños, lo que reduce la eficiencia de lectura/escritura y afecta negativamente el rendimiento. Para resolverlo se aplican técnicas como:
\begin{itemize}
    \item Compactación periódica
    \item Uso de formatos columnar como Parquet
    \item Estrategias de particionado y organización eficientes
\end{itemize}

\subsection{Gestión del Estado}
El manejo de estado distribuido es clave para mantener agregaciones, ventanas y cálculos históricos. Requiere:
\begin{itemize}
    \item \textbf{Checkpointing confiable} para tolerancia a fallos
    \item \textbf{Escalabilidad del backend de estado} (e.g. RocksDB)
    \item \textbf{Estrategias de limpieza} para evitar crecimiento no controlado
\end{itemize}

\subsection{Backpressure}
Cuando el sistema no puede procesar datos tan rápido como los recibe, se produce backpressure. Un buen diseño debe:
\begin{itemize}
    \item Detectar desequilibrios entre fuentes y sinks
    \item Ajustar dinámicamente el ritmo de procesamiento
    \item Monitorear métricas como tiempo en cola y uso de buffers
\end{itemize}

\subsection{Reprocesamiento}
Reprocesar datos históricos por errores o nuevas reglas de negocio requiere:
\begin{itemize}
    \item \textbf{Persistencia de eventos crudos}
    \item \textbf{Mecanismos de replay controlado}
    \item \textbf{Separación de recursos} para evitar degradar el procesamiento en línea
\end{itemize}

\subsection{Conformidad Normativa}
Cumplir con normativas como GDPR o HIPAA impone requerimientos estrictos:
\begin{itemize}
    \item \textbf{Cifrado en tránsito y en reposo}
    \item \textbf{Gestión de consentimiento y derecho al olvido}
    \item \textbf{Trazabilidad y auditoría de accesos}
\end{itemize}

\medskip
Estos desafíos deben abordarse desde la arquitectura, el diseño de datos y las políticas operativas para garantizar un sistema robusto, escalable y confiable.

\newpage