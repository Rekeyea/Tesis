\subsection{Mensajería Distribuida}

Las tecnologias de mensajería distribuidas en tiempo real cumplen el crucial rol de actuar como intermediarios entre las fuentes de datos 
y los sistemas que efectivamente procesan estos datos. \parencite{kleppmann} 

Deben funcionar como un conducto de alta capacidad de almacenamiento y baja latencia, capturando y canalizando los flujos de información
desde sus múltiples orígenes y hacia sus diversos destinos en tiempo real. \parencite{bigdata}

Su papel es fundamentalmente el de un sistema nervioso central, coordinando y distribuyendo datos a través de complejas arquitecturas distribuidas. 
Actúan como amortiguadores, absorbiendo picos en el flujo de datos y garantizando un procesamiento constante y eficiente. 
Además, estas tecnologías sirven como una capa de abstracción, desacoplando los productores de datos de los consumidores, 
lo que permite una mayor flexibilidad y escalabilidad en el diseño del sistema.

\subsubsection{Apache Kafka}
Apache Kafka es el estandar de facto de este tipo de sistemas. Utiliza un modelo de publicación-subscripcion (pub/sub) basado en logs, 
donde los datos se envían a "topics" y se almacenan en particiones distribuidas. Las particiones tienen una garantía de orden de los mensajes
y permiten la retención de datos a largo plazo.

Además, proporciona conectores para integración con diversos sistemas y un amplio ecosistema. A nivel de seguridad ofrece encriptación
en tránsito mediante TLS y permite ser configurado para soportar encriptación en reposo (aunque esto debe hacerse a nivel de sistema de archivos).

Por último, su arquitectura distribuida y replicada permite una alta disponibilidad y tolerancia a fallos.

\newpage
\subsubsection{Apache Pulsar}
Apache Pulsar es también una plataforma de mensajería y streaming distribuida, al igual que Apache Kafka, pero que se distingue por tener una 
arquitectura basada en capas, separando la capa de almacenamiento de la capa de procesamiento; lo que permite escalar cada uno independientemente.

Apache Pulsar soporta modelos de entrega como colas, publicación y suscripción y puede ofrecer garantías de entregar un mensaje 
exactamente una vez ("exactly-once"). Ofrece encriptación a nivel de mensaje, lo que permite una granularidad fina en cuanto a que encriptar. 
También soporta TLS para la encriptación en tránsito y permite la configuración de encriptación en reposo.

Por último, también soporta almacenamiento de mensajes a largo plazo y soporte nativo para esquemas:
Esto es, permite definir la estructura y el tipo de datos de los mensajes, lo que a su vez permite una validación automática de los datos 
y una serialización/deserialización más eficiente. Esto trae consigo además, la capacidad de evolucionar estos esquemas de mensajes,
de forma que productores y consumidores evolucionen independientemente.

\subsubsection{Amazon Kinesis}
Amazon Kinesis es un servicio de streaming de datos administrado en la nube de AWS. 
Está diseñado para recopilar, procesar y analizar datos de streaming en tiempo real a gran escala.
Kinesis, en realidad, se compone de varios servicios:
\begin{enumerate}
    \item Kinesis Data Streams para ingestión de datos en tiempo real
    \item Kinesis Data Firehose para cargar datos en los servicios de almacenamiento disponibles de AWS
    \item Kinesis Data Analytics para procesar datos con SQL o Java
    \item Kinesis Video Streams para streaming de video  
\end{enumerate}
Adicionalmente ofrece capacidades de auto-escalado, replicación entre zonas de disponibilidad para alta durabilidad, encriptación en reposo 
(y puede habilitarsele la encriptación en tránsito) y permite la retención de datos hasta 365 días.

Como se puede ver, al ser tan completo permitiría implementar, al menos en principio, una gran parte de un sistema de Big Data en tiempo real.

\subsubsection{Azure Event Hubs}
Azure Event Hubs es un servicio de ingestión de datos en tiempo real administrado en la plataforma Microsoft Azure.
Se supone que está diseñado para soportar millones de eventos por segundo con baja latencia. Al igual que Kinesis,
ofrece una muy buena con otros servicios de Microsoft Azure, lo que permite por ejemplo capturar directamente los eventos
en los servicios de almacenamiento disponibles en este proveedor de nube.

Event Hubs es compatible con el protocolo Kafka, lo que permite a las aplicaciones existentes de Kafka conectarse sin cambios de código.
Ofrece una retención de mensajes por defecto de 1 dia que puede aumentado hasta 7. En caso de necesitar una retención más a largo plazo
se recomienda guardar los eventos en Azure Blob Storage o Azure Data Lake para su posterior procesamiento. 
Cuenta tambien con encriptación en transito con TLS y en reposo.

Proporciona además, características como el procesamiento batch para optimizar el rendimiento, 
control de acceso basado en roles, y encriptación en reposo y en tránsito. 

\subsubsection{Comparación}

\paragraph{Escalabilidad}
\begin{itemize}
    \item \textbf{Apache Kafka:} Muy Alta escalabilidad horizontal, millones de mensajes/segundo.
    \item \textbf{Apache Pulsar:} Muy alta escalabilidad horizontal, millones de mensajes/segundo con separación de almacenamiento y cómputo.
    \item \textbf{Amazon Kinesis:} Buena escalabilidad, con 1.000 mensajes por segundo con la configuración por defecto aunque con configuración adicional puede llegar al millon por segundo hipotético.
    \item \textbf{Azure Event Hubs:} Buena escalabilidad, con 1.000 mensajes por segundo. También puede escalar con configuración adicional a los 20.000 mensajes. Existe la posibilidad de tener una instancia dedicada que permite escalar a millones de eventos por segundo de forma hipotética.
\end{itemize}

\newpage
\paragraph{Retención de datos}
\begin{itemize}
    \item \textbf{Apache Kafka:} Configurable, potencialmente ilimitada.
    \item \textbf{Apache Pulsar:} Ilimitada por diseño.
    \item \textbf{Amazon Kinesis:} Hasta 365 días, configurable.
    \item \textbf{Azure Event Hubs:} Hasta 7 días, opción de Capture para largo plazo.
\end{itemize}

\paragraph{Garantías de entrega}
\begin{itemize}
    \item \textbf{Apache Kafka:} At-least-once por defecto, exactly-once configurable.
    \item \textbf{Apache Pulsar:} Exactly-once nativo.
    \item \textbf{Amazon Kinesis:} At-least-once.
    \item \textbf{Azure Event Hubs:} At-least-once.
\end{itemize}

\paragraph{Encriptación}
\begin{itemize}
    \item \textbf{En reposo:}
    \begin{itemize}
        \item \textbf{Apache Kafka:} Configurable.
        \item \textbf{Apache Pulsar:} Nativo.
        \item \textbf{Amazon Kinesis:} Por defecto (AWS KMS).
        \item \textbf{Azure Event Hubs:} Por defecto.
    \end{itemize}
    \item \textbf{En tránsito:} Todos soportan TLS/SSL.
\end{itemize}

\paragraph{Observaciones clave}
\begin{itemize}
    \item Pulsar destaca en escalabilidad, retención, seguridad y consistencia nativos.
    \item Kafka ofrece el ecosistema más maduro y desarrollado que le otorga características muy completas.
    \item Servicios gestionados (Kinesis, Event Hubs) tienen encriptación en reposo por defecto, pero retención limitada.
\end{itemize}
\clearpage
\begin{table}[h]
\footnotesize
\begin{tabular}{|p{3cm}|p{2cm}|p{2cm}|p{2cm}|p{2cm}|}
\hline
\textbf{Característica} & \textbf{Kafka} & \textbf{Pulsar} & \textbf{Kinesis} & \textbf{Event Hubs} \\
\hline
    Escalabilidad & Muy Alta & Muy alta & Alta & Alta \\
\hline
Retención & Configurable & Ilimitada & 365 días máx. & 7 días máx. \\
\hline
Garantía de entrega & Exactly-once & Exactly-once & At-least-once & At-least-once \\
\hline
Encriptación en reposo & Configurable & Sí & Sí (AWS KMS) & Sí \\
\hline
Encriptación en tránsito & Sí & Sí & Sí & Sí \\
\hline
Throughput & {Muy alto} & {Muy alto} & {Alto} & {Alto} \\
\hline
\end{tabular}
\caption{Comparación de Sistemas de Mensajería}
\label{tab:comp_mensajeria}
\end{table}
\clearpage
\newpage