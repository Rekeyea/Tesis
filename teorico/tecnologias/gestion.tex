\subsection{Tecnologías Complementarias}

Consideramos tecnologías complementarias, aquellas que no son directamente parte de la arquitectura, pero que son necesarias para su correcto funcionamiento.\newline
Estas tecnologías son sobre todo referentes a la gestión de contenedores, monitorización y visualización de datos.

\subsubsection{Docker}

Docker es una plataforma de software que permite crear, desplegar y ejecutar aplicaciones en contenedores. 
Los contenedores son entornos ligeros y portátiles que encapsulan una aplicación y todas sus dependencias, lo que garantiza que funcionen de manera consistente en diferentes entornos.\newline

A nivel de desarrollo, también permite crear imágenes de contenedores que pueden ser compartidas y reutilizadas, facilitando la colaboración entre equipos y la integración continua.
Es fácil de usar y provee Docker Compose, una herramienta que permite definir y ejecutar aplicaciones multi-contenedor.

\subsubsection{Prometheus}
Prometheus es un sistema de monitorización y alerta open-source, Es muy eficaz para monitorizar entornos dinámicos como los que se encuentran en arquitecturas de microservicios y contenedores.
Permite recolectar y analizar métricas en tiempo real de manera altamente eficiente y tiene un potente lenguaje de consultas. \newline

Destaca por su capacidad para manejar millones de métricas simultáneamente y su arquitectura pull-based, que le permite escalar horizontalmente sin problemas. 
Además, cuenta con un sistema de alertas flexible y puede integrarse fácilmente con Docker y otras herramientas de orquestación, 
lo que lo convierte en una herramienta fundamental para monitorizar aplicaciones modernas a gran escala.

\newpage

\subsubsection{Grafana}
Grafana es una plataforma de visualización y análisis de datos de open-source que destaca cuando se combina con Prometheus. 
Su principal valor radica en su capacidad para transformar datos complejos en visualizaciones claras e interactivas a través de tableros personalizables.\newline

Cuando se utiliza junto con Prometheus, Grafana actúa como la capa de visualización, permitiendo crear paneles intuitivos que muestran en tiempo real el estado y rendimiento de los sistemas. 
Esto facilita la detección temprana de problemas, el análisis de tendencias y la toma de decisiones basada en datos. \newline

Una de sus características más poderosas es la capacidad de combinar datos de múltiples fuentes en un único dashboard, proporcionando una vista unificada del sistema. 

\newpage