\subsection{Tecnologías Complementarias}

\subsubsection{Kubernetes}

Kubernetes es una plataforma open-source diseñada para automatizar la implementación, el escalado y la gestión de aplicaciones en contenedores. 
Su principal fortaleza radica en su capacidad de escalado: puede escalar automáticamente las aplicaciones según la demanda en tiempo real, 
añadiendo o eliminando contenedores según sea necesario. 
Esto significa que puede aumentar los recursos cuando hay picos de tráfico y reducirlos en momentos de baja actividad, optimizando así los costos de infraestructura. 
Además, maneja automáticamente la distribución de carga entre contenedores, la recuperación ante fallos y actualizaciones continuas sin tiempo de inactividad, 
lo que lo hace especialmente valioso para aplicaciones que requieren alta disponibilidad y escalabilidad como las necesarias para Big Data.

\subsubsection{Prometheus}
Prometheus es un sistema de monitorización y alerta open-source, Es especialmente eficaz para monitorizar entornos dinámicos como los que se encuentran en arquitecturas de microservicios y contenedores.
Permite recolectar y analizar métricas en tiempo real de manera altamente eficiente y tiene un potente lenguaje de consultas. 
Destaca por su capacidad para manejar millones de métricas simultáneamente y su arquitectura pull-based, que le permite escalar horizontalmente sin problemas. 
Además, cuenta con un sistema de alertas flexible y puede integrarse fácilmente con Kubernetes y otras herramientas de orquestación, 
lo que lo convierte en una herramienta fundamental para monitorizar aplicaciones modernas a gran escala.

\subsubsection{Grafana}
Grafana es una plataforma de visualización y análisis de datos de open-source que destaca especialmente cuando se combina con Prometheus. 
Su principal valor radica en su capacidad para transformar datos complejos en visualizaciones claras e interactivas a través de tableros personalizables.
Cuando se utiliza junto con Prometheus, Grafana actúa como la capa de visualización, permitiendo crear paneles intuitivos que muestran en tiempo real el estado y rendimiento de los sistemas. 
Esto facilita la detección temprana de problemas, el análisis de tendencias y la toma de decisiones basada en datos. 
Una de sus características más poderosas es la capacidad de combinar datos de múltiples fuentes en un único dashboard, proporcionando una vista unificada del sistema. 

\subsubsection{K6}
K6 es una herramienta open-source diseñada para realizar pruebas de rendimiento y carga con enfoque en desarrolladores, permitiendo escribir pruebas de carga en JavaScript.
Puede exportar sus métricas directamente a Prometheus y visualizarlas en Grafana, creando así un ecosistema completo de pruebas y monitorización. 
Esto permite no solo ejecutar pruebas de carga, sino también analizar los resultados en tiempo real y mantener un histórico del rendimiento del sistema. 

\subsubsection{Apache Airflow}

Apache Airflow es una plataforma open-source de orquestación que permite programar, ejecutar y monitorizar flujos complejos de tareas. 
En el contexto de Big Data, destaca especialmente por su capacidad para automatizar tareas de mantenimiento y procesos de manera robusta y programable. 
Su modelo de programación está basado en DAGs (Directed Acyclic Graphs), lo que permite definir dependencias claras entre tareas y asegurar que se ejecuten en el orden correcto.
Una de sus mayores fortalezas es su capacidad para automatizar tareas rutinarias como la limpieza de datos antiguos, compactación de tablas, rebalanceo de particiones y regeneración de índices. 
Airflow puede orquestar estas tareas entre diferentes sistemas, manejando automáticamente los reintentos cuando las tareas fallan 
y permitiendo programar tareas condicionales basadas en el éxito de tareas previas.