\section{Arquitecturas de Referencia}

Una arquitectura de referencia representa una plantilla abstracta y probada que encapsula las decisiones arquitectónicas fundamentales de un sistema, 
mejores prácticas y experiencias acumuladas en un dominio específico. \newline

Esta proporciona un vocabulario común, además de componentes estandarizados y patrones de interacción que sirven como base para el desarrollo de los sistemas concretos. \newline

La arquitectura de referencia no solo define la estructura y comportamiento base del sistema, sino que también establece los principios de diseño, 
restricciones técnicas y mecanismos de extensibilidad que guiarán estas implementaciones.
\newpage

\subsection{Instancias de Arquitectura}

Una plataforma que soporte el Monitoreo Remoto de Pacientes, necesita soportar grandes volúmenes de datos analizados como streaming.
Además, para poder dar soporte al uso de la plataforma se requiere poder analizar el histórico de dichos datos. \newline

Por último, el tiempo en que el análisis de los datos de streaming es disponibilizado debe ser lo más cercano a tiempo real para poder tomar 
decisiones a tiempo para la salud del paciente. \newline

Existen tres grandes familias de arquitecturas de referencia que cubren estos tres casos: 
\begin{itemize}
    \item Lambda
    \item Kappa
    \item Delta
\end{itemize}

De entre ellas, se instanciarán y se compararán Kappa y Delta; ya que son evoluciones propuestas sobre Lambda que siguen distintos caminos para alcanzar el mismo objetivo. \newline
 
Realizar esta instanciación implica seleccionar las tecnologías que se usarán para cumplir las condiciones de la arquitectura de referencia; 
así como también componentes que si bien no son descritos por la arquitectura de referencia, son necesarios para cumplir con las características de calidad
necesarias para el caso de uso propuesto.\newline

Por último, las dos instancias de arquitectura implementadas utilizarán las mismas tecnologías, o tanto como sea posible, de modo que los resultados sean comparables.  

\newpage

\subsection{Nomenclatura de Diagramas}

En los diagramas de arquitectura presentados en este trabajo se utiliza una nomenclatura consistente para facilitar la comprensión y comparación entre las distintas propuestas.

\begin{itemize}
    \item Los rectángulos con esquinas redondeadas representan componentes de procesamiento activo (motores y servicios).
    \item Los rectángulos con esquinas rectas simbolizan las capas o secciones lógicas.
    \item Los círculos representan las fuentes de datos.
    \item Las flechas continuas indican flujo de datos entre capas o componentes.
    \item Los rectángulos con líneas discontínuas representan zonas de disponibilidad en AWS.
    \item Los rombos representan los sistemas que realizan consultas sobre los
\end{itemize}
 
Esta nomenclatura se mantiene constante en todos los diagramas para permitir identificar rápidamente las diferencias estructurales entre las arquitecturas analizadas.
\newpage