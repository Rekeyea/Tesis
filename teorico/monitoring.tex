\section{Monitoreo Remoto de Pacientes}

Los Sistemas de Monitorización Remota de Pacientes (RPM, por sus siglas en inglés) constituyen un paradigma tecnológico de salud en el área de la Telemedicina 
que permite la adquisición, transmisión y análisis, idealmente en tiempo real, de datos fisiológicos del paciente fuera de los entornos clínicos tradicionales, 
mediante una red de dispositivos médicos y sensores de dispositivos inteligentes. \newline

Este enfoque contribuye a mejores resultados para los pacientes, disminuye costos para las instituciones de salud y permite dar un acceso más 
generalizado a los servicios médicos. Es especialmente para el seguimiento de condiciones pre-existente,
población anciana y monitoreo luego de intervenciones quirúrgicas.\parencite{rpm_iot}


\subsection{Monitoreo de Signos Vitales}

\subsubsection{Frecuencia Respiratoria}
\begin{itemize}
    \item Número de ciclos respiratorios (inspiración/espiración) por minuto
    \item Valores normales adulto: 12-20 respiraciones/min
\end{itemize}

\subsubsection{Saturación de Oxígeno}
\begin{itemize}
    \item Porcentaje de hemoglobina unida a oxígeno en sangre arterial
    \item Valores normales: {95-100\%}
\end{itemize}

\subsubsection{Presión Sistólica}
\begin{itemize}
    \item Presión máxima ejercida por la sangre sobre las paredes arteriales durante la sístole
    \item Valores normales: 90-120 mmHg
    \item 
\end{itemize}

\subsubsection{Frecuencia Cardíaca}
\begin{itemize}
    \item Número de contracciones cardíacas por minuto
    \item Valores normales adulto: 60-100 latidos/min
\end{itemize}

\subsubsection{Temperatura}
\begin{itemize}
    \item Medida del calor corporal
    \item Valores normales: 36.5-37.5°C
\end{itemize}

\subsubsection{Escala Glasgow}
\begin{itemize}
    \item Escala neurológica que evalúa nivel de consciencia
    \item Evalúa la apertura ocular, la respuesta verbal y la respuesta motora en distintos rangos
    \item Valores normales: 15
    \item Es difícil de automatizar
\end{itemize}

\subsubsection{Nivel de Conciencia}
\begin{itemize}
    \item Sistema simplificado de evaluación del estado de consciencia utilizado en valoración inicial y monitoreo
    \item Utiliza el sistema APVU: Alerta, Respuesta a estímulos verbales, Respuesta a estímulos dolorosos, Sin respuesta
    \item Valores normales: 0
    \item Al ifual que la escala Glasgow es dificil de automatizar
\end{itemize}
\newpage
\subsection{Identificación de Riesgo en Pacientes}

En un entorno hospitalario, la monitorización de los signos vitales constituye un pilar fundamental en la evaluación del estado clínico de un paciente. 
Estos parámetros fisiológicos esenciales incluyen la presión arterial (PA), la saturación de oxígeno en sangre (SpO2), la temperatura corporal (T), 
la frecuencia cardíaca (FC) y la frecuencia respiratoria (FR), los cuales son registrados sistemáticamente en intervalos de 4 a 6 horas 
como parte del protocolo estándar de vigilancia para detectar posibles deterioros en la condición del paciente.\newline

En diversos establecimientos sanitarios a nivel global, el personal médico y de enfermería implementa metodologías estandarizadas de evaluación, 
conocidas como sistemas de alerta temprana (SAT). 
Estos sistemas utilizan algoritmos validados que asignan puntuaciones específicas a las desviaciones de los rangos normales de los signos vitales, 
permitiendo la activación de alertas cuando se detectan patrones que indican un deterioro clínico. \newline

Esta práctica sistemática facilita la identificación a tiempo de pacientes en riesgo y permite la intervención terapéuticas a tiempo, 
contribuyendo significativamente a la reducción de eventos adversos y a la optimización de los resultados clínicos.\newline

Existen diferentes estándares para la detección, muchos dependientes del contexto de la unidad donde se atienda al paciente. \parencite{rpm_pm}

\newpage

\subsubsection{MEWS}
MEWS (Modified Early Warning Score) es un sistema de puntuación fisiológica validado para la detección temprana del deterioro clínico en pacientes hospitalizados, 
que evalúa cinco parámetros vitales fundamentales: frecuencia respiratoria, frecuencia cardíaca, presión arterial sistólica, temperatura 
y nivel de consciencia. \newline

Cada parámetro recibe una puntuación de 0 a 3 según la gravedad de su alteración, siendo 0 el valor normal y 3 el más patológico; 
La suma total de estos valores genera una puntuación que oscila entre 0 y 14, categorizando el riesgo del paciente en bajo (0--1), medio (2--3), alto (4--5) o crítico ($\geq$6), 
lo que determina la frecuencia de monitorización necesaria y las intervenciones requeridas, 
desde una vigilancia rutinaria cada 8-12 horas en puntuaciones bajas hasta la activación inmediata del equipo de respuesta rápida y posible traslado a UCI en puntuaciones críticas.

\subsubsection{NEWS2}
El NEWS2 (National Early Warning Score 2) es una versión mejorada y actualizada del sistema de alerta temprana, adoptado como estándar por el Servicio Nacional de Salud del Reino Unido, 
que evalúa siete parámetros fisiológicos: frecuencia respiratoria (3-0 puntos), saturación de oxígeno (con dos escalas distintas según el riesgo de insuficiencia respiratoria hipercápnica, 3-0 puntos), 
uso de oxígeno suplementario (2 puntos si requiere), temperatura (3-0 puntos), presión arterial sistólica (3-0 puntos), frecuencia cardíaca (3-0 puntos) 
y nivel de consciencia utilizando la escala ACVPU.\newline

La puntuación total varía de 0 a 20, estratificando el riesgo en bajo (0-4), medio (5-6), alto (7 o más, o cualquier parámetro individual con puntuación de 3) y determinando la respuesta clínica necesaria, 
desde monitorización estándar hasta evaluación urgente por equipo de cuidados críticos, representando una mejora significativa respecto al MEWS al incluir la saturación de oxígeno 
y la confusión como nuevo nivel de consciencia.

\newpage

\subsubsection{SOFA}
El SOFA (Sequential Organ Failure Assessment Score) es un sistema de puntuación diseñado para evaluación diaria (cada 24 horas) de la disfunción/fallo multiorgánico en unidades de cuidados intensivos, 
evaluando seis sistemas orgánicos: respiratorio (mediante la relación PaO2/FiO2 evaluada con cada gasometría, 0-4 puntos), 
cardiovascular (mediante presión arterial media y requerimiento de vasopresores monitorizados continuamente, 0-4 puntos), hepático (mediante bilirrubina sérica medida diariamente, 0-4 puntos), 
coagulación (mediante recuento plaquetario diario, 0-4 puntos), renal (mediante creatinina sérica diaria o gasto urinario horario, 0-4 puntos) 
y neurológico (mediante la escala de Glasgow evaluada cada 4 horas o con cambios clínicos, 0-4 puntos).\newline

Cada sistema recibe una puntuación de 0 (normal) a 4 (máxima disfunción), con una puntuación total que varía de 0 a 24 puntos, calculándose cada 24 horas o antes si hay deterioro clínico significativo, 
siendo especialmente relevante el cambio en la puntuación a lo largo del tiempo.

\newpage
\subsubsection{qSOFA}
El qSOFA (quick Sequential Organ Failure Assessment) es una versión simplificada del SOFA, diseñada para la identificación rápida de pacientes con sospecha de sepsis y alto riesgo de mortalidad fuera de la UCI, 
evaluando únicamente tres parámetros clínicos que se pueden medir de manera inmediata a pie de cama, sin necesidad de pruebas de laboratorio: 
alteración del estado mental (escala de Glasgow $\leq$13 puntos, 1 punto), frecuencia respiratoria elevada ($\geq$22 respiraciones/minuto, 1 punto) y presión arterial sistólica baja ($\leq$100 mmHg, 1 punto).\newline

La puntuación total varía de 0 a 3 puntos, donde una puntuación $\geq$2 indica alto riesgo de mortalidad y la necesidad de evaluación más exhaustiva, monitorización estrecha
y consideración de traslado a un nivel superior de cuidados; el qSOFA debe reevaluarse con cada valoración del paciente o ante cualquier cambio en su estado clínico, 
típicamente cada 1-2 horas en pacientes inestables o con sospecha de sepsis, siendo una herramienta especialmente útil en servicios de urgencias, plantas de hospitalización y entornos extrahospitalarios.

\newpage

\subsection{Desafíos}

Los sensores empleados en la monitorización remota permiten un monitoreo continuo de los parámetros fisiológicos correspondientes, 
proporcionando una frecuencia de recolección de datos significativamente superior a los intervalos tradicionales establecidos en entornos convencional. \newline

Sin embargo, la implementación de estos sistemas de monitorización remota presenta diversos desafíos técnicos y operativos que requieren consideración. \newline

Entre las principales se encuentran:

\begin{itemize}
    \item Movilidad del usuario que puede afectar la calidad de la señal
    \item Uso correcto del dispositivo
    \item Variabilidad en las condiciones ambientales
    \item Fallos intermitentes de los sensores
    \item Agotamiento de la batería de los dispositivos
    \item Pérdida de conectividad en la transmisión de datos
    \item Falta de datos por dificultad de medición como en el caso de la escala Glasgow o niveles de conciencia
\end{itemize}

Esto genera la necesidad de procesar los datos adecuadamente y desarrollar técnicas para la limpieza y validación de datos.
A su vez, se deben implementar estrategias para el manejo de datos faltantes e incompletos. \parencite{rpm_pm}
