\subsection{Conceptos clave en Streaming}

El streaming se refiere al análisis y manipulación de datos en tiempo real a medida que se generan o reciben. Según Carbone et al. \parencite{carbone2015apache}, los conceptos fundamentales incluyen:

\begin{itemize}
    \item \textbf{Flujo de datos}: Una secuencia potencialmente infinita de registros que llegan continuamente \parencite{akidau2015dataflow}.
    \item \textbf{Latencia}: El tiempo entre la llegada de un dato y su procesamiento, crucial para aplicaciones en tiempo real \parencite{akidau2015dataflow}.
    \item \textbf{Ventanas}: Mecanismos para agrupar datos en intervalos finitos para su procesamiento \parencite{akidau2015dataflow}.
    \item \textbf{Estado}: Información que se mantiene entre eventos para cálculos incrementales \parencite{carbone2015apache}.
    \item \textbf{Watermarks}: Indicadores de progreso del tiempo en el flujo de datos \parencite{akidau2015dataflow}.
\end{itemize}
\newpage
\subsection{Batch vs Streaming}

La elección entre batch y streaming depende de los requisitos específicos de la aplicación:

\begin{table}[h]
\centering
\begin{tabular}{|p{3cm}|p{5cm}|p{5cm}|}
\hline
\textbf{Característica} & \textbf{Batch} & \textbf{Streaming} \\
\hline
Latencia & Alta (horas a días) & Baja (milisegundos a minutos) \\
\hline
Procesamiento & Alto & Moderado a alto \\
\hline
Complejidad & Menor & Mayor \\
\hline
Consistencia & Fuerte & Eventual \\
\hline
Uso típico & Análisis histórico, reportes & Monitoreo, alertas, decisiones inmediatas \\
\hline
\end{tabular}
\caption{Comparación de batch y streaming}
\label{tab:batch_vs_streaming}
\end{table}

Se puede decir que el streaming es esencial para aplicaciones
que requieren decisiones inmediatas, mientras que el análisis batch es más 
adecuado para análisis profundos de grandes volúmenes de datos históricos.\parencite{stonebraker2005one}
\newpage
\subsection{Evolución de las arquitecturas de procesamiento de datos}

La evolución de las arquitecturas de procesamiento de datos ha sido impulsada por la necesidad de manejar volúmenes cada vez mayores de datos en tiempo real:

\begin{enumerate}
    \item \textbf{Arquitecturas batch}: Sistemas tradicionales como Hadoop MapReduce, diseñados para procesar grandes volúmenes de datos estáticos \parencite{dean2008mapreduce}.
    \item \textbf{Arquitecturas de streaming puro}: Como Apache Storm, enfocadas en el procesamiento en tiempo real pero con limitaciones en la consistencia y exactitud \parencite{toshniwal2014storm}.
    \item \textbf{Arquitectura Lambda}: Propuesta por Marz \parencite{marz2011cap}, combina procesamiento batch y en tiempo real para balancear latencia, throughput y tolerancia a fallos.
    \item \textbf{Arquitectura Kappa}: Introducida por Kreps \parencite{kreps2014questioning}, simplifica Lambda tratando todos los datos como streams.
    \item \textbf{Arquitectura Delta}: Desarrollada por Databricks, también como respuesta a Lambda optimiza el procesamiento de datos tanto batch como streaming \parencite{deltalake} \parencite{delta}.
\end{enumerate}
\newpage