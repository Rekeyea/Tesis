\subsection{Conceptos clave en el procesamiento de streaming}

El procesamiento de streaming se refiere al análisis y manipulación de datos en tiempo real a medida que se generan o reciben. Según Carbone et al. \parencite{carbone2015apache}, los conceptos fundamentales incluyen:

\begin{itemize}
    \item \textbf{Flujo de datos}: Una secuencia potencialmente infinita de registros que llegan continuamente \parencite{akidau2015dataflow}.
    \item \textbf{Latencia}: El tiempo entre la llegada de un dato y su procesamiento, crucial para aplicaciones en tiempo real \parencite{akidau2015dataflow}.
    \item \textbf{Ventanas}: Mecanismos para agrupar datos en intervalos finitos para su procesamiento \parencite{akidau2015dataflow}.
    \item \textbf{Estado}: Información que se mantiene entre eventos para cálculos incrementales \parencite{carbone2015apache}.
    \item \textbf{Watermarks}: Indicadores de progreso del tiempo en el flujo de datos \parencite{akidau2015dataflow}.
\end{itemize}
\newpage
\subsection{Comparación entre procesamiento por lotes y en tiempo real}

La elección entre procesamiento por lotes y en tiempo real depende de los requisitos específicos de la aplicación:

\begin{table}[h]
\centering
\begin{tabular}{|p{3cm}|p{5cm}|p{5cm}|}
\hline
\textbf{Característica} & \textbf{Procesamiento por lotes} & \textbf{Procesamiento en tiempo real} \\
\hline
Latencia & Alta (minutos a horas) & Baja (milisegundos a segundos) \\
\hline
Throughput & Alto & Moderado a alto \\
\hline
Complejidad & Menor & Mayor \\
\hline
Consistencia & Fuerte & Eventual \\
\hline
Uso típico & Análisis histórico, reportes & Monitoreo, alertas, decisiones inmediatas \\
\hline
\end{tabular}
\caption{Comparación de procesamiento por lotes y en tiempo real}
\label{tab:batch_vs_streaming}
\end{table}

Como sugiere Stonebraker et al. \parencite{stonebraker2005one}, el procesamiento en tiempo real es esencial para aplicaciones
que requieren decisiones inmediatas, mientras que el procesamiento por lotes es más 
adecuado para análisis profundos de grandes volúmenes de datos históricos.
\newpage
\subsection{Evolución de las arquitecturas de procesamiento de datos}

La evolución de las arquitecturas de procesamiento de datos ha sido impulsada por la necesidad de manejar volúmenes cada vez mayores de datos en tiempo real:

\begin{enumerate}
    \item \textbf{Arquitecturas por lotes}: Sistemas tradicionales como Hadoop MapReduce, diseñados para procesar grandes volúmenes de datos estáticos \parencite{dean2008mapreduce}.
    \item \textbf{Arquitecturas de streaming puro}: Como Apache Storm, enfocadas en el procesamiento en tiempo real pero con limitaciones en la consistencia y exactitud \parencite{toshniwal2014storm}.
    \item \textbf{Arquitectura Lambda}: Propuesta por Marz \parencite{marz2011cap}, combina procesamiento por lotes y en tiempo real para balancear latencia, throughput y tolerancia a fallos.
    \item \textbf{Arquitectura Kappa}: Introducida por Kreps \parencite{kreps2014questioning}, simplifica la Lambda tratando todos los datos como streams.
    \item \textbf{Arquitectura Delta}: Desarrollada por Databricks, combina las ventajas de las arquitecturas Lambda y Kappa, optimizando el procesamiento de datos tanto en batch como en streaming \parencite{deltalake} \parencite{delta}.
\end{enumerate}
\newpage