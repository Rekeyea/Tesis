\begin{center}
    \Large\bfseries Descripción de Capítulos
\end{center}
\vspace{1cm}

Se describirá brevemente el contenido de cada capítulo del trabajo para darle al lector una idea general de la estructura del mismo.

\begin{description}
    \item[Capítulo 1: Introducción]
    En este capítulo se describe el contexto del trabajo, los objetivos planteados y la importancia que tiene en el contexto de la salud digital. 
    \item[Capítulo 2: Fundamentos Teóricos]
    En este capítulo se exploran los conceptos necesarios para entender el trabajo.
    Se hace una investigación desde lo más general como conceptos de Big Data y Streamings, 
    hasta lo más específico como las tecnologías utilizadas en la implementación de las arquitecturas 
    y los desafíos en el monitoreo remoto de pacientes.  

    Es la intención de este capítulo dar todas las herramientas posibles para preparar al lector para comprender las decisiones tomadas en el desarrollo del trabajo.
    \item[Capítulo 3: Metodología]
    En este capítulo se describe la metodología utilizada para comparar las arquitecturas.
    Se presentan los criterios de evaluación, las métricas definidas y el caso de uso específico.
    También se discuten las tecnologías elegidas para la implementación y el análisis comparativo.
    También, se presenta el conjunto de datos que será utilizado.
    \item[Capítulo 4: Desarrollo]
    En este capítulo se detalla el proceso de implementación de las arquitecturas.
    Se describe el caso de uso que se implementará, 
    el modelo de procesamiento, el despliegue de componentes teórico, la visualización de los datos y las limitaciones del desarrollo.
    Por último, se presentan los principios de diseño y las decisiones tomadas en la implementación.
    \item[Capítulo 5: Resultados]
    En este capítulo se presentan los resultados obtenidos a partir del desarrollo.
    Se describen las expectativas iniciales y los resultados de las pruebas.
    \item[Capítulo 6: Conclusiones y Trabajo Futuro]
    En este capítulo se resumen las conclusiones del trabajo y se discuten las implicaciones de los resultados obtenidos. 
    Se proponen posibles líneas de investigación futura y se reflexiona sobre el uso de estas arquitecturas en el entorno de salud.
\end{description}