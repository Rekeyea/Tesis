% Abstract
\renewcommand{\abstractname}{Abstract}
\begin{abstract}
    En el contexto del monitoreo remoto de pacientes, 
    el procesamiento eficiente y confiable de datos de sensores en tiempo real se vuelve un requisito fundamental para la prevención y mejora de la atención médica. 
    
    Este trabajo analiza y compara dos arquitecturas modernas de procesamiento de datos en streaming: \textit{Kappa} y \textit{Delta}, 
    con el objetivo de determinar cuál resulta más adecuada para este tipo de sistemas.

    Se desarrolló un sistema de monitoreo basado en sensores que simula un entorno hospitalario y utiliza puntuaciones derivadas del sistema \textit{NEWS2} para evaluar el estado de salud de los pacientes. 
    Sobre este entorno se implementaron ambas arquitecturas utilizando un stack tecnológico compuesto por herramientas como Apache Kafka, Apache Flink y Apache Doris, entre otras.

    La comparación se realizó en función de métricas técnicas como latencia, throughput, uso de recursos y tolerancia a fallos, 
    así como también aspectos operativos y de costo. 
    
    Los resultados muestran que la arquitectura \textit{Kappa} ofrece menores tiempos de latencia, 
    mientras que la arquitectura \textit{Delta} destaca por su alto throughtput, escalabilidad, gestión operativa 
    y flexibilidad para reprocesamiento histórico y mejor integración con almacenamiento analítico.

    Finalmente, se discuten los desafíos asociados a la implementación de estas arquitecturas en entornos de salud, 
    y se presentan recomendaciones para su adopción según las necesidades específicas del sistema.
\end{abstract}