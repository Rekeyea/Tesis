% Abstract

\begin{center}
    \Large\bfseries Abstract
\end{center}
\vspace{1cm}


Esta investigación presenta un análisis comparativo entre las arquitecturas de streaming Kappa y Delta en el contexto del monitoreo remoto de pacientes. 
Frente al creciente volumen de datos generados por dispositivos IoT médicos, estas arquitecturas ofrecen enfoques distintos para el procesamiento en tiempo real e histórico de información crítica para el ámbito de salud.\newline

Para la evaluación, se implementó un sistema de monitoreo basado en el protocolo NEWS2 (National Early Warning Score 2), enriquecido con métricas de calidad y frescura de datos que permiten una degradación gradual de la fiabilidad de las mediciones en condiciones subóptimas. 
Se desplegaron ambas arquitecturas utilizando un stack tecnológico común, 
evaluando su rendimiento mediante métricas de latencia, throughput, uso de recursos y costos operativos.\newline

Los resultados revelan que la arquitectura Kappa proporciona latencias significativamente menores (80\% de las mediciones por debajo de 1 segundo) frente a Delta (promedio de 180 segundos), 
haciéndola superior para escenarios que requieren respuesta inmediata. 
Por otro lado, Delta demostró un throughput 64\% mayor, mejor eficiencia en almacenamiento y un costo operativo 36\% inferior, 
resultando más adecuada para análisis históricos y procesamiento contínuo que no requieran resultados inmediatos.\newline

El estudio concluye que, en sistemas de salud integrales, una estrategia híbrida que combine ambas arquitecturas según los requerimientos específicos de cada caso de uso representaría la solución óptima, 
aprovechando las fortalezas complementarias de cada enfoque.

\newpage