\section{Sistema de Monitoreo Remoto de Pacientes}

\subsection{Introducción}

\subsubsection{Antecedentes}
El Monitoreo Remoto de Pacientes ha emergido como una tecnología para mejorar la atención médica moderna, permitiendo la vigilancia continua del paciente fuera de los entornos clínicos tradicionales. 
Sin embargo, estos sistemas enfrentan desafíos en el mantenimiento de la calidad consistente de datos y la integración de mediciones de múltiples dispositivos con diferentes niveles de confiabilidad 
y tasas de muestreo.

\subsubsection{Planteamiento del Problema}
Los sistemas tradicionales de monitoreo de signos vitales frecuentemente enfrentan dificultades con:
\begin{itemize}
    \item Temporización inconsistente de mediciones entre diferentes signos vitales
    \item Variación en la confiabilidad y precisión de los dispositivos
    \item Integración de múltiples fuentes de datos para el mismo signo vital
    \item Mantenimiento de la validez clínica con datos incompletos
\end{itemize}

\subsubsection{Solución Propuesta}
Se propone abordar estos desafíos mediante:
\begin{itemize}
    \item Fusión de datos multi-dispositivo 
    \item Puntaje de calidad del dato
    \item Puntaje de frescura del dato
    \item Capacidad de degradación gradual
\end{itemize}

\subsection{Sistema de Puntuación NEWS2}

\subsubsection{Descripción General de NEWS2}
El National Early Warning Score 2 (NEWS2) es una herramienta de evaluación estandarizada utilizada para detectar el deterioro clínico. Evalúa seis parámetros fisiológicos:
\begin{itemize}
    \item Frecuencia respiratoria
    \item Saturación de oxígeno
    \item Presión arterial sistólica
    \item Frecuencia cardíaca
    \item Nivel de consciencia
    \item Temperatura
\end{itemize}

\subsubsection{Desafíos de Implementación Tradicional}
NEWS2 fue originalmente diseñado para mediciones manuales periódicas en entornos clínicos. Su adaptación para monitoreo remoto continuo presenta varios desafíos:
\begin{itemize}
    \item Diferentes frecuencias de medición para diferentes parámetros
    \item Calidad y confiabilidad variable de las mediciones
    \item Necesidad de actualizaciones de puntuación en tiempo real
    \item Manejo de datos faltantes o degradados
\end{itemize}

\subsection{gdNEWS2}

\subsubsection{Concepto y Fundamentos}
Se propone aumentar el sistema NEWS2 con el concepto de degradación gradual (graceful degradation) de modo que la puntuación aún sea útil incluso cuando la medición de alguno de los datos de signos vitales no 
se encuentre presente o no se confíe del todo en la calidad del mismo.
Cada parámetro de NEWS2 tiene un puntaje de calidad y de frescura asociado, que se utilizarán para definir el nivel de confianza que se le tiene a dicho parámetro.
De esta forma, se puede calcular una puntuación de alerta temprana incluso cuando no se tienen todos los datos disponibles 
y brindarle transparencia al profesional de la salud para que tome las decisiones correspondientes en cuanto al tratamiento del paciente.

\subsection{Formato de Datos de Mediciones Brutas}

Los datos serán recibidos en formato JSON, con un esquema común para todos los tipos de mediciones.
El esquema general es el siguiente:
\begin{lstlisting}[
    frame=single,
    numbers=left,
    numbersep=5pt,
    xleftmargin=20pt,
    language=JSON,
    basicstyle=\ttfamily,
    commentstyle=\color{gray},
    caption={JSON example},
    label={json-example}]
{
    "measurement_type": "RESPIRATORY_RATE" | "HEART_RATE" | "OXYGEN_SATURATION" | "BLOOD_PRESSURE_SYSTOLIC" | "TEMPERATURE" | "CONSCIOUSNESS",
    "measurement_timestamp": "datetime",
    "device_id": "string",
    "raw_value": "number",
    "battery": "number",
    "signal_strength": "number"
}
\end{lstlisting}
\newpage

\subsection{Algoritmo de Puntuación de Calidad}
Un nuevo algoritmo de puntuación de calidad debería basarse en la experiencia clínica y la evidencia científica. Para el caso de estudio presentado, 
se propone un algoritmo sencillo a fin de ilustrar su potencial y mantener limitado el enfoque de este trabajo.

Se tomarán los identificadores de los dispositivos, que luego serán clasificados en 3 grupos cada uno con un peso asociado: 

\begin{longtable}{|p{6cm}|p{3cm}|}
    \hline
    \textbf{Clasificación de Dispositivos} & \textbf{Peso Asociado} \\
    \hline
    \endhead
    Dispositivos de calidad médica & 1.0 \\
    \hline
    Dispositivos de calidad premium & 0.7 \\
    \hline
    Dispositivos de calidad de consumo & 0.4 \\
    \hline
    \caption{Clasificación de dispositivos y sus pesos correspondientes}
    \label{tab:dispositivos}
\end{longtable}
    

Además, se tomará en cuenta la señal de batería y la intensidad de la señal, que serán clasificados en 3 grupos cada uno con un peso asociado:

\begin{longtable}{|p{6cm}|p{3cm}|}
    \hline
    \textbf{Nivel de Batería} & \textbf{Peso Asociado} \\
    \hline
    \endhead
    Batería a más del 80\% & 1.0 \\
    \hline
    Batería entre 80\% y 50\% & 0.7 \\
    \hline
    Batería entre 50\% y 20\% & 0.6 \\
    \hline
    Batería a menos de 20\% & 0.4 \\
    \hline
    \caption{Clasificación de niveles de batería y sus pesos correspondientes}
    \label{tab:bateria}
\end{longtable}

Por último, se tomará en cuenta la intensidad de la señal:
\begin{longtable}{|p{6cm}|p{3cm}|}
    \hline
    \textbf{Intensidad de Señal} & \textbf{Peso Asociado} \\
    \hline
    \endhead
    Valor de señal de más de 0.8 & 1.0 \\
    \hline
    Valor de señal entre 0.8 y 0.6 & 0.8 \\
    \hline
    Valor de señal entre 0.5 y 0.6 & 0.6 \\
    \hline
    Valor de señal menor a 0.5 & 0.4 \\
    \hline
    \caption{Clasificación de intensidad de señal y sus pesos correspondientes}
    \label{tab:senal}
\end{longtable}

\newpage
De esta manera, el cálculo de la puntuación de calidad se realizará de la siguiente manera:


\begin{equation}
    \text{Quality Score} = 0.7 \times \text{Device Quality} + 0.2 \times \text{Battery Quality} + 0.1 \times \text{Signal Quality}
\end{equation}


En este momento, los valores tanto de los pesos como de los parametros son arbitrarios y se espera que en caso de encontrar útil este acercamiento, 
futuras iteraciones ajusten estos parámetros o utilicen un criterio diferente para su cálculo.

\subsection{Algoritmo de Puntuación de Frescura}

Para medir la frescura de los datos, se propone un enfoque simple que considera el tiempo transcurrido desde la última medición 
y el tiempo que le tomó a la medición actual llegar a ser procesada.

Tiempo desde medición hasta procesamiento:
\begin{itemize}
    \item Menos de una hora: 1.0
    \item Entre una y seis horas: 0.9
    \item Entre seis y doce horas: 0.7
    \item Entre doce y veinticuatro horas: 0.5
    \item Entre veinticuatro y cuarenta y ocho horas: 0.3
    \item En cualquier otro caso: 0.2
\end{itemize}

Tiempos entre mediciones:
\begin{itemize}
    \item Menos de cuatro horas: 1.0
    \item Entre cuatro y ocho horas: 0.8
    \item Entre ocho y doce horas: 0.6
    \item Entre doce y veinticuatro horas: 0.4
    \item Más de veinticuatro horas: 0.2
\end{itemize}

De esta manera, se propone el siguiente algoritmo de puntuación de frescura:

\begin{equation}
    \text{Freshness Score} = 0.5 \times \text{Time Since Last Measurement} + 0.5 \times \text{Time Since Measurement}
\end{equation}

\newpage

\subsection{Algoritmo de Puntuación de Degradación}

Este algoritmo se encargará de calcular la puntuación de degradación de la puntuación NEWS2 en caso de que no se tengan todos los datos disponibles.
Simboliza la confianza que se le tiene a la puntuación NEWS2 en base a la calidad y frescura de los datos.

Se propone utilizar la siguiente fórmula para calcular la puntuación de degradación:

\begin{equation}
    \text{Degradation Score} = 0.7 \times \text{Quality Score} + 0.3 \times \text{Freshness Score}
\end{equation}

Se calcula este puntaje para cada uno de los parámetros de NEWS2 y se promedia para obtener la puntuación de degradación final.

\subsection{Algoritmo de Puntuación de NEWS2}

El algoritmo de puntuación NEWS2 se basa en la suma de los puntajes de cada uno de los parámetros,
\begin{equation}
    \text{NEWS2 Score} = \sum_{i=1}^{n} \text{Parameter Score}_i
\end{equation}
donde $n$ es la cantidad de parámetros que se tienen disponibles.
En caso de que no se tenga un parámetro disponible, se asumirá una puntuación de cero para ese parámetro en su lugar.
De esta manera, se puede calcular la puntuación NEWS2 incluso cuando no se tienen todos los datos disponibles.

Por otro lado, se propone utilizar la puntuación de degradación para dar contexto sobre la puntuación NEWS2 final.
Así, se puede explicar que tan confiable es la puntuación NEWS2 calculada en base a los datos disponibles.