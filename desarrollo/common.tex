\section{Implementación}

Se implementará le especificación definida en el capítulo anterior de modo tal que para ambas arquitecturas los detalles de implementación sean lo más similares posible.
Para esto, se utilizará como lenguaje de procesamiento Flink SQL, que permite desarrollar los trabajos de procesamiento utilizando un lenguaje agnóstico a las plataformas subyascentes. 

\subsection{Pipeline de Procesamiento}
El pipeline de procesamiento se encargará de recibir los datos en formato JSON,
realizar el procesamiento de los mismos y devolver la puntuación NEWS2 calculada.

Esto se realizará de la siguiente manera:
\begin{itemize}
    \item Recepción de datos en formato JSON mediante un topico de Kafka
    \item Enriquecimiento de los datos con los puntajes de calidad y frescura
    \item Enrutamiento de los datos para su procesamiento particular según el signo vital
    \item Cálculo de la puntuación NEWS2 para cada una de las Componentes
    \item Unión y agrupación según una ventana de tiempo
    \item Calculo de valores de agregación de los puntajes de NEWS2 y de degradación
\end{itemize}

Debido a una limitante en el hardware de procesamiento, se simplificó el calculo de frescura de los datos para no tener en cuenta los anteriores. 
Esto provocaba que el sistema se quedara sin memoria y no pudiera procesar los datos para ambas arquitecturas. 
Por lo que se optó por no tener en cuenta los datos anteriores y solo calcular la frescura de los datos con las medidas de tiempo propias de cada registro.

A continuación se presenta un diagrama de flujo del pipeline de procesamiento:
\begin{figure}[h]
    \centering
    \includegraphics[width=1\textwidth]{desarrollo/pipeline.png}
    \caption{Diagrama de flujo del pipeline de procesamiento}
    \label{fig:flowchart}
\end{figure}

\clearpage

\subsection{Despliegue de Componentes}

El despliegue de los componentes se realizó mediante el uso de \textbf{Docker Compose}, y se midió según las métricas expuestas por esta herramienta.
Por otro lado, para el calculo de costos, se asume un despliegue de alta disponibilidad en la nube de AWS basado en el siguiente diagrama:

\begin{figure}[h]
    \centering
    \includegraphics[width=1\textwidth]{desarrollo/deployment.png}
    \caption{Diagrama de despliegue de componentes}
    \label{fig:infraestructura}
\end{figure}

\clearpage

La razón de este despliegue es la alta disponibilidad y la tolerancia a fallos, por lo que se despliega en una única región 
y se dividen los servicios en zonas de disponibilidad para asegurar que si una de ellas falla,
el sistema siga funcionando lo mejor posible.

Tres de las zonas de disponibilidad (a, b y c) son idempotentes en cuanto a su funcionamiento, 
cada una cuenta con un nodo de Kafka, un nodo Zookeeper, un nodo de procesamiento de Flink y un nodo de backend de Doris.
La cuarta zona de disponibilidad (d) tiene el nodo de frontend de Doris y el nodo de gestión de Flink; así como también un nodo extra de procesamiento de Flink.
No se incluye MinIO en este despliegue porque se utiliza S3 nativo, que se define en una región y esta igualmente comunicado con todas las zonas de disponibilidad.

A su vez, estarían idealmente desplegados mediante un orquestador de contenedores como Kubernetes, utilizando algún servicio como Elastic Kubernetes Serice (EKS).


\subsection{Repositorio de Código}

Se definireron tres repositorios de código para el desarrollo de la arquitectura Kappa y Delta.
El primero de ellos es el repositorio de la arquitectura Kappa, que contiene el código de procesamiento de datos y la configuración de los componentes.
El segundo es el repositorio de la arquitectura Delta, que contiene el código de procesamiento de datos y la configuración de los componentes.
El tercero es el repositorio del generador de datos sintéticos que se utilizó para realizar las pruebas de carga y estrés.

El código de cada uno de los repositorios se encuentra disponible en el siguiente enlace:

\begin{itemize}
    \item \url{https://github.com/Rekeyea/Tesis-Kappa}\\
    \item \url{https://github.com/Rekeyea/Tesis-Delta}\\
    \item \url{https://github.com/Rekeyea/Tesis-SynthDS}\\
\end{itemize}