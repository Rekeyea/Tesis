\documentclass[12pt,a4paper]{report}

\usepackage[utf8]{inputenc}
\usepackage[T1]{fontenc}
\usepackage[spanish,es-noshorthands]{babel}
\usepackage{csquotes} 
\usepackage{graphicx}

\usepackage{tabularx}
\usepackage{makecell}
\usepackage{multirow}
\usepackage{array}
\usepackage{cellspace}
\usepackage{listings}
\usepackage{xcolor}
\usepackage{amsmath}

\usepackage{tcolorbox}
\usepackage{listings}

\usepackage{longtable}
\usepackage{booktabs}
\usepackage{array}


% Define JSON language
\definecolor{delim}{RGB}{20,105,176}
\definecolor{numb}{RGB}{106, 109, 32}
\definecolor{string}{rgb}{0.64,0.08,0.08}

% JSON-specific configuration
\lstdefinelanguage{JSON}{
  string=[s]{"}{"},
  stringstyle=\color{red},
  comment=[l]{:},
  commentstyle=\color{blue},
  keywords={false,true,null},
  keywordstyle=\color{purple}\bfseries,
  morecomment=[l]{//},
  morecomment=[s]{/*}{*/},
  morestring=[b]',
  morestring=[b]",
  sensitive=false
}

\lstset{
  language=JSON,
  backgroundcolor=\color{gray!5},
  basicstyle=\ttfamily\footnotesize,
  breaklines=true,
  breakatwhitespace=true,
  columns=fullflexible,
  showstringspaces=false,
  frame=single,
  frameround=tttt,
  framesep=5pt,
  numbers=left,
  numberstyle=\tiny\color{gray},
  numbersep=10pt,
  tabsize=2,
  literate=
    {,}{{\color{blue},}}1
    {:}{{\color{blue}:}}1
    {\{}{{\color{darkgray}\{}}1
    {\}}{{\color{darkgray}\}}}1
    {[}{{\color{darkgray}[}}1
    {]}{{\color{darkgray}]}}1,
}

\lstdefinelanguage{CSV}{
  string=[s]{"}{"},
  stringstyle=\color{red},
  comment=[l]{\#},
  commentstyle=\color{gray},
  keywords={},
  morecomment=[l]{//},
  morestring=[b]',
  morestring=[b]",
  sensitive=false
}

\lstset{
  language=CSV,
  backgroundcolor=\color{gray!5},
  basicstyle=\ttfamily\footnotesize,
  breaklines=true,
  breakatwhitespace=true,
  columns=fullflexible,
  showstringspaces=false,
  frame=single,
  frameround=tttt,
  framesep=2pt,
  numbers=left,
  numberstyle=\tiny\color{gray},
  numbersep=5pt,
  tabsize=2,
  literate=
    {,}{{\color{blue},}}1
    {\{}{{\color{darkgray}\{}}1
    {\}}{{\color{darkgray}\}}}1
    {[}{{\color{darkgray}[}}1
    {]}{{\color{darkgray}]}}1,
}

\lstdefinelanguage{SQL}{
  keywords={CREATE, SELECT, INSERT, UPDATE, DELETE, FROM, WHERE, AND, OR, 
    NOT, ORDER, BY, GROUP, HAVING, JOIN, INNER, OUTER, LEFT, RIGHT, FULL, 
    ON, AS, UNION, ALL, IN, IS, NULL, LIKE, BETWEEN, EXISTS, CASE, WHEN, 
    THEN, ELSE, END, WITH, TABLE, VIEW, FUNCTION, PROCEDURE, TRIGGER, 
    DATABASE, SCHEMA, INDEX, PRIMARY, KEY, FOREIGN, REFERENCES, 
    CONSTRAINT, UNIQUE, CHECK, DEFAULT, ALTER, DROP, TRUNCATE, GRANT, 
    REVOKE, COMMIT, ROLLBACK, BEGIN, TRANSACTION, SET, VALUES, TOP, 
    LIMIT, OFFSET, ASC, DESC, DISTINCT, COUNT, SUM, AVG, MIN, MAX, INTO},
  sensitive=true,
  morecomment=[l]--,
  morecomment=[s]{/*}{*/},
  morestring=[b]',
  morestring=[b]",
}

\lstset{
 language=SQL,
 backgroundcolor=\color{gray!5},
 basicstyle=\fontfamily{lmtt}\selectfont\footnotesize,
 keywordstyle=\color{blue!80!black},
 identifierstyle=\color{black},
 commentstyle=\color{green!60!black},
 stringstyle=\color{red!70!black},
 numberstyle=\tiny\color{gray!70},
 numbers=left,
 stepnumber=1,
 breaklines=true,
 breakatwhitespace=false,
 showstringspaces=false,
 tabsize=2,
 captionpos=b,
 frame=single,
 frameround=tttt,
 rulecolor=\color{gray!50},
 framesep=3pt,
 xleftmargin=5pt,
}


\usepackage{tikz}
\usetikzlibrary{shapes,arrows,positioning}

% Load biblatex with Spanish options
\usepackage[spanish]{babel}
\usepackage{csquotes}
\usepackage[
  backend=biber,
  style=ieee,          % Estilo IEEE, muy usado en informática
  giveninits=true,     % Usa iniciales para nombres
  maxcitenames=3,      % Usa "et al." después de 3 autores en citas
  maxbibnames=3,       % Muestra hasta 6 autores en la bibliografía
  mincitenames=1,      % Mínimo 1 autor antes de "et al."
  uniquelist=false,    % No intenta hacer listas de autores únicas
  uniquename=false,    % No intenta hacer nombres únicos
  minbibnames=3,       % Mínimo 3 autores en bibliografía antes de "et al."
  sorting=nyt,         % Ordena por nombre, año, título
  url=false,           % No muestra URLs en entradas que tienen DOI
  doi=true,            % Muestra DOIs
  isbn=false,          % No muestra ISBN en la bibliografía
  dashed=false,        % No usa guiones para autores repetidos
  urldate=comp,        % Formato compacto para fechas de acceso
  useprefix=false      % Maneja correctamente prefijos en apellidos como "de
]{biblatex}

% Traducir "et al." a "y otros" en español
\DefineBibliographyStrings{spanish}{
  andothers = {y otros},
}
\renewcommand*{\bibinitperiod}{.}
\renewcommand*{\bibinitdelim}{\addnbthinspace}

\DeclareNameFormat{given-family}{%
  \ifgiveninits
    {\usebibmacro{name:given-family}
      {\namepartfamily}
      {\namepartgiveni}
      {\namepartprefix}
      {\namepartsuffix}}
    {\usebibmacro{name:given-family}
      {\namepartfamily}
      {\namepartgiven}
      {\namepartprefix}
      {\namepartsuffix}}%
  \usebibmacro{name:andothers}}

% Add your bibliography file
\addbibresource{references.bib}

\begin{document}

% Carátula
\begin{titlepage}
    \centering
    {\huge\bfseries Comparación de las arquitecturas Kappa y Delta para el monitoreo remoto de pacientes basado en datos de sensores \par}
    \vspace{2cm}
    {\Large Entregado como requisito para la obtencion del
titulo de Master en Big Data\par}
    \vspace{1cm}
    {\large Emiliano Conti - 289917\par}
    \vspace{1cm}
    {\large Tutor: Alejandro Bianchi \par}
    \vspace{1cm}
    {\small Universidad ORT\par}
    \vspace{1cm}
    {\large \today\par}
\end{titlepage}

\pagestyle{empty}

% Disclaimer
\begin{center}
    \Large\bfseries Disclaimer
\end{center}
\vspace{1cm}

\noindent Yo, Emiliano Conti declaro que el trabajo que se presenta en esta obra es de mi propia mano. Puedo asegurar que:

\begin{itemize}
    \item La obra fue producida en su totalidad mientras realizaba el Proyecto Final del Master en Big Data;
    \item Cuando he consultado el trabajo publicado por otros, lo he atribuido con claridad;
    \item Cuando he citado obras de otros, he indicado las fuentes. Con excepci\'on de estas citas, la obra es enteramente m\'ia;
    \item En la obra, he acusado recibo de las ayudas recibidas;
    \item Cuando la obra se basa en trabajo realizado conjuntamente con otros, he explicado claramente qu\'e fue contribuido por otros, y qu\'e fue contribuido por mi;
    \item Ninguna parte de este trabajo ha sido publicada previamente a su entrega, excepto donde se han realizado las aclaraciones correspondientes.
\end{itemize}

\vspace{2cm}

\noindent Firma: \rule{5cm}{0.1pt}

\vspace{1cm}

\noindent Fecha: 18/04/2025
\newpage

% Abstract
\renewcommand{\abstractname}{Abstract}
\begin{abstract}
    En el contexto del monitoreo remoto de pacientes, 
    el procesamiento eficiente y confiable de datos de sensores en tiempo real se vuelve un requisito fundamental para la prevención y mejora de la atención médica. 
    
    Este trabajo analiza y compara dos arquitecturas modernas de procesamiento de datos en streaming: \textit{Kappa} y \textit{Delta}, 
    con el objetivo de determinar cuál resulta más adecuada para este tipo de sistemas.

    Se desarrolló un sistema de monitoreo basado en sensores que simula un entorno hospitalario y utiliza puntuaciones derivadas del sistema \textit{NEWS2} para evaluar el estado de salud de los pacientes. 
    Sobre este entorno se implementaron ambas arquitecturas utilizando un stack tecnológico compuesto por herramientas como Apache Kafka, Apache Flink y Apache Doris, entre otras.

    La comparación se realizó en función de métricas técnicas como latencia, throughput, uso de recursos y tolerancia a fallos, 
    así como también aspectos operativos y de costo. 
    
    Los resultados muestran que la arquitectura \textit{Kappa} ofrece menores tiempos de latencia, 
    mientras que la arquitectura \textit{Delta} destaca por su alto throughtput, escalabilidad, gestión operativa 
    y flexibilidad para reprocesamiento histórico y mejor integración con almacenamiento analítico.

    Finalmente, se discuten los desafíos asociados a la implementación de estas arquitecturas en entornos de salud, 
    y se presentan recomendaciones para su adopción según las necesidades específicas del sistema.
\end{abstract}

\begin{center}
    \Large\bfseries Agradecimientos
\end{center}
\vspace{1cm}

Quiero agradecer a todos los que de una forma u otra estuvieron acompañandome en este camino.\newline

A mi familia, por su apoyo incondicional y por su esfuerzo que me permitió alcanzar mis metas.\newline

A mi pareja por su paciencia y por estar siempre a mi lado. \newline

A mis amigos por su compañía, escucharme y apoyarme en los momentos difíciles. \newline

También al Emiliano del pasado por haber tomado la decisión de 
realizar esta maestría y por haberme dado la oportunidad de aprender tanto en el proceso. \newline

Y por último un agradecimiento especial a Martin y Diego porque 
literalmente sin su confianza esto no hubiera sido posible.
\newpage

\begin{center}
    \Large\bfseries Descripción de Capítulos
\end{center}
\vspace{1cm}

Se describirá brevemente el contenido de cada capítulo del trabajo para darle al lector una idea general de la estructura del mismo.

\begin{description}
    \item[Capítulo 1: Introducción]
    En este capítulo se describe el contexto del trabajo, los objetivos planteados y la importancia que tiene en el contexto de la salud digital. 
    \item[Capítulo 2: Fundamentos Teóricos]
    En este capítulo se exploran los conceptos necesarios para entender el trabajo.
    Se hace una investigación desde lo más general como conceptos de Big Data y Streamings, 
    hasta lo más específico como las tecnologías utilizadas en la implementación de las arquitecturas 
    y los desafíos en el monitoreo remoto de pacientes.  

    Es la intención de este capítulo dar todas las herramientas posibles para preparar al lector para comprender las decisiones tomadas en el desarrollo del trabajo.
    \item[Capítulo 3: Metodología]
    En este capítulo se describe la metodología utilizada para comparar las arquitecturas.
    Se presentan los criterios de evaluación, las métricas definidas y el caso de uso específico.
    También se discuten las tecnologías elegidas para la implementación y el análisis comparativo.
    También, se presenta el conjunto de datos que será utilizado.
    \item[Capítulo 4: Desarrollo]
    En este capítulo se detalla el proceso de implementación de las arquitecturas.
    Se describe el caso de uso que se implementará, 
    el modelo de procesamiento, el despliegue de componentes teórico, la visualización de los datos y las limitaciones del desarrollo.
    Por último, se presentan los principios de diseño y las decisiones tomadas en la implementación.
    \item[Capítulo 5: Resultados]
    En este capítulo se presentan los resultados obtenidos a partir del desarrollo.
    Se describen las expectativas iniciales y los resultados de las pruebas.
    \item[Capítulo 6: Conclusiones y Trabajo Futuro]
    En este capítulo se resumen las conclusiones del trabajo y se discuten las implicaciones de los resultados obtenidos. 
    Se proponen posibles líneas de investigación futura y se reflexiona sobre el uso de estas arquitecturas en el entorno de salud.
\end{description}

\cleardoublepage
\begingroup
\makeatletter
\let\ps@plain\ps@empty
\makeatother
\tableofcontents
\endgroup
\cleardoublepage
\setcounter{page}{9}  % <-- Ajustá según lo que quieras
\pagestyle{plain}


% Capítulos
\chapter{Introducción}


\section{Descripción del Proyecto}

Este proyecto compara las arquitecturas Kappa y Delta en el contexto del monitoreo remoto de pacientes mediante sensores IoT. En la era de la salud digital, estos sistemas generan grandes volúmenes de datos en tiempo real que requieren procesamiento eficiente. Se analizarán ambas arquitecturas, se definirán métricas de comparación y se implementarán en un caso de uso de monitoreo de pacientes. El objetivo es determinar la arquitectura más adecuada, considerando factores como latencia, escalabilidad, complejidad de implementación y manejo de datos históricos y en tiempo real.

\newpage

\section{Objetivos}

\begin{enumerate}
    \item Realizar un análisis teórico exhaustivo de las arquitecturas Kappa y Delta, detallando sus componentes, flujos de datos y casos de uso típicos.
    
    \item Definir un conjunto de métricas y criterios para la comparación objetiva de ambas arquitecturas en el contexto del monitoreo remoto de pacientes.
    
    \item Implementar ambas arquitecturas utilizando un conjunto de datos simulado de sensores de monitoreo de pacientes.
    
    \item Ejecutar pruebas de rendimiento y funcionalidad en ambas implementaciones.
    
    \item Analizar los resultados obtenidos y determinar la arquitectura más adecuada para el caso de uso específico de monitoreo remoto de pacientes.
    
    \item Proporcionar recomendaciones para la selección e implementación de arquitecturas de procesamiento de Big Data en el ámbito de la salud digital.
\end{enumerate}

\newpage

\section{Caso de Estudio: Big Data en Sistema de Salud}

\subsection{Contexto del Sistema}
Sistema de salud integral que incluye perfiles de pacientes, telemedicina e integración con dispositivos IoT. El objetivo es mejorar la atención médica, con foco en prevencion, utilizando tecnologia.

\subsection{Descripción del Caso de Uso}
Monitoreo continuo y en tiempo real de la salud del paciente mediante el uso de Big Data. Se espera ademas, tener la capacidad de identificar patrones y tendencias en los datos medicos. Asi como tambien proporcionar recomendaciones personalizadas.

\subsection{Proceso}
\begin{enumerate}
    \item \textbf{Recopilación de Datos}:
    \begin{itemize}
        \item Dispositivos IoT (datos en tiempo real)
    \end{itemize}
    
    \item \textbf{Almacenamiento y Gestión}:
    \begin{itemize}
        \item Almacenamiento de datos centralizada, segura y escalable
    \end{itemize}
    
    \item \textbf{Análisis de Datos}:
    \begin{itemize}
        \item Procesamiento en tiempo real
        \item Análisis histórico
        \item Modelos predictivos (machine learning)
    \end{itemize}
    
    \item \textbf{Generación de Insights}:
    \begin{itemize}
        \item Tableros
        \item Alertas en tiempo real
    \end{itemize}
    
    \item \textbf{Intervención y Seguimiento}:
    \begin{itemize}
        \item Monitoreo continuo
        \item Feedback y mejora continua del sistema
    \end{itemize}
\end{enumerate}
\chapter{Marco Teórico}

\section{Introducción a Big Data y Streaming de Datos}

\subsection{Sistemas Distribuidos}
Un sistema distribuido es una colección de elementos computacionales autonomos que para su usuario 
parecen un sistema único y coherente. \parencite{tanenbaum}

Los sistemas distribuidos tienen dos caracteristicas que pueden regularse para escalar: Procesamiento y Almacenamiento.

En el último tiempo, ha habido una tendencia a preferir que la escala de ambas propiedades sea individual. 
Es decir, que se pueda escalar por un lado la potencia de procesamiento y por otro la capacidad de almacenamiento.

\subsubsection{Consistencia}
La Consistencia es la propiedad que tiene un sistema distribuido en la que todos los nodos ven los mismos datos al mismo tiempo.
Esto significa que cualquier lectura en cualquier momento deberá devolver el valor más reciente escrito para ese dato.
Si un sistema es consistente, una vez que se realiza una escritura, todas las lecturas subsiguientes deben reflejar esa escritura;
sin importar desde que nodo se hagan. 
Esta propiedad garantiza que los clientes de los sistemas nunca vean datos desactualizados o inconsistentes.
\newpage
\subsubsection{Disponibilidad}
La Disponibilidad es la propiedad que tiene un sistema distribuido para responder a todas las peticiones, ya sean de lectura o escritura, sin fallos.
Un sistema disponible garantiza que cada solicitud reciba una respuesta sin importar el estado individual de cada nodo que lo compone.
Esto significa que incluso si algunos nodos estan caidos, el sistema en su conjunto debe poder seguir dando servicio a las peticiones que recibe.
\subsubsection{Tolerancia a Particiones}
La Tolerancia a Particiones es la propiedad que tiene un sistema distribuido en la que continua funcionando a pesar de la perdida de 
conectividad entre nodos. Una partición ocurre cuando hay una ruptura en la comunicación dentro de la red, 
lo que resulta en que dos o más segmentos de la red no puedan comunicarse entre sí.
Un sistema tolerante a particiones puede seguir operando incluso cuando estas particiones ocurren, 
lo que significa que puede manejar retrasos o pérdidas de mensajes entre nodos sin fallar por completo. 

\newpage
\subsection{Definición y características del Big Data}

Big Data es un término paraguas que se usa en la industria de IT para denominar a un conjunto de tecnologías que manejan grandes volúmenes de datos.
La pregunta que se presenta entonces es: ¿qué tan grandes deberían ser estos volúmenes para ser considerados Big Data?
O incluso, ¿existen otras características que definan lo que es Big Data?
Una definición generalmente aceptada es la siguiente:

\begin{quote}
Las tecnologías de Big Data están orientadas a
procesar datos (conjuntos/activos) de alto volumen, alta velocidad y alta variedad
para extraer el valor de datos previsto y asegurar una alta
veracidad de los datos originales y la información obtenida, lo que demanda
formas de procesamiento de datos e información (análisis) rentables e innovadoras
para mejorar el conocimiento, la toma de decisiones y el control de procesos;
todo esto exige (debe ser apoyado por) nuevos modelos de datos
(que soporten todos los estados y etapas de los datos durante todo su ciclo de vida)
y nuevos servicios y herramientas de infraestructura que permitan obtener (y procesar)
datos de una variedad de fuentes (incluidas las redes de sensores) y
entregar datos en una variedad de formas a diferentes consumidores y dispositivos
de datos e información. \parencite{demchenko2014addressing}
\end{quote}

\newpage
Por lo que podríamos considerar que es Big Data todo aquello que esté orientado a datos
cuyo volumen, velocidad y variedad no puedan ser tratados por un modelo de procesamiento
de datos tradicional (como podrían ser las bases de datos relacionales). Con el objetivo
de generar valor, asegurando la veracidad de los datos originales y la información obtenida.

\subsection{Streaming de datos}

El streaming de datos, también conocido como procesamiento de flujo, es un paradigma 
de procesamiento de datos en el que los datos se tratan como un flujo continuo e 
ilimitado de eventos discretos. En el contexto de Big Data, el streaming permite procesar 
y analizar grandes volúmenes de datos en tiempo real o casi real, a medida que se generan 
o llegan al sistema. \parencite{flink}

\subsection{Teorema CAP}
El Teorema CAP es un concepto fundamental en el diseño de sistemas distribuidos. 
Este establece que es imposible garantizar al mismo tiempo, tanto la Consistencia (Consistency), 
Disponibilidad (Availability) y la Tolerancia a las Particiones (Partition Tolerance).

Según esto, un sistema distribuido sólo es capaz de garantizar dos de estas propiedades al mismo tiempo. 
En general, para los sistemas de Big Data de Streaming, la disponibilidad es una propiedad obligatoria, ya que cualquier inactividad puede 
resultar en la pérdida de datos valiosos o en la imposibilidad de realizar acciones.

Por otro lado, la Tolerancia a Particiones es también indispensable para estos sistemas, que por su naturaleza requieren que su capacidad de 
procesamiento este distribuida a través de múltiples nodos dispersos en una red no confiable; por lo que son suceptibles 
a que se genere una partición. Por lo tanto, si no tuviera esta propiedad el servicio podría dejar de ser disponible. 

Entonces, como corolario, un sistema de Big Data de Streaming debe tambien ser tolerante a las particiones para poder ser disponible.
Esto nos deja con una única opción: relajar el "grado de consistencia" hasta un punto razonable que permita que el sistema siga siendo eficáz.\parencite{capteo}
\newpage
\subsubsection{Consistencia Eventual}
La consistencia eventual es un modelo de consistencia en sistemas distribuidos que garantiza que, 
si no se realizan nuevas actualizaciones a un objeto, en algun momento (eventualmente) todos los accesos a ese objeto 
devolverán el último valor actualizado.  
La consistencia eventual se alinea con las compensaciones descritas por el teorema CAP, 
permitiendo que estos sistemas prioricen la disponibilidad y la tolerancia a particiones. 
Además, facilita la escalabilidad horizontal, crucial para manejar el crecimiento continuo de datos y clientes de los sistemas.
Por último, es importante diseñar cuidadosamente el sistema para manejar las posibles inconsistencias temporales 
y asegurar que la aplicación pueda tolerar y resolver estas situaciones de manera apropiada \parencite{capteo}

\subsection{Desafíos en el manejo de datos de streaming}

\begin{enumerate}
    \item \textbf{Procesamiento en tiempo real y baja latencia}
    
    El procesamiento de datos debe ocurrir con un retraso mínimo para proporcionar resultados en tiempo real.
    
    Un desafío clave en el procesamiento de streams es lograr equilibrar la latencia, el costo y la correctitud simultáneamente \parencite{akidau2015dataflow}.

    \item \textbf{Manejo de datos fuera de orden}
    
    Los datos pueden llegar en un orden diferente al que fueron generados, lo que complica el procesamiento.
    
    El procesamiento de eventos fuera de orden es un desafío fundamental en los sistemas de procesamiento de streams \parencite[p.~87]{flink}.

    \item \textbf{Escalabilidad}
    
    Los sistemas deben poder manejar volúmenes crecientes de datos sin degradación del rendimiento.
    
    La escalabilidad en sistemas de streaming implica la capacidad de aumentar el rendimiento añadiendo recursos computacionales \parencite{samurai}.

    \newpage
    \item \textbf{Tolerancia a fallos y consistencia}
    
    El sistema debe poder recuperarse de fallos sin pérdida de datos y mantener la consistencia eventual de los resultados.
    
    Garantizar la semántica de "exactamente una vez" en presencia de fallos es un desafío significativo en el procesamiento de streams \parencite{carbone2015apache}.

    \item \textbf{Procesamiento de ventanas temporales}
    
    Definir y procesar eficientemente ventanas de tiempo sobre streams de datos continuos.
    
    El procesamiento de ventanas temporales es fundamental en aplicaciones de streaming y requiere consideraciones cuidadosas en cuanto a la semántica del tiempo y la completitud de los datos \parencite{akidau2015dataflow}.

    \item \textbf{Integración con sistemas batch}
    
    Combinar eficazmente el procesamiento de streams con sistemas batch existentes.
    
    La integración de paradigmas batch y streaming, a menudo referida como 'procesamiento híbrido', presenta desafíos únicos en términos de consistencia de datos y modelos de programación \parencite{carbone2015apache}.
\end{enumerate}
\newpage

\subsection{Conceptos clave en el procesamiento de streaming}

El procesamiento de streaming se refiere al análisis y manipulación de datos en tiempo real a medida que se generan o reciben. Según Carbone et al. \parencite{carbone2015apache}, los conceptos fundamentales incluyen:

\begin{itemize}
    \item \textbf{Flujo de datos}: Una secuencia potencialmente infinita de registros que llegan continuamente \parencite{akidau2015dataflow}.
    \item \textbf{Latencia}: El tiempo entre la llegada de un dato y su procesamiento, crucial para aplicaciones en tiempo real \parencite{akidau2015dataflow}.
    \item \textbf{Ventanas}: Mecanismos para agrupar datos en intervalos finitos para su procesamiento \parencite{akidau2015dataflow}.
    \item \textbf{Estado}: Información que se mantiene entre eventos para cálculos incrementales \parencite{carbone2015apache}.
    \item \textbf{Watermarks}: Indicadores de progreso del tiempo en el flujo de datos \parencite{akidau2015dataflow}.
\end{itemize}
\newpage
\subsection{Comparación entre procesamiento por lotes y en tiempo real}

La elección entre procesamiento por lotes y en tiempo real depende de los requisitos específicos de la aplicación:

\begin{table}[h]
\centering
\begin{tabular}{|p{3cm}|p{5cm}|p{5cm}|}
\hline
\textbf{Característica} & \textbf{Procesamiento por lotes} & \textbf{Procesamiento en tiempo real} \\
\hline
Latencia & Alta (minutos a horas) & Baja (milisegundos a segundos) \\
\hline
Throughput & Alto & Moderado a alto \\
\hline
Complejidad & Menor & Mayor \\
\hline
Consistencia & Fuerte & Eventual \\
\hline
Uso típico & Análisis histórico, reportes & Monitoreo, alertas, decisiones inmediatas \\
\hline
\end{tabular}
\caption{Comparación de procesamiento por lotes y en tiempo real}
\label{tab:batch_vs_streaming}
\end{table}

Como sugiere Stonebraker et al. \parencite{stonebraker2005one}, el procesamiento en tiempo real es esencial para aplicaciones
que requieren decisiones inmediatas, mientras que el procesamiento por lotes es más 
adecuado para análisis profundos de grandes volúmenes de datos históricos.
\newpage
\subsection{Evolución de las arquitecturas de procesamiento de datos}

La evolución de las arquitecturas de procesamiento de datos ha sido impulsada por la necesidad de manejar volúmenes cada vez mayores de datos en tiempo real:

\begin{enumerate}
    \item \textbf{Arquitecturas por lotes}: Sistemas tradicionales como Hadoop MapReduce, diseñados para procesar grandes volúmenes de datos estáticos \parencite{dean2008mapreduce}.
    \item \textbf{Arquitecturas de streaming puro}: Como Apache Storm, enfocadas en el procesamiento en tiempo real pero con limitaciones en la consistencia y exactitud \parencite{toshniwal2014storm}.
    \item \textbf{Arquitectura Lambda}: Propuesta por Marz \parencite{marz2011cap}, combina procesamiento por lotes y en tiempo real para balancear latencia, throughput y tolerancia a fallos.
    \item \textbf{Arquitectura Kappa}: Introducida por Kreps \parencite{kreps2014questioning}, simplifica la Lambda tratando todos los datos como streams.
    \item \textbf{Arquitectura Delta}: Desarrollada por Databricks, combina las ventajas de las arquitecturas Lambda y Kappa, optimizando el procesamiento de datos tanto en batch como en streaming \parencite{deltalake} \parencite{delta}.
\end{enumerate}
\newpage
\section{Tecnologias para Streaming en Big Data}

\subsection{Mensajería Distribuida}

Las tecnologías de mensajería distribuidas en tiempo real cumplen el crucial rol de actuar como intermediarios entre las fuentes de datos 
y los sistemas que efectivamente procesan estos datos. \parencite{kleppmann} \newline

Deben funcionar como un conducto de alta capacidad de almacenamiento y baja latencia, capturando y canalizando los flujos de información
desde sus múltiples orígenes y hacia sus diversos destinos en tiempo real. \parencite{bigdata} \newline

Su papel es fundamentalmente el de un sistema nervioso central, coordinando y distribuyendo datos a través de complejas arquitecturas distribuidas. 
Actúan como amortiguadores, absorbiendo picos en el flujo de datos y garantizando un procesamiento constante y eficiente. 
Además, estas tecnologías sirven como una capa de abstracción, desacoplando los productores de datos de los consumidores, 
lo que permite una mayor flexibilidad y escalabilidad en el diseño del sistema.

\subsubsection{Apache Kafka}
Apache Kafka es el estandar de facto de este tipo de sistemas. Utiliza un modelo de publicación-subscripcion (pub/sub) basado en logs, 
donde los datos se envían a "topics" y se almacenan en particiones distribuidas. Las particiones tienen una garantía de orden de los mensajes
y permiten la retención de datos a largo plazo.\newline

Además, proporciona conectores para integración con diversos sistemas y un amplio ecosistema. A nivel de seguridad ofrece encriptación
en tránsito mediante TLS y permite ser configurado para soportar encriptación en reposo (aunque esto debe hacerse a nivel de sistema de archivos).
\newline

Por último, su arquitectura distribuida y replicada permite una alta disponibilidad y tolerancia a fallos.

\newpage
\subsubsection{Apache Pulsar}
Apache Pulsar es también una plataforma de mensajería y streaming distribuida, al igual que Apache Kafka, pero que se distingue por tener una 
arquitectura basada en capas, separando la capa de almacenamiento de la capa de procesamiento; lo que permite escalar cada uno independientemente.
\newline

Apache Pulsar soporta modelos de entrega como colas, publicación y suscripción y puede ofrecer garantías de entregar un mensaje 
exactamente una vez ("exactly-once"). Ofrece encriptación a nivel de mensaje, lo que permite una granularidad fina en cuanto a que encriptar. 
También soporta TLS para la encriptación en tránsito y permite la configuración de encriptación en reposo.
\newline

Por último, también soporta almacenamiento de mensajes a largo plazo y soporte nativo para esquemas:
Esto es, permite definir la estructura y el tipo de datos de los mensajes, lo que a su vez permite una validación automática de los datos 
y una serialización/deserialización más eficiente. Esto trae consigo además, la capacidad de evolucionar estos esquemas de mensajes,
de forma que productores y consumidores evolucionen independientemente.

\subsubsection{Amazon Kinesis}
Amazon Kinesis es un servicio de streaming de datos administrado en la nube de AWS. 
Está diseñado para recopilar, procesar y analizar datos de streaming en tiempo real a gran escala.
Kinesis, en realidad, se compone de varios servicios:
\begin{enumerate}
    \item Kinesis Data Streams para ingestión de datos en tiempo real
    \item Kinesis Data Firehose para cargar datos en los servicios de almacenamiento disponibles de AWS
    \item Kinesis Data Analytics para procesar datos con SQL o Java
    \item Kinesis Video Streams para streaming de video  
\end{enumerate}
Adicionalmente ofrece capacidades de auto-escalado, replicación entre zonas de disponibilidad para alta durabilidad, encriptación en reposo 
(y puede habilitarsele la encriptación en tránsito) y permite la retención de datos hasta 365 días.
\newline

Como se puede ver, al ser tan completo permitiría implementar, al menos en principio, una gran parte de un sistema de Big Data en tiempo real.

\subsubsection{Azure Event Hubs}
Azure Event Hubs es un servicio de ingestión de datos en tiempo real administrado en la plataforma Microsoft Azure.
Se supone que está diseñado para soportar millones de eventos por segundo con baja latencia. Al igual que Kinesis,
ofrece una muy buena con otros servicios de Microsoft Azure, lo que permite por ejemplo capturar directamente los eventos
en los servicios de almacenamiento disponibles en este proveedor de nube.\newline

Event Hubs es compatible con el protocolo Kafka, lo que permite a las aplicaciones existentes de Kafka conectarse sin cambios de código.
Ofrece una retención de mensajes por defecto de 1 dia que puede aumentado hasta 7. En caso de necesitar una retención más a largo plazo
se recomienda guardar los eventos en Azure Blob Storage o Azure Data Lake para su posterior procesamiento. 
Cuenta también con encriptación en transito con TLS y en reposo.\newline

Proporciona además, características como el procesamiento batch para optimizar el rendimiento, 
control de acceso basado en roles, y encriptación en reposo y en tránsito. 

\subsection{Comparación}

\paragraph{Escalabilidad}
\begin{itemize}
    \item \textbf{Apache Kafka:} Muy Alta escalabilidad horizontal, millones de mensajes/segundo
    \item \textbf{Apache Pulsar:} Muy alta escalabilidad horizontal, millones de mensajes/segundo con separación de almacenamiento y cómputo
    \item \textbf{Amazon Kinesis:} Buena escalabilidad, con 1.000 mensajes por segundo con la configuración por defecto aunque con configuración adicional puede llegar al millon por segundo hipotético
    \item \textbf{Azure Event Hubs:} Buena escalabilidad, con 1.000 mensajes por segundo. También puede escalar con configuración adicional a los 20.000 mensajes. Existe la posibilidad de tener una instancia dedicada que permite escalar a millones de eventos por segundo de forma hipotética
\end{itemize}

\newpage
\paragraph{Retención de datos}
\begin{itemize}
    \item \textbf{Apache Kafka:} Configurable, potencialmente ilimitada
    \item \textbf{Apache Pulsar:} Ilimitada por diseño
    \item \textbf{Amazon Kinesis:} Hasta 365 días, configurable
    \item \textbf{Azure Event Hubs:} Hasta 7 días, opción de Capture para largo plazo
\end{itemize}

\paragraph{Garantías de entrega}
\begin{itemize}
    \item \textbf{Apache Kafka:} At-least-once por defecto, exactly-once configurable
    \item \textbf{Apache Pulsar:} Exactly-once nativo
    \item \textbf{Amazon Kinesis:} At-least-once
    \item \textbf{Azure Event Hubs:} At-least-once
\end{itemize}

\paragraph{Encriptación}
\begin{itemize}
    \item \textbf{En reposo:}
    \begin{itemize}
        \item \textbf{Apache Kafka:} Configurable
        \item \textbf{Apache Pulsar:} Nativo
        \item \textbf{Amazon Kinesis:} Por defecto (AWS KMS)
        \item \textbf{Azure Event Hubs:} Por defecto
    \end{itemize}
    \item \textbf{En tránsito:} Todos soportan TLS/SSL
\end{itemize}

\paragraph{Observaciones clave}
\begin{itemize}
    \item Pulsar destaca en escalabilidad, retención, seguridad y consistencia nativos
    \item Kafka ofrece el ecosistema más maduro y desarrollado que le otorga características muy completas
    \item Servicios gestionados (Kinesis, Event Hubs) tienen encriptación en reposo por defecto, pero retención limitada
\end{itemize}
\clearpage

\begin{longtable}{|p{3cm}|p{2.5cm}|p{3cm}|p{3cm}|p{2.5cm}|}
    \hline
    \textbf{Característica} & \textbf{Kafka} & \textbf{Pulsar} & \textbf{Kinesis} & \textbf{Event Hubs} \\
    \hline
    Escalabilidad & Muy Alta & Muy alta & Alta & Alta \\
    \hline
    Retención & Configurable & Ilimitada & 365 días máx. & 7 días máx. \\
    \hline
    Garantías & Exactly-once & Exactly-once & At-least-once & At-least-once \\
    \hline
    Enc. en reposo & Configurable & Sí & Sí (AWS KMS) & Sí \\
    \hline
    Enc. en tránsito & Sí & Sí & Sí & Sí \\
    \hline
    Throughput & {Muy alto} & {Muy alto} & {Alto} & {Alto} \\
    \hline
    \caption{Comparación de Sistemas de Mensajería}
\end{longtable}
\clearpage
\newpage
\subsection{Motores de Procesamiento}

Si las tecnologías de mensajería distribuida son el sistema nervioso de un sistema de Big Data de Streaming, 
los motores de procesamiento podrían considerarse su cerebro.

Actúan como la capa que transforma, enriquece y analiza los datos cumpliendo varios roles:
\begin{itemize}
    \item Aplicar la lógica de negocio sobre los datos mientras estos fluyen
    \item Detectar patrones y anomalías sobre el flujo de datos
    \item Mantener el contexto y estado necesario para las operaciones con históricos o agregaciones
    \item Garantizar la consistencia de las operaciones incluso ante fallos del sistema
    \item Distribuir la carga de trabajo entre diferentes nodos, paralelizando tareas
\end{itemize}

Estos componentes pueden enlazarse y programarse de diversas maneras. Permitiendo ensamblarlos de forma de cumplir 
con los requisitos de negocio.
\newpage
\subsubsection{Apache Spark}

Apache Spark se destaca como uno de los motores de procesamiento más populares y versátiles en el ecosistema de Big Data.
Ofrece dos APIs principales para el procesamiento en tiempo real: Spark Streaming y Structured Streaming,
siendo esta última la más moderna y recomendada. \newline

Implementa el procesamiento en tiempo real tratando los datos streaming
como micro-batches y su enfoque de procesamiento es en memoria, siendo capaz de mantener su estado distribuido a través
de checkpoints, que son archivos que se guardan en un sistema de almacenamiento al que todos los nodos pueden acceder.\newline

Su modelo de procesamiento le permite ofrecer un cierto equilibrio entre latencia y throughput. La abstracción fundamental
de Spark son los RDDs (Resilient Distributed Datasets), que son colecciones inmutables de datos distribuidos que pueden
ser procesadas en paralelo, y los DataFrames, que proporcionan una abstracción de más alto nivel similar a una tabla de
base de datos.\newline

Spark destaca por su amplio soporte de lenguajes de programación:

\begin{itemize}
    \item Scala
    \item Python
    \item Java 
    \item R
    \item SQL
\end{itemize}

Su API unificada y extenso catálogo de bibliotecas incluye MLlib para machine learning, GraphX para procesamiento de
grafos, y Spark SQL para procesamiento estructurado.

\newpage
\subsubsection{Apache Flink}

Apache Flink adopta un enfoque nativo de streaming, tratando al procesamiento batch como un caso especial de streaming
con límites finitos. Su arquitectura está diseñada para mantener estado distribuido con garantías de consistencia muy
fuertes y latencias extremadamente bajas.\newline

El motor gestiona automáticamente la distribución del estado y los checkpoints,
asegurando semánticas de exactly-once y permitiendo recuperación exacta ante fallos sin duplicados.
Su modelo de procesamiento se basa en dos conceptos fundamentales:

\begin{itemize}
    \item Marcas de agua (Watermarks): Son metadatos que fluyen en el stream de datos indicando el progreso del tiempo del evento,
    permitiendo manejar datos desordenados.
    \item Ventanas de tiempo (Windows): Permiten agrupar y procesar datos en intervalos temporales definidos, soportando diversos
    tipos como tumbling, sliding y session windows. 
\end{itemize}

Flink proporciona APIs de diferentes niveles:

\begin{itemize}
    \item ProcessFunction: API de bajo nivel que ofrece máximo control sobre tiempo, estado y ventanas
    \item DataStream API: API de alto nivel para operaciones de streaming comunes
    \item Table API y SQL: APIs declarativas para operaciones relacionales   
\end{itemize}

Los lenguajes soportados son:

\begin{itemize}
    \item Java
    \item Scala
    \item Python
    \item SQL
\end{itemize}

\subsubsection{Apache Beam}

Apache Beam, por su lado, se distingue por proporcionar un modelo de programación unificado que abstrae el motor de ejecución subyacente.
Su potencia radica en la capacidad de escribir la lógica de procesamiento una vez y ejecutarla en 
diferentes motores de procesamiento (runners) como Spark o Flink.\newline

Esta capacidad de abstracción es particularmente valiosa en escenarios donde la portabilidad y la flexibilidad de despliegue son requisitos clave, 
permitiendo cambiar de motor de procesamiento según evolucionen las necesidades y sin reescribir el código de procesamiento.

\subsubsection{Apache Samza}

Apache Samza se distingue por su estrecha integración con Apache Kafka y su arquitectura diseñada para mantener el estado 
de procesamiento de forma distribuida con un modelo de particionamiento que permite escalar horizontalmente.\newline

Samza proporciona un modelo de procesamiento simple pero potente, con fuerte énfasis en la gestión de estado local y la tolerancia a fallos.
Se utiliza en Linkedin y la arquitectura Kappa fue propuesta inicialmente pensando en la utilización de este motor de procesamiento.

\subsubsection{Apache NiFi}

Apache NiFi aborda el procesamiento de datos desde una perspectiva de orquestación y gobierno de datos; 
centrándose en la automatización del flujo de datos entre sistemas.\newline

Su arquitectura está orientada a la trazabilidad y auditabilidad de cada dato que fluye por el sistema, 
manteniendo un registro detallado de todas las transformaciones y movimientos. Los datos fluyen 
a través de un grafo de procesadores que pueden transformar, enrutar y mediar entre diferentes protocolos y formatos. 
NiFi se destaca por su capacidad para garantizar la entrega confiable de datos, proveer linaje de datos completo y 
permitir modificaciones de flujos en tiempo real sin necesidad de detener el sistema.\newline

Por último, NiFi proporciona una interfaz visual para diseñar, controlar y 
monitorizar flujos de datos.

\newpage
\subsubsection{Apache Kafka Streams}

Apache Kafka Streams es una biblioteca de procesamiento de streaming que forma parte del ecosistema de Apache Kafka, 
diseñada para construir aplicaciones y microservicios de procesamiento en tiempo real.\newline

Opera con un modelo de procesamiento que permite operaciones con manejo de estado, incluyendo agregaciones por ventanas, manejo de múltiples streams de datos 
y transformaciones complejas mediante APIs de alto y bajo nivel. Tiene un manejo de estado distribuido que se gestionan con los mismos logs de Kafka.\newline

En cuanto a rendimiento, Kafka Streams alcanza una latencia típica de decenas de milisegundos a segundos, dependiendo de la complejidad del procesamiento y la configuración, 
mientras que su throughput puede escalar linealmente añadiendo más instancias.\newline 

También ofrece grantías de procesamiento at-least-once con posibilidad de exactly-once mediante configuración.


\subsubsection{Apache Pulsar Functions}

Apache Pulsar Functions ofrece capacidad de cómputo integrandose directamente en la infraestructura de Apache Pulsar. \newline

Este framework permite implementar funciones livianas que procesan mensaje a mensaje. 
El manejo del estado es distribuido y se realiza mediante un almacenamiento basado en RocksDB, que permite mantener información 
entre invocaciones de funciones de manera consistente y tolerante a fallos. En términos de rendimiento, está optimizado para baja latencia, típicamente en el rango de milisegundos, 
gracias a su modelo de procesamiento directo. \newline

El throughput puede escalar horizontalmente añadiendo más instancias de funciones, y el sistema proporciona garantías de procesamiento exactly-once 
La arquitectura de este sistema está diseñada para ser simple y eficiente, permitiendo casos de uso como enriquecimiento de datos, filtrado, y transformaciones 
en tiempo real.


\newpage
\subsection{Comparación}

\paragraph{Modelo de Procesamiento}
\begin{itemize}
    \item \textbf{Apache Spark:} Micro-batches.
    \item \textbf{Apache Flink:} Streaming nativo con procesamiento registro a registro.
    \item \textbf{Apache Beam:} Mismo que el modelo de su motor asociado 
    \item \textbf{Apache Samza:} Streaming nativo con procesamiento basado en tiempo
    \item \textbf{Apache NiFi:} Procesamiento basado en flujos de datos dirigidos
    \item \textbf{Kafka Streams:} Procesamiento de baja complejidad mensaje a mensaje integrado con Kafka
    \item \textbf{Pulsar Functons:} Procesamiento de baja complejidad mensaje a mensaje integrado con Pulsar
\end{itemize}

\paragraph{Manejo de Estado}
\begin{itemize}
    \item \textbf{Apache Spark:} Estado en memoria distribuido entre los nodos procesadores.
    \item \textbf{Apache Flink:} Estado distribuido utilizando snapshots.
    \item \textbf{Apache Beam:} Mismo que el manejo de su motor asociado 
    \item \textbf{Apache Samza:} Estado local con respaldo en Kafka
    \item \textbf{Apache NiFi:} Se maneja localmente en memoria en el nodo que procesa el flujo
    \item \textbf{Kafka Streams:} Estado local con respaldo en Kafka
    \item \textbf{Pulsar Functons:} Almacenado y gestionado con la instancia de manejo de almacenamiento (BookKeeper) de Pulsar
\end{itemize}

\paragraph{Latencia}
\begin{itemize}
    \item \textbf{Apache Spark:} Dependiendo de la configuración del micro-batch puede ir desde milisegundos a segundos.
    \item \textbf{Apache Flink:} Microsegundos a milisegundos.
    \item \textbf{Apache Beam:} Mismo que el de su motor asociado 
    \item \textbf{Apache Samza:} Milisegundos a segundos
    \item \textbf{Apache NiFi:} Segundos a Minutos
    \item \textbf{Kafka Streams:} Milisegundos a segundos
    \item \textbf{Pulsar Functons:} Milisegundos a segundos
\end{itemize}

\paragraph{Capacidad de Procesamiento}
\begin{itemize}
    \item \textbf{Apache Spark:} Cientos de miles de eventos por segundo por nodo
    \item \textbf{Apache Flink:} Millones de eventos por segundo por nodo
    \item \textbf{Apache Beam:} Mismo que el de su motor asociado 
    \item \textbf{Apache Samza:} Cientos de miles de eventos por segundo por partición
    \item \textbf{Apache NiFi:} Miles de eventos por segundo por nodo
    \item \textbf{Kafka Streams:} Cientos de miles de eventos por segundo por partición
    \item \textbf{Pulsar Functons:} Decenas de miles de eventos por segundo por nodo
\end{itemize}
\newpage
\paragraph{Garantías de Entrega}
\begin{itemize}
    \item \textbf{Apache Spark:} at-least-once por defecto aunque puede ser configurado para permitir exactly-once
    \item \textbf{Apache Flink:} exactly-once nativo
    \item \textbf{Apache Beam:} exactly-once, at-least-once y at-most-once dependiendo del motor utilizado
    \item \textbf{Apache Samza:} at-least-once por defecto aunque puede ser configurado para permitir exactly-once
    \item \textbf{Apache NiFi:} at-least--once con garantía de entrega a pesar de fallas del sistema
    \item \textbf{Kafka Streams:} at-least-once y exactly-once
    \item \textbf{Pulsar Functons:} mejor rendimiento con at-least-once pero soporta exactly-once y at-most-once
\end{itemize}

\paragraph{Observaciones clave}
\begin{itemize}
    \item Spark es el que tiene un ecosistema más extendido y es más accesible
    \item Flink tiene las mejores prestaciones a nivel de latencia y rendimiento
    \item NiFi no es una buena opción para streaming por su latencia 
    \item Kafka Streams y Pulsar Functions pueden ser usados de forma ligera para ruteo de mensajes
\end{itemize}

\clearpage
\newpage

\begin{longtable}{|p{2cm}|p{2.5cm}|p{3cm}|p{2cm}|p{2cm}|p{2cm}|}
    \hline
    \textbf{ } & \textbf{Modelo} & \textbf{Estado} & \textbf{Latencia} & \textbf{Proc.} & \textbf{Garantías} \\
    \hline
    \textbf{Apache Spark} & Micro-batches & En memoria & Baja & Alto & Exactly-once \\
    \hline
    \textbf{Apache Flink} & Streaming & Distribuido & Muy baja & Muy Alto & Exactly-once \\
    \hline
    \textbf{Apache Beam} & Variable & Variable & Variable & Variable & Variable \\
    \hline
    \textbf{Apache Samza} & Streaming & Por partición & Baja & Alto & Exactly-once \\
    \hline
    \textbf{Apache NiFi} & Flujo dirigido & Procesador & Media-Alta & Medio & At-least-once \\
    \hline
    \textbf{Kafka Streams} & Streaming & Distribuido & Baja & Alto & Exactly-once \\
    \hline
    \textbf{Pulsar Functions} & Mensaje & Distribuido & Baja & Alto & Exactly-once \\
    \hline
    \caption{Comparativa de Motores de Procesamiento Big Data}
\end{longtable}

\clearpage
\newpage
\subsection{Almacenamiento de Datos}

\subsubsection {Formatos de Almacenamiento}

Los formatos de datos son un componente fundamental en cualquier arquitectura de sistemas de información moderna, 
ya que determinan no solo cómo se almacena la información, sino también cómo se procesa, transmite y analiza. 
La elección adecuada del formato de datos puede tener un impacto significativo en el rendimiento, 
la escalabilidad y la eficiencia del sistema en su conjunto. Para un sistema de Big Data, 
donde se manejan grandes volúmenes de información, la importancia de estos formatos se magnifica, 
ya que pueden significar la diferencia entre un sistema eficiente y uno que consume recursos excesivos. 
Además, los formatos de datos actúan como un lenguaje común entre diferentes componentes, 
facilitando la interoperabilidad y la integración de tecnologías.

\subsubsection{Formatos Orientados a Filas}
Los formatos orientados a filas representan la forma tradicional de almacenamiento de datos, 
donde cada registro se almacena de manera secuencial. Este enfoque ha sido la base de los sistemas 
de gestión de bases de datos durante décadas y sigue siendo crucial en muchos escenarios.

\begin{itemize}
    \item Los registros completos se almacenan de manera contigua en disco
    \item Cada fila contiene todos los campos de un registro
    \item Optimizado para acceder a registros completos
    \item Los nuevos registros se añaden secuencialmente de forma eficiente
    \item Óptimo cuando las consultan necesitan todos los campos
    \item Fácil modificación de registros individuales
    \item Debe leer datos innecesarios cuando solo se necesitan algunas columnas
    \item Los datos heterogéneos juntos reducen la efectividad de los métodos de compresión
    \item Menos eficiente para análisis de columnas específicas
\end{itemize}

\subsubsection{Formatos Orientados a Columnas}

Los formatos de almacenamiento columnar representan un paradigma fundamental en el manejo de datos masivos, 
especialmente en entornos analíticos. A diferencia del almacenamiento tradicional orientado a filas, 
donde los registros se almacenan secuencialmente, el almacenamiento columnar organiza los datos por columnas, 
lo que ofrece ventajas significativas en ciertos escenarios.

\begin{itemize}
    \item En lugar de almacenar registros completos de manera contigua, los datos se organizan por columnas
    \item Cada columna se almacena en bloques separados de memoria o disco
    \item Los valores similares se almacenan juntos, mejorando la compresión
    \item Los datos similares almacenados juntos permiten mayores tasas de compresión
    \item Solo se leen las columnas necesarias para una consulta
    \item Facilita operaciones como SUM, AVG, COUNT sobre columnas específicas
    \item Permite procesamiento eficiente de datos en hardware
\end{itemize}

\subsubsection{Comparativa con Almacenamiento por Filas:}

Consideremos una tabla simple de usuarios:

\textbf{Formato por Filas:}
\begin{verbatim}
[ID1, "Juan", 25] -> [ID2, "Ana", 30] -> [ID3, "Pedro", 28]
\end{verbatim}

\textbf{Formato Columnar:}
\begin{verbatim}
IDs:    [ID1 -> ID2 -> ID3]
Nombres: ["Juan" -> "Ana" -> "Pedro"]
Edades:  [25 -> 30 -> 28]
\end{verbatim}

\begin{tabular}{|l|l|l|}
\hline
\textbf{Aspecto} & \textbf{Formato Columnar} & \textbf{Formato por Filas} \\
\hline
Lectura parcial & Muy eficiente & Menos eficiente \\
Inserción de registros & Más lenta & Más rápida \\
Compresión & Alta & Moderada \\
Consultas analíticas & Excelente & Regular \\
Consultas transaccionales & Regular & Excelente \\
\hline
\end{tabular}

\subsubsection{Formatos Específicos}

\paragraph{JSON (JavaScript Object Notation)} 
se ha convertido en el estándar de facto para el intercambio 
de datos en aplicaciones modernas, especialmente en entornos Web y APIs. Su popularidad se debe a su 
simplicidad, legibilidad humana y amplia compatibilidad con prácticamente todos los lenguajes de 
programación. A pesar de no ser el más eficiente en términos de espacio y rendimiento (ya que es un formato basado en filas), 
su flexibilidad para representar datos estructurados y semiestructurados lo hace invaluable en sistemas donde la 
interoperabilidad y la facilidad de desarrollo son prioritarias. Es particularmente útil en aplicaciones
donde las transacciones individuales y la flexibilidad del esquema son más importantes que la 
eficiencia en el procesamiento de grandes volúmenes de datos.

\paragraph{Apache AVRO} 
destaca como un formato de serialización de datos binario que combina la 
eficiencia del almacenamiento binario con la flexibilidad de esquemas evolutivos. Su característica 
más distintiva es su capacidad para manejar cambios en el esquema de datos a lo largo del tiempo sin 
requerir cambios en el código o reescritura de datos existentes. Para esto, AVRO almacena el esquema junto con 
los datos, lo que permite una deserialización precisa y eficiente. Es especialmente valioso en 
sistemas de mensajería y streaming de datos, donde la evolución del esquema y la 
eficiencia en la transmisión son cruciales. Su formato binario compacto y su capacidad de compresión 
lo hacen ideal para sistemas distribuidos donde el ancho de banda y el almacenamiento son 
consideraciones importantes.

\paragraph{Apache Parquet} 
se ha establecido como el formato columnar dominante en el ecosistema de 
Big Data, especialmente para cargas de trabajo analíticas. Su diseño columnar permite una compresión 
altamente eficiente y un muy buen rendimiento en consultas que involucran solo un subconjunto de 
columnas. Parquet destaca particularmente en escenarios de análisis de datos, 
donde su capacidad para manejar esquemas complejos anidados y su integración con 
casi todas las herramientas lo hacen indispensable. La adopción generalizada de Parquet en la 
industria, lo ha convertido en el estándar de facto para almacenamiento de datos analíticos.

\newpage
\paragraph{Optimized Row Columnar (ORC)} 
inicialmente fué desarrollado para optimizar Hive, 
y aunque ofrece excelentes capacidades de compresión y rendimiento en consultas, su relevancia 
ha disminuido significativamente en los últimos años frente a Parquet. Aunque ORC sigue siendo 
relevante en sistemas legacy y específicos de Hive, la tendencia de la industria se ha movido 
claramente hacia Parquet como el formato columnar preferido para análisis de datos a gran escala.

\subsubsection{Almacenamiento de Objetos en la Nube}

El almacenamiento de objetos en la nube constituye un paradigma de almacenamiento donde los datos se organizan y gestionan como objetos independientes dentro de una estructura plana.
Cada objeto almacenado comprende tres elementos fundamentales: los datos en sí mismos, un conjunto extenso de metadatos que describen y categorizan la información, 
y un identificador único global que permite su localización y recuperación. 
Dicho paradigma, se caracteriza por su naturaleza distribuida y su capacidad para manejar tanto datos estructurados como no estructurados, 
permitiendo almacenar desde documentos, archivos multimedia y hasta flujos de datos en tiempo real.

Los sistemas de almacenamiento de objetos se destaca por su capacidad para satisfacer las demandas contemporáneas de procesamiento de datos a gran escala.
Como por ejemplo, ofrece una escalabilidad prácticamente ilimitada, pudiendo crecer según las necesidades sin preocuparse por restricciones de capacidad.
Por otro lado, la durabilidad y disponibilidad de los datos se garantiza a través de la replicación automática en múltiples ubicaciones de forma automática y transparente.
Adicionalmente, proveen APIs normalmente basadas en el protocolo HTTP que permite acceder de forma interoperable y estandarizada a los recursos almacenados

El uso de estos sistemas tambien conlleva sus propios desafíos. El más importante puede ser la latencia; pero también deben los grados de consistencia que ofrecen.

\clearpage

\subsubsection{Formato de Tabla Analítica}

Los formatos de tabla analítica son tecnologías diseñadas para resolver los desafíos del manejo de datos
a gran escala. Surgen como respuesta a las limitaciones de los formatos de archivo tradicionales como Parquet y ORC
cuando se trabaja con ellos en la nube. 

Las características más importantes que aporta un formato de tabla analítica son:

\begin{itemize}
    \item Proveen abstracciones sobre la metadata de archivos
    \item Permiten el uso de tablas con semántica SQL y evolución de esquema en las mismas
    \item Transacciones ACID
    \item Actualizaciones y Borrados
    \item Optimización de datos para mejoras de rendimiento
    \item Compatibilidad con múltiples motores de procesamiento
    \item Control de versiones en los datos
\end{itemize}

Los formatos de tablas analíticas dan a sus sistemas subyacentes estas características a través de diferentes mecanismos y estrategias. 
La principal es la gestión de metadatos, ya que implementan estructuras de datos altamente optimizadas que permiten rastrear 
eficientemente los archivos y sus cambios; mientras mantienen un historial detallado de transacciones que garantiza la consistencia de los datos, 
complementando con estrategias efectivas de particionamiento y organización. 

Para la optimización del rendimiento, estos formatos emplean diversas técnicas como la compactación automática de archivos, 
estrategias de caching de datos para acceso rápido, utilización de formatos de archivo columnares como Parquet u ORC, 
y la incorporación de capacidades avanzadas de indexación y filtrado optimizado. 
La consistencia de los datos se garantiza mediante la implementación de transacciones ACID completas, 
que proporcionan un sólido aislamiento entre operaciones de lectura y escritura, manejan conflictos de manera automática 
y aseguran la consistencia en escenarios de operaciones concurrentes. 
En cuanto a la evolución y mantenimiento, estos formatos facilitan cambios de esquema sin interrupciones en el servicio. 

Todas estas mecanismos trabajan en conjunto para proporcionar una solución completa para el manejo de datos a escala masiva,
que como subproducto permite tratar la escritura de los archivos como un canal de mensajes sobre el que se puede hacer streaming.

Los exponentes más importantes de estos formatos son: Delta Table, Apache Iceberg y Apache Hudi

\paragraph{Delta Lake}es un sistema de almacenamiento de datos diseñado por Databricks que utiliza archivos Parquet como base, 
organizándolos en una estructura de directorios con dos componentes principales: 
los archivos de datos en formato Parquet y el directorio {\_delta\_log} para metadatos y registro de transacciones. 
Esta arquitectura mantiene las ventajas de Parquet mientras añade capacidades transaccionales y de control de versiones, 
implementando un sistema de checkpoints para optimizar el rendimiento y un manejo de concurrencia que combina control optimista 
con serialización de escrituras.

Las características fundamentales de Delta Lake incluyen transacciones ACID completas, 
capacidad de acceso a versiones anteriores, evolución de esquema controlada, 
operaciones de Merge sofisticadas y optimización automática de datos mediante compactación de archivos y mantenimiento de estadísticas. 
El sistema también proporciona soporte para procesamiento de datos en tiempo real con semántica exactly-once 
y una integración robusta principalmente con Spark.

\paragraph{Apache Hudi}desarrollado inicialmente por Uber, es una plataforma enfocada en crear una plataforma analítica transaccional,
disponibilizando dos tipos principales de formatos de tablas: 
Copy On Write (optimizado para lecturas) y Merge On Read (optimizado para escritura). 

Utiliza una Timeline para gestionar metadatos y registrar cronológicamente todas las acciones, 
implementando un sistema de indexación que permite optimizar operaciones y facilitar búsquedas rápidas, 
además maneja control de concurrencia optimista que mantiene lecturas sin bloqueos mientras serializa escrituras.
Las características principales de Hudi incluyen procesamiento incremental para manejar streams de datos, 
gestión sofisticada de registros individuales con versionado a nivel de registro, borrado lógico y evolución de esquemas.
También proporciona garantías de consistencia con transacciones ACID y semántica exactly-once, 
ofreciendo buena integración con motores de procesamiento como Spark y Flink.

Hudi, al ser una plataforma, no ofrece solo su formato de tabla analítica, 
sino también "Table Services" que son componentes computacionales que permiten la optimización como compactación automática 
y limpieza de almacenamiento. 

\newpage
\paragraph{Apache Iceberg}diseñado inicialmente por Netflix, implementa una arquitectura de almacenamiento que se caracteriza 
por un modelo de metadatos enfocado en la evolución del esquema. 

Utiliza una estructura jerárquica que separa completamente los metadatos de los datos, 
implementando control de versiones basado en snapshots atómicos e inmutables, 
donde cada snapshot representa un punto en el tiempo de la tabla y contiene referencias a todos los archivos de datos válidos para esa versión, 
permitiendo operaciones concurrentes sin necesidad de bloqueos pesados.

Entre sus características principales destacan una gestión de esquemas flexible que permite evolucionar tanto el esquema 
como la estrategia de particionamiento sin reescribir datos, 
una optimización de consultas avanzada basada en estadísticas detalladas a nivel de columna y archivos, 
y un sistema de control de concurrencia optimista. 

Además, disponibiliza herramientas de mantenimiento como expiración de snapshots, compactación de archivos 
y reescritura de datos para optimización física, 
junto con una robusta integración con diferentes motores de procesamiento como Spark y Flink.

\newpage

\subsubsection{Catálogos de Metadatos}
Los catálogos de metadatos actúan como un registro centralizado y organizado de toda la información sobre las tablas y conjuntos de datos en un sistema. 
Son una capa de abstracción que mantiene información crítica sobre la estructura, ubicación, esquema, particiones, historial de versiones y estadísticas de los datos, 
permitiendo una gestión eficiente y un acceso optimizado a los mismos.

Son necesarios para que sistemas externos conozcan la ubicación y la estructura de los datos almacenados. 

\paragraph{Hive Metastore} es la implementación más adoptada y el estándar de estos catálogos. Opera como un servicio centralizado que almacena la información sobre las estructuras de datos,
típicamente usando una base de datos relacional como base (por ejemplo PostgreSQL o MySQL). Sin embargo, puede presentar problemas con operaciones concurrentes complejas. 
Sin embargo, se pueden implementar capas de abstracción que permitan manejar estos escenarios, y generalmente se prefiere utilizar este sistema ya que está ampliamente probado.

\paragraph{Project Nessie} es otra implementación de catálogo de datos, aunque con un enfoque moderno, aplicando conceptos de control de versiones a los datos; similar a como lo hace git. 
No intenta mantener compatibilidad con Hive Metastore, sino que ofrece un nuevo paradigma de gestión de datos, por lo que puede implementar características más complejas. 
Sin embargo, esto también significa que requiere más esfuerzo en integración con herramientas existentes ya que su ecosistema aún no está tan desarrollado.

\paragraph{Apache Polaris} es un proyecto relativamente nuevo que busca estandarizar y modernizar la gestión de metadatos. Es un esfuerzo para estandarizar y unificar los catálogos de metadatos.
De esta manera, Apache Polaris busca ser una solución neutral y agnóstica a los distintos proveedores de catálogos.  

\newpage

\subsubsection{Bases de Datos Analíticas}

Las bases de datos analíticas, también conocidas como bases de datos OLAP (Online Analytical Processing) orientadas a tiempo real, 
son sistemas especializados diseñados para procesar y analizar grandes volúmenes de datos con énfasis particular en consultas complejas y agregaciones. 
Estos sistemas están arquitecturalmente optimizados para procesar rápidamente consultas que involucran múltiples dimensiones y métricas sobre conjuntos masivos de datos, 
permitiendo análisis en tiempo real o casi real. 

Estas bases de datos se distinguen por su capacidad de manejar cargas de trabajo analíticas complejas mientras mantienen latencias bajas y consistentes; 
especializandose en resultados en segundos o milisegundos.Esta característica se debe a su arquitectura orientada específicamente al análisis, 
que contrasta con los sistemas diseñados primariamente para transacciones (OLTP) o almacenamiento general de datos.

Por diseño permiten un análisis multidimensional eficiente, soportan alta concurrencia de usuarios, y pueden integrarse efectivamente con fuentes de datos en streaming.
Además, su diseño orientado a columnas permite una compresión más eficiente y mejor rendimiento en consultas analíticas que típicamente involucran solo un subconjunto de ellas. 
Proporcionando capacidades avanzadas de agregación y pueden manejar eficientemente tanto datos históricos como en tiempo real.

Ninguno de estos beneficios vienen sin sus propios desafios, ya que se requiere una cuidadosa planificacion, operación y mantenimiento para ,amtemer su rendimiento.
Además, más que nunca, es necesario organizar correctamente los indices y los esquemas de particionamiento son claves.

Ejemplos de estos sistemas son:

\paragraph{Apache Druid}
es una base de datos analítica distribuida diseñada principalmente para análisis en tiempo real de grandes volúmenes de datos de series temporales. 
Su arquitectura se distingue por su capacidad de ingesta en tiempo real combinada con consultas de baja latencia, 
utilizando un modelo de almacenamiento columnar híbrido que combina datos en memoria con almacenamiento en disco. 

\paragraph{Apache Pinot} 
fué desarrollado incialmente por LinkedIn y se enfoca en proporcionar análisis en tiempo real con latencias extremadamente bajas, 
incluso en escenarios de alta concurrencia de usuarios. 
Su arquitectura está optimizada para consultas de lectura masivas y paralelas. 
Además, destaca por su modelo de consistencia eventual y su capacidad para manejar esquemas dinámicos.

\paragraph{Apache Doris} 
inicialmente fue desarrollado por Baidu, integra capacidades de almacenamiento columnar MPP (Procesamiento Paralelo Masivo) con funcionalidades OLAP, 
ofreciendo una solución más cercana a una base de datos tradicional pero con capacidades analíticas avanzadas. 
Su arquitectura es más simple en comparación con Druid y Pinot, lo que facilita la operación y mantenimiento, manteniendo un buen rendimiento para consultas analíticas.

\newpage

\subsection{Desafíos}
\subsubsection{Procesamiento fuera de Orden}
\subsubsection{Small File Issue}
\section{Arquitecturas de Referencia}

Una arquitectura de referencia representa una plantilla abstracta y probada que encapsula las decisiones arquitectónicas fundamentales de un sistema, 
mejores prácticas y experiencias acumuladas en un dominio específico. \newline

Esta proporciona un vocabulario común, además de componentes estandarizados y patrones de interacción que sirven como base para el desarrollo de los sistemas concretos. \newline

La arquitectura de referencia no solo define la estructura y comportamiento base del sistema, sino que también establece los principios de diseño, 
restricciones técnicas y mecanismos de extensibilidad que guiarán estas implementaciones.
\newpage

\subsection{Instancias de Arquitectura}

Una plataforma que soporte el Monitoreo Remoto de Pacientes, necesita soportar grandes volúmenes de datos analizados como streaming.
Además, para poder dar soporte al uso de la plataforma se requiere poder analizar el histórico de dichos datos. \newline

Por último, el tiempo en que el análisis de los datos de streaming es disponibilizado debe ser lo más cercano a tiempo real para poder tomar 
decisiones a tiempo para la salud del paciente. \newline

Existen tres grandes familias de arquitecturas de referencia que cubren estos tres casos: 
\begin{itemize}
    \item Lambda
    \item Kappa
    \item Delta
\end{itemize}

De entre ellas, se instanciarán y se compararán Kappa y Delta; ya que son evoluciones propuestas sobre Lambda que siguen distintos caminos para alcanzar el mismo objetivo. \newline
 
Realizar esta instanciación implica seleccionar las tecnologías que se usarán para cumplir las condiciones de la arquitectura de referencia; 
así como también componentes que si bien no son descritos por la arquitectura de referencia, son necesarios para cumplir con las características de calidad
necesarias para el caso de uso propuesto.\newline

Por último, las dos instancias de arquitectura implementadas utilizarán las mismas tecnologías, o tanto como sea posible, de modo que los resultados sean comparables.  
\newpage
\section{Arquitectura Lambda}

\subsection{Descripción General}
La Arquitectura Lambda es un paradigma de procesamiento de datos diseñado para manejar grandes cantidades de información en sistemas de Big Data. 
Propuesta por Nathan Marz en 2011, esta arquitectura busca abordar las limitaciones de los sistemas de procesamiento batch (batch) y en tiempo real, 
combinando ambos enfoques para proporcionar una vista completa y actualizada de los datos.

\subsection{Componentes Principales}
La Arquitectura Lambda se compone de tres capas fundamentales:

\subsubsection{Batch Layer}
\begin{itemize}
\item Almacena el conjunto completo de datos históricos.
\item Procesa periódicamente volúmenes arbitrarios de datos.
\item Genera vistas pre-computadas para consultas eficientes.
\end{itemize}

\subsubsection{Serving Layer}
\begin{itemize}
\item Almacena las vistas pre-computadas de la capa de lotes.
\item Proporciona acceso de baja latencia a los resultados.
\end{itemize}

\subsubsection{Speed Layer}
\begin{itemize}
\item Procesa datos en tiempo real.
\item Genera vistas de estos datos.
\item Mantiene los datos guardados unicamente hasta que la Batch Layer haya hecho el reprocesamiento de los datos historicos.
\end{itemize}

\newpage
\subsubsection{Vista Lógica}
\begin{figure}[h]
\centering
\includegraphics[width=0.8\textwidth]{teorico/arquitecturas/lambda.png}
\caption{Diagrama de la Arquitectura Lambda}
\label{fig:arquitectura_lambda}
\end{figure}
\clearpage
\newpage

\subsection{Capacidades}
\begin{itemize}
\item \textbf{Procesamiento de datos a gran escala}: Maneja eficientemente volúmenes masivos de datos.
\item \textbf{Baja latencia}: Proporciona resultados en tiempo real para consultas.
\item \textbf{Tolerancia a fallos}: Mantiene la integridad de los datos incluso en caso de fallos del sistema.
\item \textbf{Escalabilidad}: Se adapta fácilmente al crecimiento del volumen de datos.
\item \textbf{Flexibilidad}: Permite el procesamiento tanto batch como en tiempo real.
\item \textbf{Consistencia eventual}: Garantiza que los datos eventualmente reflejarán todos los cambios.
\item \textbf{Reprocesamiento}: En caso de necesitar reprocesar los datos, este proceso es trivial, pues se tiene almacenado el histórico completo.
\end{itemize}

\subsection{Desafíos}
\begin{itemize}
\item \textbf{Complejidad}: La implementación y mantenimiento pueden ser complejos debido a la duplicación de lógica en las capas de lotes y velocidad.
\item \textbf{Latencia}: El procesamiento batch genera latencia debido al tiempo de la actualización de vistas.
\item \textbf{Costo}: Al utilizar recursos computacionales diferentes entre el procesamiento batch y en stream, esto puede requerir varios nodos computacionales, lo que incrementa los costos.
\end{itemize}
\newpage
\section{Arquitectura Kappa}

\subsection{Principios de Diseño}

La principal característica de esta arquitectura es su fuerte uso de un registro de eventos inmutable 
y ordenado cronológicamente que actúa como única fuente de verdad sobre los datos ingresados al sistema.

De esta manera, se logra unificar el procesamiento de datos en batch y streaming tratándolos como un flujo continuo de eventos, 
eliminando la dualidad de código y reduciendo la complejidad operativa.

El procesamiento de estos datos se realiza mediante motores de procesamiento de eventos que leen este registro, 
aplican transformaciones determinísticas 
y generan resultados derivados que pueden recomputarse en cualquier momento desde el inicio del log.

Este principio de reproducibilidad permite regenerar el estado completo del sistema cuando cambian los requisitos 
o algoritmos de procesamiento, sin necesidad de mantener rutas de código separadas.

Las vistas materializadas son otro principio fundamental, 
donde los resultados procesados se almacenan en sistemas optimizados para consultas, 
proporcionando acceso eficiente al estado actual sin necesidad de reprocesar todo el historial de eventos.

\newpage
\subsection{Stack Tecnológico}

Para la capa de ingesta y transporte de datos, la Arquitectura Kappa implementa \textbf{Apache Kafka} como componente central, 
funcionando no solo como sistema de mensajería sino como la fuente única de verdad y almacén principal de eventos. 
En Kappa se configura Kafka con períodos de retención extendidos, 
aprovechando la capacidad de compactación de logs para mantener el historial completo de eventos mientras 
se optimiza el espacio de almacenamiento. 
Esto se logra agregando la capacidad de almacenamiento en capas, mediante la cual se pueden mantener los eventos
en Object Storage (utilizando \textbf{MinIO}), cuando pasa un tiempo definido de mantención en almacenamiento local.

El procesamiento de datos se realiza mediante \textbf{Apache Flink},
se despliega en un cluster con un nodo Job Manager y cuatro nodos Task Manager; de forma de distribuir la carga de trabajo lo mejor posible.
En este caso, se define como punto de entrada un tópico de Kafka, para procesar los datos en tiempo real y
enviarlos a un nuevo topico y continuar con el procesamiento más adelante en la arquitectura.

En el último paso, se guarda el resultado del procesamiento en \textbf{Apache Doris}, un motor de análisis de datos
distribuido que permite realizar consultas SQL en tiempo real sobre grandes volúmenes de datos con una interfaz basada en MySQL.
Este componente permite escalar de forma diferente el acceso a los datos del procesamiento, 
siendo desplegado como un nodo frontend y tres nodos backend. 
Estos comparten el trabajo de procesamiento de consultas y almacenamiento de datos, 
mientras que el frontend se encarga de la distribución de las mismas. 

\newpage
\subsection{Vista de Componentes}

\begin{figure}[h]
\centering
\includegraphics[width=1\textwidth]{desarrollo/Kappa.png}
\caption{Diagrama de la Arquitectura Kappa}
\label{fig:des_arquitectura_kappa}
\end{figure}

\newpage

\subsection{Flujo de Procesamiento}

El siguiente es un ejemplo de uno de los trabajos de procesamiento de datos desarrollados:

\begin{lstlisting}[language=sql]
    SET 'execution.runtime-mode' = 'streaming';
    SET 'execution.checkpointing.mode' = 'EXACTLY_ONCE';
    SET 'table.local-time-zone' = 'UTC';
    SET 'execution.checkpointing.interval' = '60000';
    SET 'execution.checkpointing.timeout' = '30000';
    SET 'state.backend' = 'hashmap';
    SET 'table.exec.state.ttl' = '300000';
    SET 'parallelism.default' = '4';
\end{lstlisting}

\begin{lstlisting}[language=sql]
    -- Raw measurements table with original timestamps and device metrics
    CREATE TABLE raw_measurements (
        measurement_timestamp TIMESTAMP(3),
        measurement_type STRING,
        raw_value STRING,
        device_id STRING,
        battery DOUBLE,
        signal_strength DOUBLE,
        ingestion_timestamp TIMESTAMP(3) METADATA FROM 'timestamp' VIRTUAL,
        WATERMARK FOR measurement_timestamp AS measurement_timestamp - INTERVAL '10' SECONDS
    ) WITH (
        'topic' = 'raw.measurements',
        'connector' = 'kafka',
        'properties.bootstrap.servers' = 'kafka-1:19091,kafka-2:19092,kafka-3:19093',
        'format' = 'json',
        'json.timestamp-format.standard' = 'ISO-8601',
        'scan.startup.mode' = 'latest-offset'
    );
\end{lstlisting}
\newpage
\begin{lstlisting}[language=sql]
    CREATE TABLE enriched_measurements (
        measurement_type STRING,
        `value` DOUBLE,
        device_id STRING,
        patient_id STRING,
        
        -- Weights
        quality_weight DOUBLE,
        freshness_weight DOUBLE,
        
        -- Timestamps
        measurement_timestamp TIMESTAMP(3),
        ingestion_timestamp TIMESTAMP(3),
        enrichment_timestamp TIMESTAMP(3) METADATA FROM 'timestamp' VIRTUAL,
        WATERMARK FOR measurement_timestamp AS measurement_timestamp - INTERVAL '10' SECONDS
    ) WITH (
        'topic' = 'enriched.measurements',
        'connector' = 'kafka',
        'properties.bootstrap.servers' = 'kafka-1:19091,kafka-2:19092,kafka-3:19093',
        'format' = 'json',
        'json.timestamp-format.standard' = 'ISO-8601',
        'scan.startup.mode' = 'latest-offset'
    );
\end{lstlisting}
\newpage
\begin{lstlisting}[language=sql]
    -- Insert with quality and freshness calculations
    INSERT INTO enriched_measurements
    SELECT
        measurement_type,
        CAST(raw_value AS DOUBLE) AS `value`,
        device_id,
        REGEXP_EXTRACT(device_id, '.*_(P\d+)$', 1) AS patient_id,

        -- Quality components
        CAST((
            CASE
                WHEN device_id LIKE 'MEDICAL%' THEN 1.0
                WHEN device_id LIKE 'PREMIUM%' THEN 0.7
                ELSE 0.4
            END * 0.7 +
            CASE
                WHEN battery >= 80 THEN 1.0
                WHEN battery >= 50 THEN 0.8
                WHEN battery >= 20 THEN 0.6
                ELSE 0.4
            END * 0.2 +
            CASE
                WHEN signal_strength >= 0.8 THEN 1.0
                WHEN signal_strength >= 0.6 THEN 0.8
                WHEN signal_strength >= 0.4 THEN 0.6
                ELSE 0.4
            END * 0.1
        ) AS DECIMAL(7,2)) AS quality_weight,

        -- Combined freshness calculation
        CASE
            WHEN TIMESTAMPDIFF(HOUR, measurement_timestamp, ingestion_timestamp) <= 1 THEN 1.0
            WHEN TIMESTAMPDIFF(HOUR, measurement_timestamp, ingestion_timestamp) <= 6 THEN 0.9
            WHEN TIMESTAMPDIFF(HOUR, measurement_timestamp, ingestion_timestamp) <= 12 THEN 0.7
            WHEN TIMESTAMPDIFF(HOUR, measurement_timestamp, ingestion_timestamp) <= 24 THEN 0.5
            WHEN TIMESTAMPDIFF(HOUR, measurement_timestamp, ingestion_timestamp) <= 48 THEN 0.3
            ELSE 0.2
        END AS freshness_weight,
        
        -- Timestamps
        measurement_timestamp,
        ingestion_timestamp
    FROM raw_measurements;
\end{lstlisting}

Como se puede ver, FLink SQL permite tratar a los tópicos de Kafka como tablas, pudiendose así leer y escribir sobre ellos. 
Esto permite realizar un procesamiento de datos en tiempo real,
enriquecerlos y enviarlos a otro tópico de Kafka para su posterior procesamiento.

Para esta arquitectura se utilizaron dos conectores diferentes de Kafka. El primero, visto en los ejemplos, permite leer y escribir pero no modificar. 
Por otro lado, para las agregaciones, se utilizó \textbf{upsert-kafka} que agrega la semántica de actualización y borrado de mensajes,
que es muy útil para cuando se necesita un procesamiento incremental de la información, como es el caso de las agregaciones. 
Aunque cabe destacar que la potencia de Flink permite que se pueda hacer esto incluso para otros destinos de datos como se verá más adelante para Paimon.
Todo esto sin cambiar el código del trabajo de procesamiento. 

\newpage

\begin{lstlisting}[language=sql]
    CREATE TABLE scores (
        patient_id STRING,
        window_start TIMESTAMP(3),
        window_end TIMESTAMP(3),

        respiratory_rate_value DOUBLE,
        oxygen_saturation_value DOUBLE,
        blood_pressure_value DOUBLE,
        heart_rate_value DOUBLE,
        temperature_value DOUBLE,
        consciousness_value DOUBLE,

        respiratory_rate_score DOUBLE,
        oxygen_saturation_score DOUBLE,
        blood_pressure_score DOUBLE,
        heart_rate_score DOUBLE,
        temperature_score DOUBLE,
        consciousness_score DOUBLE,

        respiratory_rate_trust_score DOUBLE,
        oxygen_saturation_trust_score DOUBLE,
        blood_pressure_trust_score DOUBLE,
        heart_rate_trust_score DOUBLE,
        temperature_trust_score DOUBLE,
        consciousness_trust_score DOUBLE,

        measurement_timestamp TIMESTAMP(3),
        ingestion_timestamp TIMESTAMP(3),
        enrichment_timestamp TIMESTAMP(3),
        routing_timestamp TIMESTAMP(3),
        scoring_timestamp TIMESTAMP(3),
        union_timestamp TIMESTAMP(3),
        WATERMARK FOR union_timestamp AS union_timestamp - INTERVAL '10' SECONDS,
        PRIMARY KEY (patient_id, window_start) NOT ENFORCED
    ) WITH (
        'connector' = 'upsert-kafka',
        'topic' = 'scores',
        'properties.bootstrap.servers' = 'kafka-1:19091,kafka-2:19092,kafka-3:19093',
        'key.format' = 'json',
        'value.format' = 'json'
    );
\end{lstlisting}

\newpage

\begin{lstlisting}[language=sql]
    INSERT INTO scores
    SELECT * FROM (
        WITH unions as (
            ...
        )
        SELECT 
            patient_id,
            window_start,
            MAX(window_end) as window_end,

            MAX(CASE WHEN measurement_type = 'RESPIRATORY_RATE' THEN `value` END) as respiratory_rate_value,
            MAX(CASE WHEN measurement_type = 'OXYGEN_SATURATION' THEN `value` END) as oxygen_saturation_value,
            MAX(CASE WHEN measurement_type = 'BLOOD_PRESSURE_SYSTOLIC' THEN `value` END) as blood_pressure_value,
            MAX(CASE WHEN measurement_type = 'HEART_RATE' THEN `value` END) as heart_rate_value,
            MAX(CASE WHEN measurement_type = 'TEMPERATURE' THEN `value` END) as temperature_value,
            MAX(CASE WHEN measurement_type = 'CONSCIOUSNESS' THEN `value` END) as consciousness_value,

            ...

            MIN(measurement_timestamp) AS measurement_timestamp,
            MIN(ingestion_timestamp) AS ingestion_timestamp,
            MIN(enrichment_timestamp) AS enrichment_timestamp,
            MIN(routing_timestamp) AS routing_timestamp,
            MIN(scoring_timestamp) AS scoring_timestamp,
            CURRENT_TIMESTAMP as union_timestamp
        FROM TABLE(
            TUMBLE(
                TABLE unions, 
                DESCRIPTOR(measurement_timestamp), 
                INTERVAL '1' MINUTES
            )
        ) AS unions 
        GROUP BY patient_id, window_start
    ) as t;
\end{lstlisting}

Por último, se guarda el resultado del procesamiento en \textbf{Apache Doris} directamente desde Flink.
Para esto, es necesario que la tabla en Doris haya sido creada previamente y además definir un nombre con el que llamarla en el trabajo de procesamiento.
Luego, se puede insertar los datos y Flink y Doris acordarán la forma de hacerlo. Según las pruebas realizadas, esto se hace en batches. 
El tiempo, entre que se terminó de procesar y fue insertado en Doris no fué posible de medir ya que no se encontraró una forma de definir la fecha de inserción real.

\newpage

\begin{lstlisting}[language=SQL]
    CREATE TABLE doris_gdnews2_scores (
        patient_id STRING,
        window_start TIMESTAMP(3),
        window_end TIMESTAMP(3),

        -- AVG Raw measurements
        ...

        -- Raw NEWS2 scores
        ...
        news2_score DOUBLE,

        -- Trust gdNEWS2 scores
        ...

        news2_trust_score DOUBLE,

        -- Timestamps
        measurement_timestamp TIMESTAMP(3),
        ingestion_timestamp TIMESTAMP(3),
        enrichment_timestamp TIMESTAMP(3),
        routing_timestamp TIMESTAMP(3),
        scoring_timestamp TIMESTAMP(3),

        flink_timestamp TIMESTAMP(3),
        aggregation_timestamp TIMESTAMP(3),
        PRIMARY KEY (patient_id, window_start) NOT ENFORCED
    ) WITH (
        'connector' = 'doris',
        'fenodes' = '172.20.4.2:8030',
        'table.identifier' = 'kappa.gdnews2_scores',
        'username' = 'kappa',
        'password' = 'kappa',
        'sink.label-prefix' = 'doris_sink_gdnews2',
        'sink.properties.format' = 'json',
        'sink.properties.timezone' = 'UTC'
    );
\end{lstlisting}

\begin{lstlisting}[language=SQL]
    INSERT INTO doris_gdnews2_scores
    SELECT *
    FROM gdnews2_scores;
\end{lstlisting}
\section{Arquitectura Delta}


La Arquitectura Delta surge desde Databricks, al igual que la Arquitectura Kappa, 
como una respuesta a los desafíos que presenta la arquitectura Lambda.\newline

A diferencia de la suposición que hace Kappa de que todo puede ser tratado como un Stream, 
Delta por su parte, utiliza el almacenamiento de objetos en la nube como base y 
agrega por encima tecnología de metadata que otorga la posibilidad de utilizar este sistema de almacenamiento
como un canal de mensajes de modo que se puede hacer Streaming sobre él; además de agregar otras capacidades.\newline

Esto permite utilizar un único flujo de datos tanto para análisis en tiempo real como análisis histórico. 


\subsection{Descripción General}

Delta surge de la necesidad de procesar datos masivos a bajo costo; intentando aprovechar la estructura existente
en las organizaciones que utilizan almacenamiento en la nube.\newline

Los formatos de tabla analítica permiten abstraer el almacenamiento y tratarlo como si fuera una tabla, 
por lo que se ingestan y luego, mediante el uso de motores de procesamiento se realiza el análisis de dichos datos,
que se vuelcan en otras tablas dentro de la misma infrastructura.\newline

Esto permite no tener que distinguir entre batch y streaming, ya que los formatos de tabla analítica proveen 
mecanismos para detectar los cambios en las tablas y transmitirlos, generando un stream interno que puede ser
aprovechado para realizar un análisis incremental. 

\newpage

Por lo general se utiliza un patrón de diseño de datos llamado Medallion, que define tres niveles de calidad de datos:
\begin{itemize}
    \item Bronze: Donde se almacenan los datos que llegan en crudo
    \item Silver: Donde se filtran, limpian y enriquecen los datos de Bronze
    \item Gold: Donde se analizan los datos para generar información valiosa para el negocio 
\end{itemize}


\newpage
\subsection{Componentes Principales}

\subsubsection{Ingestion Layer}
\begin{itemize}
    \item Recibe eventos y los envía al Data Lakehouse Layer
    \item Su función es más limitada que en la Arquitectura Kappa
\end{itemize}

\subsubsection{Data Lakehouse Layer}
\begin{itemize}
    \item Es una capa montada sobre almacenamiento barato como los servicios de almacenamiento de objetos en la nube
    \item Los formatos de tabla analítica se montan sobre este almacenamiento
    \item Recursos de cómputo llamados Table Services pertenecen a esta capa y dan mantenimiento al almacenamiento 
    \item Los motores de procesamiento analizan los datos en varias etapas y las vuelvan nuevamente sobre el almacenamiento
    \item Generalmente se opta por una sub-arquitectura en niveles, cuyo último nivel son los datos procesados disponibles para el negocio
\end{itemize}

\subsubsection{Catalog Layer}
\begin{itemize}
    \item Provee una capa de gobernanza permitiendo acceso granular a los datos, auditoría y políticas de retención
    \item Permite interoperar con terceros, permitiéndoles descubrir tablas en base a metadatos
    \item Ofrece estadísticas de las tablas y herramientas para optimizar las consultas sobre ellas
    \item Es necesario para acceder a los datos históricos
\end{itemize}

\subsubsection{Serving Layer}
\begin{itemize}
    \item Almacena los resultados procesados del stream por un periodo de tiempo.
    \item Proporciona acceso de baja latencia a los resultados del procesamiento y al histórico de datos disponibles.
\end{itemize}

\newpage
\subsection{Vista Lógica}

\begin{figure}[h]
\centering
\includegraphics[width=0.8\textwidth]{teorico/arquitecturas/delta.png}
\caption{Diagrama de la Arquitectura Delta}
\label{fig:arquitectura_delta}
\end{figure}

\subsection{Capacidades}
\begin{itemize}
    \item Garantiza transacciones para un sistema distribuido
    \item Reduce los costos de almacenamiento y procesamiento de la información 
    \item Define una fuente de verdad única que puede ser usada por todos los procesos de análisis
    \item Reduce la cantidad de código que se debe mantener
    \item Permite agregar nuevas fuentes de datos sin necesidad de cambios en los procesos de análisis
\end{itemize}

\subsection{Desafíos}
\begin{itemize}
    \item La latencia es un problema si se necesitan capacidades de análisis en tiempo real
    \item Se requieren compromisos de latencia y rendimiento por el problema de "Manejo de archivos pequeños"
    \item Si el Catalog Layer no se construye correctamente no es posible que la arquitectura escale
\end{itemize}
\newpage
\section{Monitoreo Remoto de Pacientes}

Los Sistemas de Monitorización Remota de Pacientes (RPM, por sus siglas en inglés) constituyen un paradigma tecnológico de salud en el área de la Telemedicina 
que permite la adquisición, transmisión y análisis, idealmente en tiempo real, de datos fisiológicos del paciente fuera de los entornos clínicos tradicionales, 
mediante una red de dispositivos médicos y sensores de dispositivos inteligentes. 

Este enfoque contribuye a mejores resultados para los pacientes, disminuye costos para las instituciones de salud y permite dar un acceso más 
generalizado a los servicios médicos. Es especialmente para el seguimiento de condiciones pre-existente,
población anciana y monitoreo luego de intervenciones quirúrgicas.\parencite{rpm_iot}


\subsection{Monitoreo de Signos Vitales}

\subsubsection{Frecuencia Respiratoria}
\begin{itemize}
    \item Número de ciclos respiratorios (inspiración/espiración) por minuto
    \item Valores normales adulto: 12-20 respiraciones/min
\end{itemize}

\subsubsection{Saturación de Oxígeno}
\begin{itemize}
    \item Porcentaje de hemoglobina unida a oxígeno en sangre arterial
    \item Valores normales: {95-100\%}
\end{itemize}

\subsubsection{Presión Sistólica}
\begin{itemize}
    \item Presión máxima ejercida por la sangre sobre las paredes arteriales durante la sístole
    \item Valores normales: 90-120 mmHg
    \item 
\end{itemize}

\subsubsection{Frecuencia Cardíaca}
\begin{itemize}
    \item Número de contracciones cardíacas por minuto
    \item Valores normales adulto: 60-100 latidos/min
\end{itemize}

\subsubsection{Temperatura}
\begin{itemize}
    \item Medida del calor corporal
    \item Valores normales: 36.5-37.5°C
\end{itemize}

\subsubsection{Escala Glasgow}
\begin{itemize}
    \item Escala neurológica que evalúa nivel de consciencia
    \item Evalúa la apertura ocular, la respuesta verbal y la respuesta motora en distintos rangos
    \item Valores normales: 15
    \item Es difícil de automatizar
\end{itemize}

\subsubsection{Nivel de Conciencia}
\begin{itemize}
    \item Sistema simplificado de evaluación del estado de consciencia utilizado en valoración inicial y monitoreo
    \item Utiliza el sistema APVU: Alerta, Respuesta a estímulos verbales, Respuesta a estímulos dolorosos, Sin respuesta
    \item Valores normales: 0
    \item Al ifual que la escala Glasgow es dificil de automatizar
\end{itemize}
\newpage
\subsection{Identificación de Riesgo en Pacientes}

En un entorno hospitalario, la monitorización de los signos vitales constituye un pilar fundamental en la evaluación del estado clínico de un paciente. 
Estos parámetros fisiológicos esenciales incluyen la presión arterial (PA), la saturación de oxígeno en sangre (SpO2), la temperatura corporal (T), 
la frecuencia cardíaca (FC) y la frecuencia respiratoria (FR), los cuales son registrados sistemáticamente en intervalos de 4 a 6 horas 
como parte del protocolo estándar de vigilancia para detectar posibles deterioros en la condición del paciente.

En diversos establecimientos sanitarios a nivel global, el personal médico y de enfermería implementa metodologías estandarizadas de evaluación, 
conocidas como sistemas de alerta temprana (SAT). 
Estos sistemas utilizan algoritmos validados que asignan puntuaciones específicas a las desviaciones de los rangos normales de los signos vitales, 
permitiendo la activación de alertas cuando se detectan patrones que indican un deterioro clínico. 

Esta práctica sistemática facilita la identificación a tiempo de pacientes en riesgo y permite la intervención terapéuticas a tiempo, 
contribuyendo significativamente a la reducción de eventos adversos y a la optimización de los resultados clínicos.

Existen diferentes estándares para la detección, muchos dependientes del contexto de la unidad donde se atienda al paciente. \parencite{rpm_pm}

\subsubsection{MEWS}
MEWS (Modified Early Warning Score) es un sistema de puntuación fisiológica validado para la detección temprana del deterioro clínico en pacientes hospitalizados, 
que evalúa cinco parámetros vitales fundamentales: frecuencia respiratoria, frecuencia cardíaca, presión arterial sistólica, temperatura 
y nivel de consciencia. 

Cada parámetro recibe una puntuación de 0 a 3 según la gravedad de su alteración, siendo 0 el valor normal y 3 el más patológico; 
La suma total de estos valores genera una puntuación que oscila entre 0 y 14, categorizando el riesgo del paciente en bajo (0--1), medio (2--3), alto (4--5) o crítico ($\geq$6), 
lo que determina la frecuencia de monitorización necesaria y las intervenciones requeridas, 
desde una vigilancia rutinaria cada 8-12 horas en puntuaciones bajas hasta la activación inmediata del equipo de respuesta rápida y posible traslado a UCI en puntuaciones críticas.

\subsubsection{NEWS2}
El NEWS2 (National Early Warning Score 2) es una versión mejorada y actualizada del sistema de alerta temprana, adoptado como estándar por el Servicio Nacional de Salud del Reino Unido, 
que evalúa siete parámetros fisiológicos: frecuencia respiratoria (3-0 puntos), saturación de oxígeno (con dos escalas distintas según el riesgo de insuficiencia respiratoria hipercápnica, 3-0 puntos), 
uso de oxígeno suplementario (2 puntos si requiere), temperatura (3-0 puntos), presión arterial sistólica (3-0 puntos), frecuencia cardíaca (3-0 puntos) 
y nivel de consciencia utilizando la escala ACVPU; 
la puntuación total varía de 0 a 20, estratificando el riesgo en bajo (0-4), medio (5-6), alto (7 o más, o cualquier parámetro individual con puntuación de 3) y determinando la respuesta clínica necesaria, 
desde monitorización estándar hasta evaluación urgente por equipo de cuidados críticos, representando una mejora significativa respecto al MEWS al incluir la saturación de oxígeno 
y la confusión como nuevo nivel de consciencia.

\subsubsection{SOFA}
El SOFA (Sequential Organ Failure Assessment Score) es un sistema de puntuación diseñado para evaluación diaria (cada 24 horas) de la disfunción/fallo multiorgánico en unidades de cuidados intensivos, 
evaluando seis sistemas orgánicos: respiratorio (mediante la relación PaO2/FiO2 evaluada con cada gasometría, 0-4 puntos), 
cardiovascular (mediante presión arterial media y requerimiento de vasopresores monitorizados continuamente, 0-4 puntos), hepático (mediante bilirrubina sérica medida diariamente, 0-4 puntos), 
coagulación (mediante recuento plaquetario diario, 0-4 puntos), renal (mediante creatinina sérica diaria o gasto urinario horario, 0-4 puntos) 
y neurológico (mediante la escala de Glasgow evaluada cada 4 horas o con cambios clínicos, 0-4 puntos); 
cada sistema recibe una puntuación de 0 (normal) a 4 (máxima disfunción), con una puntuación total que varía de 0 a 24 puntos, calculándose cada 24 horas o antes si hay deterioro clínico significativo, 
siendo especialmente relevante el cambio en la puntuación a lo largo del tiempo.

\newpage
\subsubsection{qSOFA}
El qSOFA (quick Sequential Organ Failure Assessment) es una versión simplificada del SOFA, diseñada para la identificación rápida de pacientes con sospecha de sepsis y alto riesgo de mortalidad fuera de la UCI, 
evaluando únicamente tres parámetros clínicos que se pueden medir de manera inmediata a pie de cama, sin necesidad de pruebas de laboratorio: 
alteración del estado mental (escala de Glasgow $\leq$13 puntos, 1 punto), frecuencia respiratoria elevada ($\geq$22 respiraciones/minuto, 1 punto) y presión arterial sistólica baja ($\leq$100 mmHg, 1 punto);
la puntuación total varía de 0 a 3 puntos, donde una puntuación $\geq$2 indica alto riesgo de mortalidad y la necesidad de evaluación más exhaustiva, monitorización estrecha
y consideración de traslado a un nivel superior de cuidados; el qSOFA debe reevaluarse con cada valoración del paciente o ante cualquier cambio en su estado clínico, 
típicamente cada 1-2 horas en pacientes inestables o con sospecha de sepsis, siendo una herramienta especialmente útil en servicios de urgencias, plantas de hospitalización y entornos extrahospitalarios.

\newpage

\subsection{Desafíos}

Los sensores empleados en la monitorización remota permiten un monitoreo continuo de los parámetros fisiológicos correspondientes, 
proporcionando una frecuencia de recolección de datos significativamente superior a los intervalos tradicionales establecidos en entornos convencional. 
Sin embargo, la implementación de estos sistemas de monitorización remota presenta diversos desafíos técnicos y operativos que requieren consideración. 

Entre las principales se encuentran:

\begin{itemize}
    \item Movilidad del usuario que puede afectar la calidad de la señal
    \item Uso correcto del dispositivo
    \item Variabilidad en las condiciones ambientales
    \item Fallos intermitentes de los sensores
    \item Agotamiento de la batería de los dispositivos
    \item Pérdida de conectividad en la transmisión de datos
    \item Falta de datos por dificultad de medición como en el caso de la escala Glasgow o niveles de conciencia
\end{itemize}

Esto genera la necesidad de procesar los datos adecuadamente y desarrollar técnicas para la limpieza y validación de datos.
A su vez, se deben implementar estrategias para el manejo de datos faltantes e incompletos. \parencite{rpm_pm}

\chapter{Metodología}


\section{Criterios de Evaluación para Arquitecturas de Streaming}

Para evaluar y comparar las arquitecturas Kappa y Delta en el contexto del monitoreo remoto de pacientes, se considerarán los siguientes criterios:

\subsection{Latencia de Procesamiento}
\begin{itemize}
    \item Tiempo de respuesta para el procesamiento de datos en tiempo real
    \item Capacidad para manejar picos de datos sin aumentar significativamente la latencia
\end{itemize}

\subsection{Escalabilidad}
\begin{itemize}
    \item Capacidad para manejar un aumento en el volumen de datos
    \item Facilidad de agregar recursos computacionales según sea necesario
    \item Rendimiento bajo diferentes cargas de trabajo
\end{itemize}

\subsection{Consistencia de Datos}
\begin{itemize}
    \item Garantía de consistencia entre datos en tiempo real y datos históricos
    \item Manejo de datos fuera de orden o retrasados
\end{itemize}

\subsection{Tolerancia a Fallos}
\begin{itemize}
    \item Capacidad de recuperación ante fallos del sistema
    \item Prevención de pérdida de datos en caso de interrupciones
\end{itemize}

\subsection{Manejo de Datos Históricos}
\begin{itemize}
    \item Eficiencia en el acceso y análisis de datos históricos
    \item Capacidad para reprocesar datos históricos cuando sea necesario
\end{itemize}

\subsection{Costo Operativo}
\begin{itemize}
    \item Requisitos de hardware y software
    \item Costos de mantenimiento y operación a largo plazo
\end{itemize}

\subsection{Seguridad y Cumplimiento Normativo}
\begin{itemize}
    \item Capacidad para cifrar datos en tránsito y en reposo
    \item Cumplimiento con regulaciones de protección de datos en salud (por ejemplo, HIPAA)
\end{itemize}

\subsection{Rendimiento en Análisis Complejos}
\begin{itemize}
    \item Capacidad para realizar análisis en tiempo real de múltiples fuentes de datos
    \item Eficiencia en la ejecución de modelos de machine learning
\end{itemize}

Estos criterios servirán como base para una evaluación exhaustiva y objetiva de las arquitecturas Kappa y Delta en el contexto del monitoreo remoto de pacientes, permitiendo una comparación detallada y fundamentada.
\chapter{Desarrollo}

\section{Aclaraciones}

El desarrollo fué realizado a lo largo de 9 meses de trabajo, y todos los avances fueron subidos a repositorios privados de GitHub.
Para mejorar la transparencia en el proceso de desarrollo y monitorear avances estos repositorios se hicieron públicos.
De esta forma, se pueden ver los cambios hechos en cada momento del desarrollo.
\newpage
\section{Sistema de Monitoreo Remoto de Pacientes}

\subsection{Introducción}

\subsubsection{Antecedentes}
El Monitoreo Remoto de Pacientes ha emergido como una tecnología para mejorar la atención médica moderna, permitiendo la vigilancia continua del paciente fuera de los entornos clínicos tradicionales. 
Sin embargo, estos sistemas enfrentan desafíos en el mantenimiento de la calidad consistente de datos y la integración de mediciones de múltiples dispositivos con diferentes niveles de confiabilidad 
y tasas de muestreo.

\subsubsection{Planteamiento del Problema}
Los sistemas tradicionales de monitoreo de signos vitales frecuentemente enfrentan dificultades con:
\begin{itemize}
    \item Temporización inconsistente de mediciones entre diferentes signos vitales
    \item Variación en la confiabilidad y precisión de los dispositivos
    \item Integración de múltiples fuentes de datos para el mismo signo vital
    \item Mantenimiento de la validez clínica con datos incompletos
\end{itemize}

\subsubsection{Solución Propuesta}
Se propone abordar estos desafíos mediante:
\begin{itemize}
    \item Fusión de datos multi-dispositivo 
    \item Puntaje de calidad del dato
    \item Puntaje de frescura del dato
    \item Capacidad de degradación gradual
\end{itemize}

\subsection{Fundamento Clínico: El Sistema de Puntuación NEWS2}

\subsubsection{Descripción General de NEWS2}
El National Early Warning Score 2 (NEWS2) es una herramienta de evaluación estandarizada utilizada para detectar el deterioro clínico. Evalúa seis parámetros fisiológicos:
\begin{itemize}
    \item Frecuencia respiratoria
    \item Saturación de oxígeno
    \item Presión arterial sistólica
    \item Frecuencia cardíaca
    \item Nivel de consciencia
    \item Temperatura
\end{itemize}

\subsubsection{Desafíos de Implementación Tradicional}
NEWS2 fue originalmente diseñado para mediciones manuales periódicas en entornos clínicos. Su adaptación para monitoreo remoto continuo presenta varios desafíos:
\begin{itemize}
    \item Diferentes frecuencias de medición para diferentes parámetros
    \item Calidad y confiabilidad variable de las mediciones
    \item Necesidad de actualizaciones de puntuación en tiempo real
    \item Manejo de datos faltantes o degradados
\end{itemize}

\subsection{gdNEWS2}

\subsubsection{Concepto y Fundamentos}
Se propone aumentar el sistema NEWS2 con el concepto de degradación gradual (graceful degradation) de modo que la puntuación aún sea útil incluso cuando la medición de alguno de los datos de signos vitales no 
se encuentre presente o no se confíe del todo en la calidad del mismo.
Cada parámetro de NEWS2 tiene un puntaje de calidad y de frescura asociado, que se utilizarán para definir el nivel de confianza que se le tiene a dicho parámetro.
De esta forma, se puede calcular una puntuación de alerta temprana incluso cuando no se tienen todos los datos disponibles 
y brindarle transparencia al profesional de la salud para que tome las decisiones correspondientes en cuanto al tratamiento del paciente.

\subsection{Formato de Datos de Mediciones Brutas}

Los datos serán recibidos en formato JSON, con un esquema común para todos los tipos de mediciones.
El esquema general es el siguiente:
\begin{lstlisting}[
    frame=single,
    numbers=left,
    numbersep=5pt,
    xleftmargin=20pt,
    language=JSON,
    basicstyle=\ttfamily,
    commentstyle=\color{gray},
    caption={JSON example},
    label={json-example}]
{
    "measurement_type": "RESPIRATORY_RATE" | "HEART_RATE" | "OXYGEN_SATURATION" | "BLOOD_PRESSURE_SYSTOLIC" | "TEMPERATURE" | "CONSCIOUSNESS",
    "measurement_timestamp": "datetime",
    "device_id": "string",
    "raw_value": "number",
    "battery": "number",
    "signal_strength": "number"
}
\end{lstlisting}
\newpage

\subsection{Algoritmo de Puntuación de Calidad}
Un nuevo algoritmo de puntuación de calidad debería basarse en la experiencia clínica y la evidencia científica. Para el caso de estudio presentado, 
se propone un algoritmo sencillo a fin de ilustrar su potencial y mantener limitado el enfoque de este trabajo.

Se tomarán los identificadores de los dispositivos, que luego serán clasificados en 3 grupos cada uno con un peso asociado: 
\begin{itemize}
    \item Dispositivos de calidad médica: 1.0
    \item Dispositivos de calidad premium: 0.7
    \item Dispositivos de calidad de consumo: 0.4
\end{itemize}

Además, se tomará en cuenta la señal de batería y la intensidad de la señal, que serán clasificados en 3 grupos cada uno con un peso asociado:
\begin{itemize}
    \item Batería a mas del 80\%: 1.0
    \item Batería entre 80\% y 50\%: 0.7
    \item Batería entre 50\% y 20\%: 0.6
    \item Batería a menos de 20\%: 0.4
\end{itemize}

Por último, se tomará en cuenta la intensidad de la señal:
\begin{itemize}
    \item Valor de señal de mas de 0.8: 1.0
    \item Valor de señal entre 0.8 y 0.6: 0.8
    \item Valor de señal entre 0.5 y 0.6: 0.6
    \item Valor de señal menor a 0.5: 0.4
\end{itemize}

De esta manera, el cálculo de la puntuación de calidad se realizará de la siguiente manera:


\begin{equation}
    \text{Quality Score} = 0.7 \times \text{Device Quality} + 0.2 \times \text{Battery Quality} + 0.1 \times \text{Signal Quality}
\end{equation}


En este momento, los valores tanto de los pesos como de los parametros son arbitrarios y se espera que en caso de encontrar útil este acercamiento, 
futuras iteraciones ajusten estos parámetros o utilicen un criterio diferente para su cálculo.

\subsection{Algoritmo de Puntuación de Frescura}

Para medir la frescura de los datos, se propone un enfoque simple que considera el tiempo transcurrido desde la última medición 
y el tiempo que le tomó a la medición actual llegar a ser procesada.

Tiempo desde medición hasta procesamiento:
\begin{itemize}
    \item Menos de una hora: 1.0
    \item Entre una y seis horas: 0.9
    \item Entre seis y doce horas: 0.7
    \item Entre doce y veinticuatro horas: 0.5
    \item Entre veinticuatro y cuarenta y ocho horas: 0.3
    \item En cualquier otro caso: 0.2
\end{itemize}

Tiempos entre mediciones:
\begin{itemize}
    \item Menos de cuatro horas: 1.0
    \item Entre cuatro y ocho horas: 0.8
    \item Entre ocho y doce horas: 0.6
    \item Entre doce y veinticuatro horas: 0.4
    \item Más de veinticuatro horas: 0.2
\end{itemize}

De esta manera, se propone el siguiente algoritmo de puntuación de frescura:

\begin{equation}
    \text{Freshness Score} = 0.5 \times \text{Time Since Last Measurement} + 0.5 \times \text{Time Since Measurement}
\end{equation}

\newpage

\subsection{Algoritmo de Puntuación de Degradación}

Este algoritmo se encargará de calcular la puntuación de degradación de la puntuación NEWS2 en caso de que no se tengan todos los datos disponibles.
Simboliza la confianza que se le tiene a la puntuación NEWS2 en base a la calidad y frescura de los datos.

Se propone utilizar la siguiente fórmula para calcular la puntuación de degradación:

\begin{equation}
    \text{Degradation Score} = 0.7 \times \text{Quality Score} + 0.3 \times \text{Freshness Score}
\end{equation}

Se calcula este puntaje para cada uno de los parámetros de NEWS2 y se promedia para obtener la puntuación de degradación final.

\subsection{Algoritmo de Puntuación de NEWS2}

El algoritmo de puntuación NEWS2 se basa en la suma de los puntajes de cada uno de los parámetros,
\begin{equation}
    \text{NEWS2 Score} = \sum_{i=1}^{n} \text{Parameter Score}_i
\end{equation}
donde $n$ es la cantidad de parámetros que se tienen disponibles.
En caso de que no se tenga un parámetro disponible, se asumirá una puntuación de cero para ese parámetro en su lugar.
De esta manera, se puede calcular la puntuación NEWS2 incluso cuando no se tienen todos los datos disponibles.

Por otro lado, se propone utilizar la puntuación de degradación para dar contexto sobre la puntuación NEWS2 final.
Así, se puede explicar que tan confiable es la puntuación NEWS2 calculada en base a los datos disponibles.
\newpage
\section{Implementación}

Se implementará le especificación definida en el capítulo anterior de modo tal que para ambas arquitecturas los detalles de implementación sean lo más similares posible.
Para esto, se utilizará como lenguaje de procesamiento Flink SQL, que permite desarrollar los trabajos de procesamiento utilizando un lenguaje agnóstico a las plataformas subyascentes. 

\subsection{Pipeline de Procesamiento}
El pipeline de procesamiento se encargará de recibir los datos en formato JSON,
realizar el procesamiento de los mismos y devolver la puntuación NEWS2 calculada.

Esto se realizará de la siguiente manera:
\begin{itemize}
    \item Recepción de datos en formato JSON mediante un topico de Kafka
    \item Enriquecimiento de los datos con los puntajes de calidad y frescura
    \item Enrutamiento de los datos para su procesamiento particular según el signo vital
    \item Cálculo de la puntuación NEWS2 para cada una de las Componentes
    \item Unión y agrupación según una ventana de tiempo
    \item Calculo de valores de agregación de los puntajes de NEWS2 y de degradación
\end{itemize}

Debido a una limitante en el hardware de procesamiento, se simplificó el calculo de frescura de los datos para no tener en cuenta los anteriores. 
Esto provocaba que el sistema se quedara sin memoria y no pudiera procesar los datos para ambas arquitecturas. 
Por lo que se optó por no tener en cuenta los datos anteriores y solo calcular la frescura de los datos con las medidas de tiempo propias de cada registro.

A continuación se presenta un diagrama de flujo del pipeline de procesamiento:
\begin{figure}[h]
    \centering
    \includegraphics[width=1\textwidth]{desarrollo/pipeline.png}
    \caption{Diagrama de flujo del pipeline de procesamiento}
    \label{fig:flowchart}
\end{figure}

\clearpage

\subsection{Despliegue de Componentes}

El despliegue de los componentes se realizó mediante el uso de \textbf{Docker Compose}, y se midió según las métricas expuestas por esta herramienta.
Por otro lado, para el calculo de costos, se asume un despliegue de alta disponibilidad en la nube de AWS basado en el siguiente diagrama:

\begin{figure}[h]
    \centering
    \includegraphics[width=1\textwidth]{desarrollo/deployment.png}
    \caption{Diagrama de despliegue de componentes}
    \label{fig:infraestructura}
\end{figure}

\clearpage

La razón de este despliegue es la alta disponibilidad y la tolerancia a fallos, por lo que se despliega en una única región 
y se dividen los servicios en zonas de disponibilidad para asegurar que si una de ellas falla,
el sistema siga funcionando lo mejor posible.

Tres de las zonas de disponibilidad (a, b y c) son idempotentes en cuanto a su funcionamiento, 
cada una cuenta con un nodo de Kafka, un nodo Zookeeper, un nodo de procesamiento de Flink y un nodo de backend de Doris.
La cuarta zona de disponibilidad (d) tiene el nodo de frontend de Doris y el nodo de gestión de Flink; así como también un nodo extra de procesamiento de Flink.
No se incluye MinIO en este despliegue porque se utiliza S3 nativo, que se define en una región y esta igualmente comunicado con todas las zonas de disponibilidad.

A su vez, estarían idealmente desplegados mediante un orquestador de contenedores como Kubernetes, utilizando algún servicio como Elastic Kubernetes Serice (EKS).


\subsection{Repositorio de Código}

Se definireron tres repositorios de código para el desarrollo de la arquitectura Kappa y Delta.
El primero de ellos es el repositorio de la arquitectura Kappa, que contiene el código de procesamiento de datos y la configuración de los componentes.
El segundo es el repositorio de la arquitectura Delta, que contiene el código de procesamiento de datos y la configuración de los componentes.
El tercero es el repositorio del generador de datos sintéticos que se utilizó para realizar las pruebas de carga y estrés.

El código de cada uno de los repositorios se encuentra disponible en el siguiente enlace:

\begin{itemize}
    \item \url{https://github.com/Rekeyea/Tesis-Kappa}\\
    \item \url{https://github.com/Rekeyea/Tesis-Delta}\\
    \item \url{https://github.com/Rekeyea/Tesis-SynthDS}\\
\end{itemize}
\newpage
\section{Arquitectura Kappa}

\subsection{Principios de Diseño}

La principal característica de esta arquitectura es su fuerte uso de un registro de eventos inmutable 
y ordenado cronológicamente que actúa como única fuente de verdad sobre los datos ingresados al sistema.

De esta manera, se logra unificar el procesamiento de datos en batch y streaming tratándolos como un flujo continuo de eventos, 
eliminando la dualidad de código y reduciendo la complejidad operativa.

El procesamiento de estos datos se realiza mediante motores de procesamiento de eventos que leen este registro, 
aplican transformaciones determinísticas 
y generan resultados derivados que pueden recomputarse en cualquier momento desde el inicio del log.

Este principio de reproducibilidad permite regenerar el estado completo del sistema cuando cambian los requisitos 
o algoritmos de procesamiento, sin necesidad de mantener rutas de código separadas.

Las vistas materializadas son otro principio fundamental, 
donde los resultados procesados se almacenan en sistemas optimizados para consultas, 
proporcionando acceso eficiente al estado actual sin necesidad de reprocesar todo el historial de eventos.

\newpage
\subsection{Stack Tecnológico}

Para la capa de ingesta y transporte de datos, la Arquitectura Kappa implementa \textbf{Apache Kafka} como componente central, 
funcionando no solo como sistema de mensajería sino como la fuente única de verdad y almacén principal de eventos. 
En Kappa se configura Kafka con períodos de retención extendidos, 
aprovechando la capacidad de compactación de logs para mantener el historial completo de eventos mientras 
se optimiza el espacio de almacenamiento. 
Esto se logra agregando la capacidad de almacenamiento en capas, mediante la cual se pueden mantener los eventos
en Object Storage (utilizando \textbf{MinIO}), cuando pasa un tiempo definido de mantención en almacenamiento local.

El procesamiento de datos se realiza mediante \textbf{Apache Flink},
se despliega en un cluster con un nodo Job Manager y cuatro nodos Task Manager; de forma de distribuir la carga de trabajo lo mejor posible.
En este caso, se define como punto de entrada un tópico de Kafka, para procesar los datos en tiempo real y
enviarlos a un nuevo topico y continuar con el procesamiento más adelante en la arquitectura.

En el último paso, se guarda el resultado del procesamiento en \textbf{Apache Doris}, un motor de análisis de datos
distribuido que permite realizar consultas SQL en tiempo real sobre grandes volúmenes de datos con una interfaz basada en MySQL.
Este componente permite escalar de forma diferente el acceso a los datos del procesamiento, 
siendo desplegado como un nodo frontend y tres nodos backend. 
Estos comparten el trabajo de procesamiento de consultas y almacenamiento de datos, 
mientras que el frontend se encarga de la distribución de las mismas. 

\newpage
\subsection{Vista de Componentes}

\begin{figure}[h]
\centering
\includegraphics[width=1\textwidth]{desarrollo/Kappa.png}
\caption{Diagrama de la Arquitectura Kappa}
\label{fig:des_arquitectura_kappa}
\end{figure}

\newpage

\subsection{Flujo de Procesamiento}

El siguiente es un ejemplo de uno de los trabajos de procesamiento de datos desarrollados:

\begin{lstlisting}[language=sql]
    SET 'execution.runtime-mode' = 'streaming';
    SET 'execution.checkpointing.mode' = 'EXACTLY_ONCE';
    SET 'table.local-time-zone' = 'UTC';
    SET 'execution.checkpointing.interval' = '60000';
    SET 'execution.checkpointing.timeout' = '30000';
    SET 'state.backend' = 'hashmap';
    SET 'table.exec.state.ttl' = '300000';
    SET 'parallelism.default' = '4';
\end{lstlisting}

\begin{lstlisting}[language=sql]
    -- Raw measurements table with original timestamps and device metrics
    CREATE TABLE raw_measurements (
        measurement_timestamp TIMESTAMP(3),
        measurement_type STRING,
        raw_value STRING,
        device_id STRING,
        battery DOUBLE,
        signal_strength DOUBLE,
        ingestion_timestamp TIMESTAMP(3) METADATA FROM 'timestamp' VIRTUAL,
        WATERMARK FOR measurement_timestamp AS measurement_timestamp - INTERVAL '10' SECONDS
    ) WITH (
        'topic' = 'raw.measurements',
        'connector' = 'kafka',
        'properties.bootstrap.servers' = 'kafka-1:19091,kafka-2:19092,kafka-3:19093',
        'format' = 'json',
        'json.timestamp-format.standard' = 'ISO-8601',
        'scan.startup.mode' = 'latest-offset'
    );
\end{lstlisting}
\newpage
\begin{lstlisting}[language=sql]
    CREATE TABLE enriched_measurements (
        measurement_type STRING,
        `value` DOUBLE,
        device_id STRING,
        patient_id STRING,
        
        -- Weights
        quality_weight DOUBLE,
        freshness_weight DOUBLE,
        
        -- Timestamps
        measurement_timestamp TIMESTAMP(3),
        ingestion_timestamp TIMESTAMP(3),
        enrichment_timestamp TIMESTAMP(3) METADATA FROM 'timestamp' VIRTUAL,
        WATERMARK FOR measurement_timestamp AS measurement_timestamp - INTERVAL '10' SECONDS
    ) WITH (
        'topic' = 'enriched.measurements',
        'connector' = 'kafka',
        'properties.bootstrap.servers' = 'kafka-1:19091,kafka-2:19092,kafka-3:19093',
        'format' = 'json',
        'json.timestamp-format.standard' = 'ISO-8601',
        'scan.startup.mode' = 'latest-offset'
    );
\end{lstlisting}
\newpage
\begin{lstlisting}[language=sql]
    -- Insert with quality and freshness calculations
    INSERT INTO enriched_measurements
    SELECT
        measurement_type,
        CAST(raw_value AS DOUBLE) AS `value`,
        device_id,
        REGEXP_EXTRACT(device_id, '.*_(P\d+)$', 1) AS patient_id,

        -- Quality components
        CAST((
            CASE
                WHEN device_id LIKE 'MEDICAL%' THEN 1.0
                WHEN device_id LIKE 'PREMIUM%' THEN 0.7
                ELSE 0.4
            END * 0.7 +
            CASE
                WHEN battery >= 80 THEN 1.0
                WHEN battery >= 50 THEN 0.8
                WHEN battery >= 20 THEN 0.6
                ELSE 0.4
            END * 0.2 +
            CASE
                WHEN signal_strength >= 0.8 THEN 1.0
                WHEN signal_strength >= 0.6 THEN 0.8
                WHEN signal_strength >= 0.4 THEN 0.6
                ELSE 0.4
            END * 0.1
        ) AS DECIMAL(7,2)) AS quality_weight,

        -- Combined freshness calculation
        CASE
            WHEN TIMESTAMPDIFF(HOUR, measurement_timestamp, ingestion_timestamp) <= 1 THEN 1.0
            WHEN TIMESTAMPDIFF(HOUR, measurement_timestamp, ingestion_timestamp) <= 6 THEN 0.9
            WHEN TIMESTAMPDIFF(HOUR, measurement_timestamp, ingestion_timestamp) <= 12 THEN 0.7
            WHEN TIMESTAMPDIFF(HOUR, measurement_timestamp, ingestion_timestamp) <= 24 THEN 0.5
            WHEN TIMESTAMPDIFF(HOUR, measurement_timestamp, ingestion_timestamp) <= 48 THEN 0.3
            ELSE 0.2
        END AS freshness_weight,
        
        -- Timestamps
        measurement_timestamp,
        ingestion_timestamp
    FROM raw_measurements;
\end{lstlisting}

Como se puede ver, FLink SQL permite tratar a los tópicos de Kafka como tablas, pudiendose así leer y escribir sobre ellos. 
Esto permite realizar un procesamiento de datos en tiempo real,
enriquecerlos y enviarlos a otro tópico de Kafka para su posterior procesamiento.

Para esta arquitectura se utilizaron dos conectores diferentes de Kafka. El primero, visto en los ejemplos, permite leer y escribir pero no modificar. 
Por otro lado, para las agregaciones, se utilizó \textbf{upsert-kafka} que agrega la semántica de actualización y borrado de mensajes,
que es muy útil para cuando se necesita un procesamiento incremental de la información, como es el caso de las agregaciones. 
Aunque cabe destacar que la potencia de Flink permite que se pueda hacer esto incluso para otros destinos de datos como se verá más adelante para Paimon.
Todo esto sin cambiar el código del trabajo de procesamiento. 

\newpage

\begin{lstlisting}[language=sql]
    CREATE TABLE scores (
        patient_id STRING,
        window_start TIMESTAMP(3),
        window_end TIMESTAMP(3),

        respiratory_rate_value DOUBLE,
        oxygen_saturation_value DOUBLE,
        blood_pressure_value DOUBLE,
        heart_rate_value DOUBLE,
        temperature_value DOUBLE,
        consciousness_value DOUBLE,

        respiratory_rate_score DOUBLE,
        oxygen_saturation_score DOUBLE,
        blood_pressure_score DOUBLE,
        heart_rate_score DOUBLE,
        temperature_score DOUBLE,
        consciousness_score DOUBLE,

        respiratory_rate_trust_score DOUBLE,
        oxygen_saturation_trust_score DOUBLE,
        blood_pressure_trust_score DOUBLE,
        heart_rate_trust_score DOUBLE,
        temperature_trust_score DOUBLE,
        consciousness_trust_score DOUBLE,

        measurement_timestamp TIMESTAMP(3),
        ingestion_timestamp TIMESTAMP(3),
        enrichment_timestamp TIMESTAMP(3),
        routing_timestamp TIMESTAMP(3),
        scoring_timestamp TIMESTAMP(3),
        union_timestamp TIMESTAMP(3),
        WATERMARK FOR union_timestamp AS union_timestamp - INTERVAL '10' SECONDS,
        PRIMARY KEY (patient_id, window_start) NOT ENFORCED
    ) WITH (
        'connector' = 'upsert-kafka',
        'topic' = 'scores',
        'properties.bootstrap.servers' = 'kafka-1:19091,kafka-2:19092,kafka-3:19093',
        'key.format' = 'json',
        'value.format' = 'json'
    );
\end{lstlisting}

\newpage

\begin{lstlisting}[language=sql]
    INSERT INTO scores
    SELECT * FROM (
        WITH unions as (
            ...
        )
        SELECT 
            patient_id,
            window_start,
            MAX(window_end) as window_end,

            MAX(CASE WHEN measurement_type = 'RESPIRATORY_RATE' THEN `value` END) as respiratory_rate_value,
            MAX(CASE WHEN measurement_type = 'OXYGEN_SATURATION' THEN `value` END) as oxygen_saturation_value,
            MAX(CASE WHEN measurement_type = 'BLOOD_PRESSURE_SYSTOLIC' THEN `value` END) as blood_pressure_value,
            MAX(CASE WHEN measurement_type = 'HEART_RATE' THEN `value` END) as heart_rate_value,
            MAX(CASE WHEN measurement_type = 'TEMPERATURE' THEN `value` END) as temperature_value,
            MAX(CASE WHEN measurement_type = 'CONSCIOUSNESS' THEN `value` END) as consciousness_value,

            ...

            MIN(measurement_timestamp) AS measurement_timestamp,
            MIN(ingestion_timestamp) AS ingestion_timestamp,
            MIN(enrichment_timestamp) AS enrichment_timestamp,
            MIN(routing_timestamp) AS routing_timestamp,
            MIN(scoring_timestamp) AS scoring_timestamp,
            CURRENT_TIMESTAMP as union_timestamp
        FROM TABLE(
            TUMBLE(
                TABLE unions, 
                DESCRIPTOR(measurement_timestamp), 
                INTERVAL '1' MINUTES
            )
        ) AS unions 
        GROUP BY patient_id, window_start
    ) as t;
\end{lstlisting}

Por último, se guarda el resultado del procesamiento en \textbf{Apache Doris} directamente desde Flink.
Para esto, es necesario que la tabla en Doris haya sido creada previamente y además definir un nombre con el que llamarla en el trabajo de procesamiento.
Luego, se puede insertar los datos y Flink y Doris acordarán la forma de hacerlo. Según las pruebas realizadas, esto se hace en batches. 
El tiempo, entre que se terminó de procesar y fue insertado en Doris no fué posible de medir ya que no se encontraró una forma de definir la fecha de inserción real.

\newpage

\begin{lstlisting}[language=SQL]
    CREATE TABLE doris_gdnews2_scores (
        patient_id STRING,
        window_start TIMESTAMP(3),
        window_end TIMESTAMP(3),

        -- AVG Raw measurements
        ...

        -- Raw NEWS2 scores
        ...
        news2_score DOUBLE,

        -- Trust gdNEWS2 scores
        ...

        news2_trust_score DOUBLE,

        -- Timestamps
        measurement_timestamp TIMESTAMP(3),
        ingestion_timestamp TIMESTAMP(3),
        enrichment_timestamp TIMESTAMP(3),
        routing_timestamp TIMESTAMP(3),
        scoring_timestamp TIMESTAMP(3),

        flink_timestamp TIMESTAMP(3),
        aggregation_timestamp TIMESTAMP(3),
        PRIMARY KEY (patient_id, window_start) NOT ENFORCED
    ) WITH (
        'connector' = 'doris',
        'fenodes' = '172.20.4.2:8030',
        'table.identifier' = 'kappa.gdnews2_scores',
        'username' = 'kappa',
        'password' = 'kappa',
        'sink.label-prefix' = 'doris_sink_gdnews2',
        'sink.properties.format' = 'json',
        'sink.properties.timezone' = 'UTC'
    );
\end{lstlisting}

\begin{lstlisting}[language=SQL]
    INSERT INTO doris_gdnews2_scores
    SELECT *
    FROM gdnews2_scores;
\end{lstlisting}
\newpage
\section{Arquitectura Delta}

\subsection{Principios de Diseño}

Todo el almacenamiento de datos a largo plazo se realiza en un formato de tabla abierta, 
que combina archivos en formato Parquet con un registro de transacciones. 
Esto garantiza propiedades ACID y permite operaciones confiables en entornos distribuidos. \newline

Al utilizar object storage, se logra una separación clara entre los recursos de procesamiento y los de almacenamiento, 
lo que permite escalarlos de manera independiente. \newline

Además, múltiples clientes pueden acceder a los mismos datos de forma simultánea sin interferencias, 
incluso utilizando herramientas diferentes, siempre que sean compatibles con el formato subyacente.\newline

El formato Parquet, al ser columnar, permite ejecutar consultas SQL de manera eficiente 
y es compatible con la mayoría de los motores de análisis modernos. \newline

El procesamiento de datos se gestiona como un flujo continuo de eventos, 
donde el motor de procesamiento utiliza el log de transacciones como fuente de verdad para mantener el estado de los datos. 
Esto permite unificar el procesamiento batch y streaming en una misma arquitectura.
Como efecto secundario de esto, todos los datos son guardados para cada etapa del flujo de procesamiento. 
Esto significa que están disponibles para su fácil consumo en caso de que se quieran analizar; 
pero como contraparte, consumen más espacio de almacenamiento ya que se almacena potencialmente varias veces el mismo dato (aunque enriquecido). \newline

Además, el sistema se encarga automáticamente de optimizaciones como la compactación de archivos pequeños
y la gestión eficiente de metadatos, asegurando un rendimiento óptimo sin intervención manual.\newline

Por último, al estar basado en estándares abiertos, el sistema evita el vendor lock-in 
y permite integración con diversas herramientas de BI, machine learning y ETL.

\newpage
\subsection{Stack Tecnológico}
Para la capa de ingesta y transporte de datos se implementó \textbf{Apache Kafka}, 
un sistema de mensajería distribuido que proporciona alta durabilidad, replicación y garantía en el orden de los eventos. 
Kafka actúa como el punto de entrada de la arquitectura, permitiendo desacoplar la ingesta de datos del procesamiento 
y asegurando una capa de buffer que absorbe picos de tráfico mientras mantiene los datos disponibles para su consumo.
En este caso, Kafka se despliega en un cluster de tres nodos, con un factor de replicación de tres y un factor de partición de tres.
Por otro lado, se definió un tiempo de retención de mensajes acotado, en este caso de 7 días, para evitar la acumulación de datos
pero a su vez, asegurar la disponibilidad de los mismos para su procesamiento.\newline

El procesamiento de datos se realiza mediante \textbf{Apache Flink}, se despliega en un cluster con un nodo Job Manager 
y cinco nodos Task Manager; de forma de distribuir la carga de trabajo lo mejor posible.
En este caso, se define como punto de entrada un tópico de Kafka, para luego continuar procesando los datos, 
no utilizando Kafka sino aprovechando las capacidades de \textbf{Apache Paimon}. 
Este adopta un enfoque log-structured para las escrituras, 
lo que lo hace especialmente eficiente para cargas de trabajo de streaming con alta frecuencia de actualizaciones.\newline

Este permite tratar una tabla de datos como un flujo continuo de eventos, funcionando de forma efectiva como 
una cola de mensajes, pero con la ventaja de tener los datos materializados en un almacenamiento persistente y barato.\newline

Por último, este formato de almacenamiento permite ser leido por \textbf{Apache Doris}, un motor de análisis de datos
distribuido que permite realizar consultas SQL en tiempo real sobre grandes volúmenes de datos con una interfaz basada en MySQL.\newline

Para esto, tanto como para el uso de Flink, se necesita definir un catálogo de tablas.
Normalmente, esto podría hacerse utilizando Apache Hive, pero se optó por simplificar el sistema tanto como sea posible, 
y dado que no se necesitaba interactuar con un sistema existente; por lo que el catálogo se almacenó en Object Storage.\newline
\newpage
Esto último se logró mediante el uso de \textbf{MinIO}, un servidor de almacenamiento de objetos de código abierto que
permite almacenar datos de forma segura y eficiente, y que además es compatible con el protocolo S3 de Amazon Web Services.\newline

Esta combinación tecnológica permite implementar efectivamente los principios de la Arquitectura Delta, 
donde los datos fluyen desde las fuentes a través de Kafka, son procesados por Flink, 
almacenados en diferentes capas mediante Paimon sobre MinIO, y finalmente consultados a través de Doris, 
manteniendo en todo momento las propiedades ACID 
y permitiendo el procesamiento continuo así como análisis retrospectivos sobre datos históricos.

\newpage
\subsection{Vista de Componentes}


\begin{figure}[h]
    \makebox[\textwidth]{\includegraphics[width=\paperwidth]{desarrollo/Delta.png}}
    \caption{Diagrama de la Arquitectura Delta}
    \label{fig:des_arquitectura_delta}
\end{figure}

\clearpage
\newpage

\subsection{Flujo de Procesamiento}

El siguiente es un ejemplo de uno de los trabajos de procesamiento de datos desarrollados:

\begin{lstlisting}[language=sql]
    SET 'execution.runtime-mode' = 'streaming';
    SET 'execution.checkpointing.mode' = 'EXACTLY_ONCE';
    SET 'table.local-time-zone' = 'UTC';
    SET 'execution.checkpointing.interval' = '60000';
    SET 'execution.checkpointing.timeout' = '30000';
    SET 'state.backend' = 'hashmap';
    SET 'table.exec.state.ttl' = '300000';
    SET 'parallelism.default' = '2';

    CREATE CATALOG paimon WITH (
        'type' = 'paimon',
        'warehouse' = 's3://datalake/paimon',
        's3.endpoint' = 'http://minio:9000',
        's3.access-key' = 'minioadmin',  
        's3.secret-key' = 'minioadmin',
        's3.path.style.access' = 'true',
        'location-in-properties' = 'true'
    );
\end{lstlisting}

\begin{lstlisting}[language=sql]
    CREATE TABLE paimon.delta.raw_measurements (
        measurement_timestamp TIMESTAMP(3),
        measurement_type STRING,
        raw_value STRING,
        device_id STRING,
        battery DOUBLE,
        signal_strength DOUBLE,
        ingestion_timestamp TIMESTAMP(3),
        WATERMARK FOR measurement_timestamp AS measurement_timestamp - INTERVAL '10' SECONDS
    );
\end{lstlisting}

\newpage

\begin{lstlisting}[language=sql]
    CREATE TABLE default_catalog.default_database.raw_measurements (
        measurement_timestamp TIMESTAMP(3),
        measurement_type STRING,
        raw_value STRING,
        device_id STRING,
        battery DOUBLE,
        signal_strength DOUBLE,
        ingestion_timestamp TIMESTAMP(3) METADATA FROM 'timestamp' VIRTUAL,
        WATERMARK FOR measurement_timestamp AS measurement_timestamp - INTERVAL '10' SECONDS
    ) WITH (
        'topic' = 'raw.measurements',
        'connector' = 'kafka',
        'properties.bootstrap.servers' = 'kafka-1:19091,kafka-2:19092,kafka-3:19093',
        'format' = 'json',
        'json.timestamp-format.standard' = 'ISO-8601',
        'scan.startup.mode' = 'latest-offset'
    );
\end{lstlisting}

\begin{lstlisting}[language=sql]
    INSERT INTO paimon.delta.raw_measurements
    SELECT * FROM default_catalog.default_database.raw_measurements;
\end{lstlisting}

Este código muestra algunas de las características principales del uso de Flink SQL en conjunto con Paimon.
Las primeras tres reglas son las estándares para un procesamiento de streaming 
con manejo de mensajes en tiempo real y que además utilice una hora estándar para tener sincronizadas
las marcas de tiempo de todos los sistemas que intervienen.\newline

Luego se define el 'execution.checkpointing.interval' que es de los parámetros más importantes para el procesamiento de streaming,
ya que define cada cuanto tiempo se guardan los estados intermedios de los datos procesados.
Para el caso de Paimon particularmente, marca cada cuanto se impactan los datos procesados en el almacenamiento,
por lo que define que tan pronto estarán disponibles estos para su consumo.\newline

Esto afecta directamente a la latencia, por lo que es algo que se tiene que balancear fuertemente con 
los recursos del sistema para evitar retrasos en el procesamiento.\newline

\newpage

Algo importante a destacar es que si bien en este caso se define el catalogo en cada uno de los scripts, 
esto no debería ser necesario pues Flink permite definir mediante configuración sus catálogos disponibles.
Sin embargo, no fue posible configurarlo de esta manera, por lo que se optó por definirlo en cada uno de los scripts.\newline

Un último detalle a destacar, es que se define el paralelismo en 2 de modo tal que se pueda aprovechar al máximo 
las particiones del tópico de Kafka. Lo ideal sería definirlo en 3, 
pero esto no es posible por una limitante de hardware en cuanto a memoria en el nodo que ejecuta el trabajo de procesamiento.\newline

Esta primer inserción de datos que se encarga de recibir los datos en formato JSON desde el tópico de Kafka
y almacenarlos en el formato de Paimon es una de las dos grandes diferencias de esta arquitectura respecto a la anterior.
La segunda diferencia es que en este caso no es necesaria la inserción en Doris, ya que esta es sólo una herramienta que se usa para acceder a los datos pero no los almacena. 
Esto hace que el flujo de procesamiento sea mas simple y liviano al no haber un tercer componente que tenga que procesar información. 
\newpage
\section{Decisiones Técnicas de Arquitectura}

En esta sección se detallan las principales decisiones de diseño arquitectónico tomadas durante la implementación del sistema,
 con foco en los aspectos de alta disponibilidad, interoperabilidad, simplicidad operativa y escalabilidad futura.

\subsection{Cluster de Kafka y ZooKeeper}
Se configuró un clúster de Kafka con 3 brokers y 3 nodos de ZooKeeper. Esta arquitectura permite:
\begin{itemize}
    \item Alta disponibilidad y tolerancia a fallos de nodos individuales
    \item Replicación de particiones con \texttt{replication.factor = 3}
    \item Tolerancia a la pérdida de hasta un nodo en cada capa (brokers o ZooKeeper)
\end{itemize}
Esta configuración es apropiada para entornos productivos y garantiza durabilidad y disponibilidad en la capa de mensajería.

\subsection{Despliegue de Apache Doris}
Se implementaron 3 nodos Backend (BE), permitiendo la distribución de datos y procesamiento paralelo. 
Sin embargo, sólo se desplegó una instancia de Frontend (FE), lo que representa un punto único de falla (\textit{Single Point of Failure}). 
En un entorno de producción, se recomienda utilizar múltiples nodos FE en configuración maestro-seguidor para garantizar disponibilidad continua del servicio de consultas y acceso al catálogo.

\subsection{Catálogo de Datos}
Se optó por utilizar un catálogo basado en sistema de archivos (object storage) para el manejo de tablas analíticas, sin integración con Hive Metastore. 
Esta decisión simplificó la infraestructura y redujo la complejidad operativa. Si bien esta elección limita la interoperabilidad entre motores de procesamiento, 
resulta adecuada para soluciones encapsuladas donde no se requiere acceso simultáneo desde múltiples tecnologías.

\subsection{Coordinación de Procesamiento}
Se utilizó una única instancia de JobManager de Apache Flink. Esta configuración es suficiente para entornos controlados, pero representa una limitación en cuanto a tolerancia a fallos. 
En escenarios reales, se recomienda implementar Flink en modo de alta disponibilidad, utilizando múltiples JobManagers (activo y pasivos), 
un backend compartido para almacenamiento de estado (por ejemplo, S3), 
y un coordinador de liderazgo como ZooKeeper.

\subsection{Justificación General}
Las decisiones tomadas responden a un enfoque de prototipo funcional, priorizando la validación de las arquitecturas Kappa y Delta en condiciones controladas. 
Se buscó un equilibrio entre fidelidad técnica y simplicidad de despliegue, permitiendo reproducibilidad y flexibilidad sin incurrir en la complejidad de un entorno productivo completo.

\newpage
\chapter{Resultados}

\section{Expectativas Iniciales}

Inicialmente, se esperaba que la arquitectura Delta proporcionara una mayor flexibilidad y escalabilidad en comparación con la arquitectura Kappa.
Sobre todo porque a nivel publicitario, en los últimos tiempos la industria de ingeniería de datos se ha encargado de promover la arquitectura Delta, 
en particular los Data Lakehouse, como la solución definitiva para el procesamiento de datos en tiempo real y batch.\newline

Por lo que a priori, se esperaba que Delta fuera más eficiente en términos de rendimiento y costos, y en simplicidad de implementación y mantenimiento.\newline

Por otro lado, dado que Delta utiliza de forma extensiva el Object Storage, se esperaba una pérdida de rendimiento, y en particular más latencia, que su contraparte. 
Sin embargo, no se esperaba una diferencia demasiado grande ya que muchas veces se propone utilizar un Data Lakehouse como un sistema de almacenamiento de datos en tiempo real.\newline

Por último, dado el apogeo de los Data Lakehouse se suponía una facilidad en el ensamblado y funcionamiento conjunto de los distintos componentes de la arquitectura,
asumiendo que hubieran soluciones estandarizadas y abiertas que permitieran la integración de los distintos componentes de forma sencilla y rápida. 

\newpage
\section{Comparación Técnica}

Desde el punto de vista técnico se eligió realizar una comparación basandose en el común denominador tecnológico para ambas arquitecturas.
Esto implica que se realizó una optimización profunda para aprovechar al máximo las capacidades de ambas. 
Sin embargo, se pudieron ver las características más notables en los ecosistemas tecnológicos en los que se basan sus implementaciones. \newline

Un caso interesante es Kafka. Kappa surge del mismísimo creador de Kafka y por lo tanto, aprovecha al máximo las capacidades de baja latencia de esta tecnología.
Aún años después de introducida, es la tecnología de mensajería y streaming más popular; habiendo cosechado una enorme comunidad. 
Esto permitió que para ambas arquitecturas se pudiera implementar un cluster altamente distribuido, con las mejores prácticas posibles. \newline

Es de destacar que inicialmente se intentó utilizar Apache Pulsar ya que es otra plataforma que cumple con los mismos casos de uso que Kafka y promete mayor escalabilidad. 
Sin embargo, no existe suficiente documentación para llegar rápidamente a la misma calidad con la que se llegó al usar Kafka. 
Otra desventaja es en su uso en conjunto con Flink ya que los conectores existentes no funcionaban correctamente. 
Es importante mencionar también, que la comunidad de Kafka es tremendamente activa 
y en el tiempo que se desarrolló esta tesis mejoraron los protocolos de comunicación entre nodos por lo que Apache Zookeeper dejó de ser necesario en las últimas versiones.\newline

También Flink fue un caso problemático ya que si bien promete unas muy buenas prestaciones para el caso de uso de streaming, 
su comunidad es muchísimo más pequeña que la de la alternativa más clara (Apache Spark) lo que implicó mucho tiempo de investigación inicial y poco de profundización en la misma. 
Teniendo en cuenta esto, a menos que el caso de uso específico requiera un procesamiento de streaming puro y tiempos de latencia del nivel de milisegundos,
se recomienda utilizar Spark ya que es más fácil de usar y tiene una comunidad mucho más activa.\newline

\newpage

Doris por su parte, es una tecnología relativamente nueva. Sumado a esto, la mayoría de la documentación está en chino, lo que dificultó mucho su implementación.
Sin embargo, las capacidades que ofrece son muy interesantes y prometedoras. Parece tener una comunidad bastante activa y en crecimiento, 
que apuesta por la adopción y evolución de esta tecnología. 
Sin embargo, no se pudo encontrar una comunidad de usuarios en inglés que permita una rápida adopción y resolución de problemas. Por lo que tampoco se pudo utilizar 
las últimas versiones de la misma y por ende se perdió la posibilidad de poner a prueba las últimas mejoras que se han ido introduciendo.\newline

A nivel tecnológico Apache Paimon, fue un hallazgo inesperado pero afortunado ya que es una herramienta que funciona nativamente con Flink (y de hecho nace como un subproyecto de esa comunidad).
Esto permitió que si bien la comunidad es pequeña y también en su mayoría en chino, fuera fácilmente integrable para la arquitectura Delta. 
También su caso de uso más importante era exactamente el que se buscaba en este caso en un formato de tabla analítica. 

De no haberlo encontrado, se hubiera tenido que utilizar Apache Iceberg, que es la alternativa más popular y que está tomando más relevancia en el mundo de la ingenería de datos. 
De todas maneras, en base a las pruebas realizadas inicialmente, Apache Hudi hubiera sido la mejor opción por su enfoque en streaming de datos. 
Su mayor problema es que tiene una comunidad mucho más pequeña que Iceberg (aunque muchisimo más grande que Paimon), que no es tan fácil de integrar con Flink 
y que por su parte también presenta complejidades en su despliegue ya que no es solamente un formato de tabla analítica sino que se proveen de forma separada servicios de Lakehouse que se deben integrar. \newline

El catálogo de metadatos fue otro punto importante a tener en cuenta. En este caso, al no necesitar integración con otras herramientas,
se optó por un catálogo en sistema de archivos. Sin embargo, normalmente sería preferible por cuestiones de integración exponer un servicio de metadatos como Apache Hive o AWS Glue.
Este es otro aspecto muy emergentes, sobre todo en el mundo de los Data Lakehouse, donde se están desarrollando estándares abiertos para la interoperabilidad entre distintas herramientas.
Sin embargo, todas las iniciativas existentes son extremadamente nuevas y no están maduras; o no son soportadas por algún otro componente de la arquitectura. 
Tal fue el caso de Project Nessie que no tenía una forma de ser integrado a Apache Doris para su uso; y que de todas maneras será absorbido por Apache Polaris en un futuro.\newline

Desde el punto de vista de su componentes técnicos y debido al enfoque de utilizar las mismas tecnologías para ambas, 
las arquitecturas son extremadamente similares y no presentan ventajas significativas una sobre otra 
(más allá de que para utilizar Paimon en Flink es necesario incluir algunos JAR adicionales).
De hecho, una vez resuelta la implementación de la arquitectura Kappa, fue trivial hacer lo mismo con la arquitectura Delta. 

\newpage
\section{Aspectos Operativos}

Se realizó una prueba de carga de ambas arquitecturas con datos sintéticos de 32 pacientes a lo largo de un año, con un total de 110.122.654 registros; 
que implica un archivo CSV de aproximadamente 7GB.
El hardware donde se realizó la prueba cuenta con 64GB de RAM y 24 núcleos de CPU, y se utilizó Docker Compose para la ejecución de las pruebas.

Variaron no solo los tiempos de carga del total del conjunto de datos, sino también el uso de recursos.

\subsection{Throughput}
El throughput se refiere a la cantidad de datos procesados por unidad de tiempo.
En este caso, se midió en base a los mismos límites en el uso de recursos, la cantidad de datos que podían ser ingestados por el sistema hasta completar la carga total.

Para \textbf{Delta} se pudieron ingestar en un promedio de 1300 registros por segundo; y la carga completa llevó 88438 segundos.
Para \textbf{Kappa} se pudieron ingestar en un promedio de 792 registros por segundo; y la carga del 61\% de los datos llevó 84879 segundos.

La carga de datos en Kappa no pudo completarse completamente ya que en todas las instancias en que se realizó, 
el hardware usado para el despliegue de la arquitectura consumió toda la memoria de su disco duro.  

Se entiende entonces que a nivel de gestión (al menos de disco), Delta es más eficiente que Kappa.
A su vez, a nivel de cantidad de mensajes que se pueden consumir por segundo, Delta fué dos veces más eficiente que Kappa.

\newpage

\subsection{Latencia}

La latencia se refiere al tiempo que tarda un dato en ser procesado por el sistema.
En este caso, se midió en base a los mismos límites en el uso de recursos, la cantidad de tiempo que tardó un dato en pasar por todo el flujo de datos.

La latencia se midió en segundos y los resultados fueron los siguientes:

\begin{longtable}{|p{3cm}|c|c|c|}
    \hline
    \textbf{Arquitectura} & \textbf{Mínima} & \textbf{Máxima} & \textbf{Promedio} \\
    \hline
    Kappa & -1 & 322 & 0.18 \\
    \hline
    Delta & 91 & 610 & 180 \\
    \hline
\end{longtable}

\begin{longtable}{|l|r|r|}
    \hline
    \textbf{Rango (segundos)} & \textbf{Registros} & \textbf{Porcentaje (\%)} \\
    \hline
    \endhead
    90-120 & 12247607 & 64,1047 \\
    \hline
    120-150 & 13210708 & 24,1907 \\
    \hline
    150-180 & 7582878 & 47,3081 \\
    \hline
    180-210 & 7892041 & 49,2370 \\
    \hline
    210-240 & 299165 & 1,8664 \\
    \hline
    240-270 & 16,24 & 8,1E-06 \\
    \hline
    270-300 & 116,86 & 6,9E-05 \\
    \hline
    330-360 & 95,61 & 9,3E-05 \\
    \hline
    390-420 & 31,87 & 6,4E-05 \\
    \hline
    420-450 & 16,24 & 8,1E-06 \\
    \hline
    600-630 & 16,24 & 8,1E-06 \\
    \hline
    \caption{Distribución de Latencia en Delta} \\
\end{longtable}

\begin{longtable}{|l|r|r|}
    \hline
    \textbf{Rango (segundos)} & \textbf{Registros} & \textbf{Porcentaje (\%)} \\
    \hline
    \endhead
    <1 & 9962080 & 81,48 \\
    \hline
    1-10 & 2264616 & 18,52 \\
    \hline
    10-60 & 60 & 0,0005 \\
    \hline
    60-180 & 17 & 0,00014 \\
    \hline
    180-300 & 17 & 0,00014 \\
    \hline
    >300 & 2 & 1,63575E-05 \\
    \hline
    \caption{Distribución de Latencia en Kappa} \\
\end{longtable}

\newpage

Es interesante ver los resultados. Ya que la latencia mínima de Kappa es negativa. 
Esto se debe a que el conector de kafka que se usa, no actualiza el campo \textbf{aggregation\_timestamp} cuando se actualiza el registro porque llegan nuevos datos.
Esto muestra que también muestra un problema de diseño del trabajo de inserción al calcular los datos pero no se encontró una solución al mismo. 
Por otro lado, se registroó el tiempo de latencia de los registros que se insertaron en el sistema; sin embargo, debido al conector de Flink con Doris, 
este proceso se realiza en batch, por lo que nuevamente es posible que el registro de tiempo de inserción sea anterior al de procesamiento. 

Por otro lado, Delta muestra una latencia de varias magnitudes superior a Kafka. Esto es esperable debido al uso extensivo de Object Storage;
ya que las lecturas en disco en conjunto con la transferencia por red suelen ser mucho más lentas que las lecturas en memoria.
Sin embargo, es importante destacar que la latencia máxima de Delta es de 610 segundos, lo que implica que en el peor de los casos,
el sistema tardó 10 minutos en procesar un dato.
Esto es un tiempo aceptable para la mayoría de los casos de uso en los que se implementan flujos de datos; 
aunque no se puede decir que lo sea para un sistema de procesamiento de datos en tiempo real. 

\newpage
\subsection{Uso de Recursos}

El uso de recursos se refiere a la cantidad de recursos que utiliza el sistema para procesar los datos.
En este caso, se midió en base a los mismos límites en el uso de recursos, la cantidad máxima de recursos que utilizó el sistema para procesar los datos.

Para \textbf{Kappa} se presentan los siguientes resultados:

\begin{longtable}{|p{3cm}|c|c|c|c|}
    \hline
    \textbf{Componente} & \textbf{Uso de CPU} & \textbf{Memoria} & \textbf{Almacenamiento} & \textbf{Transferencia} \\
    \hline
    Job Manager & 134\% & 1.8 GB & - & 54 GB \\
    \hline
    Task Manager 1 & 260\% & 6.2 GB & - & 230 GB \\
    \hline
    Task Manager 2 & 244\% & 6.3 GB & - & 252 GB \\
    \hline
    Task Manager 3 & 272\% & 6.2 GB & - & 243 GB \\
    \hline
    Task Manager 4 & 256\% & 6.4 GB & - & 240 GB \\
    \hline
    Kafka 1 & 250\% & 8.2 GB & 234 GB & 255 GB \\
    \hline
    Kafka 2 & 282\% & 8.4 GB & 234 GB & 255 GB \\
    \hline
    Kafka 3 & 268\% & 8.5 GB & 234 GB & 255 GB \\
    \hline
    Doris Frontend & 370\% & 2.4 GB & - & 0.55 GB \\
    \hline
    Doris Backend 1 & 374\% & 4.2 GB & 0.234 GB & 0.1 GB \\
    \hline
    Doris Backend 2 & 345\% & 4.2 GB & 0.234 GB & 0.1 GB \\
    \hline
    Doris Backend 3 & 356\% & 4.3 GB & 0.234 GB & 0.1 GB \\
    \hline
    MinIO & 127\% & 4.2 GB & 384 GB & 386 GB \\
    \hline
\end{longtable}

Para \textbf{Delta} se presentan los siguientes resultados:

\begin{longtable}{|p{3cm}|c|c|c|c|}
    \hline
    \textbf{Componente} & \textbf{Uso de CPU} & \textbf{Memoria} & \textbf{Almacenamiento} & \textbf{Transferencia} \\
    \hline
    Job Manager & 250\% & 3.5 GB & - & 5.9 GB \\
    \hline
    Task Manager 1 & 208\% & 9.7 GB & - & 86.1 GB \\
    \hline
    Task Manager 2 & 166\% & 7.6 GB & - & 45.2 GB \\
    \hline
    Task Manager 3 & 194\% & 8.5 GB & - & 40.2 GB \\
    \hline
    Task Manager 4 & 206\% & 7.5 GB & - & 10.9 GB \\
    \hline
    Kafka 1 & 234\% & 2.7 GB & 38 GB & 101.8 GB \\
    \hline
    Kafka 2 & 274\% & 2.7 GB & 38 GB & 101.9 GB \\
    \hline
    Kafka 3 & 176\% & 2.7 GB & 38 GB & 102.1 GB \\
    \hline
    Doris Frontend & 365\% & 3.6 GB & - & 1.1 GB \\
    \hline
    Doris Backend 1 & 331\% & 4.1 GB & 0 GB & 0.2 GB \\
    \hline
    Doris Backend 2 & 351\% & 4.1 GB & 0 GB & 0.2 GB \\
    \hline
    Doris Backend 3 & 340\% & 4.1 GB & 0 GB & 0.2 GB \\
    \hline 
    MinIO & 127\% & 3.8 GB & 8.4 GB & 374 GB \\
    \hline
\end{longtable}

\newpage

Acorde con lo esperado, Kappa requiere un mayor uso de recursos para los nodos de Kafka que Delta.
Por su lado, Delta requiere un mayor uso de recursos para los nodos de procesamiento. También para el nodo de frontend de Doris. 
Aunque es notablemente menor el uso de memoria que requiere de los nodos Kafka.

El uso de CPU es similar en las dos arquitecturas, aunque Delta tiene más diferencias entre nodos del mismo tipo. 
Probablemente sea debido a un desbalanceo en el paralelismo de los trabajos de procesamiento. 

La sorpresa más grande es el uso de almacenamiento y de transferencia de red. 
Kappa requiere un uso de almacenamiento mucho mayor que Delta, y también un uso de transferencia de red mucho mayor.
Esto se debe a que Delta guarda sus datos en Parquet, lo que lo hace muchísimo más eficiente que el formato de Kafka.
Incluso, en el caso de Delta, todas las transformaciones de los datos están guardadas en Object Storage 
y aún así el uso de almacenamiento es notablementemenor que en Kappa. 

\newpage

\subsection{Resultados}

A nivel de throughput, Delta es más eficiente que Kappa. No solo eso sino que es más fácil de operar y consume menos recursos.
A nivel de latencia, Kappa es más eficiente que Delta. Sin embargo, la latencia máxima de Delta es aceptable para la mayoría de los casos de uso.
En cuanto a uso de memoria y procesamiento, ambas son similares. Aunque Delta es muchísimo mas eficiente en el uso de almacenamiento y transferencia de red. 

\newpage
\section{Costos}

\subsection{Costos para Delta}

\subsubsection{Selección de Tipos de Instancias AWS}

Basados en los requisitos de recursos, seleccionaremos los tipos de instancias EC2 apropiados:

\textbf{Componentes Flink (Job Manager y Task Managers)}

\begin{itemize}
    \item \textbf{Job Manager}: Requiere 2.5 núcleos y 3.5GB RAM
    \begin{itemize}
        \item Instancia recomendada: \texttt{c5.large} (2 vCPU, 4GB RAM)
    \end{itemize}
    
    \item \textbf{Task Managers}: Requieren 1.66-2.08 núcleos y 7.5-9.7GB RAM
    \begin{itemize}
        \item Instancia recomendada: \texttt{r5.large} (2 vCPU, 16GB RAM) para todos los task managers
        \item Aunque podríamos usar instancias más pequeñas para algunos task managers, usar el mismo tipo de instancia simplifica la gestión
    \end{itemize}
\end{itemize}

\subsubsection{Brokers Kafka}

\begin{itemize}
    \item \textbf{Nodos Kafka}: Requieren 1.76-2.74 núcleos y 2.7GB RAM cada uno
    \begin{itemize}
        \item Instancia recomendada: \texttt{c5.large} (2 vCPU, 4GB RAM) para todos los nodos Kafka
        \item Cada uno necesita 38GB de almacenamiento, que aprovisionaremos con volúmenes EBS gp3
    \end{itemize}
\end{itemize}

\subsubsection{Apache Doris}

\begin{itemize}
    \item \textbf{Doris Frontend}: Requiere 3.65 núcleos y 3.6GB RAM
    \begin{itemize}
        \item Instancia recomendada: \texttt{c5.2xlarge} (8 vCPU, 16GB RAM)
    \end{itemize}
    
    \item \textbf{Doris Backends}: Requieren 3.31-3.51 núcleos y 4.1GB RAM cada uno
    \begin{itemize}
        \item Instancia recomendada: \texttt{c5.2xlarge} (8 vCPU, 16GB RAM) para todos los nodos backend
        \item Actualmente muestra 0GB de almacenamiento, pero deberíamos asignar algo de almacenamiento EBS para crecimiento
    \end{itemize}
\end{itemize}

\subsubsection{Servicio de Almacenamiento}

\begin{itemize}
    \item \textbf{S3}: Reemplazando MinIO con Amazon S3 para almacenamiento de objetos (8.4GB inicialmente)
    \begin{itemize}
        \item No se requiere instancia EC2 ya que es un servicio gestionado
    \end{itemize}
\end{itemize}

\subsection{Cálculo de Costos Mensuales}

\subsubsection{Costos de Instancias EC2 (Precios bajo demanda en us-east-1)}

\begin{itemize}
    \item \texttt{c5.large}: \$0.085 por hora
    \item \texttt{r5.large}: \$0.126 por hora
    \item \texttt{c5.2xlarge}: \$0.34 por hora
\end{itemize}

\begin{table}[h]
\centering
\begin{tabular}{|l|l|c|r|}
\hline
\textbf{Componente} & \textbf{Tipo de Instancia} & \textbf{Cantidad} & \textbf{Costo Mensual} \\
\hline
Job Manager & c5.large & 1 & \$0.085 × 730 = \$62.05 \\
\hline
Task Managers & r5.large & 4 & \$0.126 × 4 × 730 = \$367.92 \\
\hline
Nodos Kafka & c5.large & 3 & \$0.085 × 3 × 730 = \$186.15 \\
\hline
Doris Frontend & c5.2xlarge & 1 & \$0.34 × 730 = \$248.20 \\
\hline
Doris Backends & c5.2xlarge & 3 & \$0.34 × 3 × 730 = \$744.60 \\
\hline
\textbf{Total EC2} & & & \textbf{\$1,608.92} \\
\hline
\end{tabular}
\end{table}

\subsubsection{Costos de Almacenamiento EBS}

\begin{itemize}
    \item Volumen gp3: \$0.08 por GB-mes
\end{itemize}

\begin{table}[h]
\centering
\begin{tabular}{|l|c|c|r|}
\hline
\textbf{Componente} & \textbf{Tamaño de Almacenamiento} & \textbf{Cantidad} & \textbf{Costo Mensual} \\
\hline
Nodos Kafka & 38 GB & 3 & \$0.08 × 38 × 3 = \$9.12 \\
\hline
Doris Backends & 20 GB (mínimo) & 3 & \$0.08 × 20 × 3 = \$4.80 \\
\hline
\textbf{Total EBS} & & & \textbf{\$13.92} \\
\hline
\end{tabular}
\end{table}

\subsubsection{Costos de Almacenamiento S3}

\begin{itemize}
    \item Almacenamiento S3 Estándar: \$0.025 por GB-mes
    \item Requisito de almacenamiento S3: 8.4 GB
    \item Costo de almacenamiento S3: \$0.025 × 8.4 = \$0.21 por mes
\end{itemize}

\subsubsection{Costos de Transferencia de Datos de Red}

\begin{itemize}
    \item Dentro de la misma AZ: Gratis
    \item Entre AZs en la misma región: \$0.01 por GB
    \item Transferencia total de datos: $\sim$622.6 GB
\end{itemize}

Según la suposición actualizada, el 25\% del tráfico está dentro de la misma AZ y el 75\% está entre AZs:
\begin{itemize}
    \item Costo entre AZs: 622.6 GB × 0.75 × \$0.01 = \$4.67 por mes
\end{itemize}

\subsubsection{Costos de EKS}

\begin{itemize}
    \item Clúster EKS: \$0.10 por hora = \$73 por mes
\end{itemize}

\subsubsection{Costo Mensual Total}

\begin{table}[h]
\centering
\begin{tabular}{|l|r|}
\hline
\textbf{Categoría de Costo} & \textbf{Importe Mensual} \\
\hline
Instancias EC2 & \$1,608.92 \\
\hline
Almacenamiento EBS & \$13.92 \\
\hline
Almacenamiento S3 & \$0.21 \\
\hline
Transferencia de Datos de Red & \$4.67 \\
\hline
Clúster EKS & \$73.00 \\
\hline
\textbf{Costo Mensual Total Estimado} & \textbf{\$1,700.72} \\
\hline
\end{tabular}
\end{table}

\subsection{Suposiciones y Consideraciones}

\begin{enumerate}
    \item Los porcentajes de CPU son relativos a un solo núcleo (100\% = 1 núcleo)
    \item Hemos colocado componentes en instancias EC2 separadas para aislamiento y escalado
    \item Los patrones de tráfico de red son estimados
    \item Se asumen volúmenes gp3 estándar
    \item Los precios se basan en tarifas bajo demanda en la región us-east-1 a abril de 2025
\end{enumerate}

\subsection{Casos de Uso Óptimo}

\section{Evolución a Futuro}

\subsection{Tecnologías Emergentes}
\chapter{Conclusiones}
Contenido del capítulo...

% Anexos
\appendix
\chapter{Primer Anexo}
Contenido del anexo...


% Bibliografía
\printbibliography
\end{document}