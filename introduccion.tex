\chapter{Introducción}


\section{Descripción del Proyecto}

Este proyecto compara las arquitecturas Kappa y Delta en el contexto del monitoreo remoto de pacientes mediante sensores IoT. En la era de la salud digital, estos sistemas generan grandes volúmenes de datos en tiempo real que requieren procesamiento eficiente. Se analizarán ambas arquitecturas, se definirán métricas de comparación y se implementarán en un caso de uso de monitoreo de pacientes. El objetivo es determinar la arquitectura más adecuada, considerando factores como latencia, escalabilidad, complejidad de implementación y manejo de datos históricos y en tiempo real.

\newpage

\section{Objetivos}

\begin{enumerate}
    \item Realizar un análisis teórico exhaustivo de las arquitecturas Kappa y Delta, detallando sus componentes y flujos de datos.
    
    \item Definir un conjunto de métricas y criterios para la comparación objetiva de ambas arquitecturas en el contexto del monitoreo remoto de pacientes.
    
    \item Implementar ambas arquitecturas utilizando un conjunto de datos simulado de sensores de monitoreo de pacientes.
    
    \item Ejecutar pruebas de rendimiento y funcionalidad en ambas implementaciones.
    
    \item Analizar los resultados obtenidos y determinar la arquitectura más adecuada para el caso de uso específico de monitoreo remoto de pacientes.
\end{enumerate}

\newpage

\section{Caso de Estudio: Big Data en Sistema de Salud}

\subsection{Contexto del Sistema}
Sistema de salud integral que incluye perfiles de pacientes, telemedicina e integración con dispositivos IoT. El objetivo es mejorar la atención médica, con foco en prevencion, utilizando tecnologia.

\subsection{Descripción del Caso de Uso}
Monitoreo continuo y en tiempo real de la salud del paciente mediante el uso de Big Data. Se espera ademas, tener la capacidad de identificar patrones y tendencias en los datos medicos. Asi como tambien proporcionar recomendaciones personalizadas.

\subsection{Proceso}
\begin{enumerate}
    \item \textbf{Recopilación de Datos}:
    \begin{itemize}
        \item Dispositivos IoT (datos en tiempo real)
    \end{itemize}
    
    \item \textbf{Almacenamiento y Gestión}:
    \begin{itemize}
        \item Almacenamiento de datos centralizada, segura y escalable
    \end{itemize}
    
    \item \textbf{Análisis de Datos}:
    \begin{itemize}
        \item Procesamiento en tiempo real
        \item Análisis históricos
    \end{itemize}
    
    \item \textbf{Generación de Insights}:
    \begin{itemize}
        \item Tableros
        \item Alertas en tiempo real
    \end{itemize}
    
    \item \textbf{Intervención y Seguimiento}:
    \begin{itemize}
        \item Monitoreo continuo
        \item Feedback y mejora continua del sistema
    \end{itemize}
\end{enumerate}