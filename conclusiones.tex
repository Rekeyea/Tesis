\chapter{Conclusiones}

\section{Conclusiones Generales}

El objetivo de este trabajo fue comparar las arquitecturas Kappa y Delta en el contexto del monitoreo remoto de pacientes mediante sensores.
Se realizó un análisis exhaustivo de ambas arquitecturas, considerando sus componentes, flujos de datos y características técnicas.\newline

Se definieron métricas y criterios para la comparación objetiva de ambas arquitecturas, y se implementaron en un caso de uso simulado.
Se implementó un sistema de monitoreo remoto de pacientes utilizando ambas arquitecturas, y se realizaron pruebas de rendimiento y funcionalidad.
Los resultados obtenidos mostraron que ambas arquitecturas tienen ventajas y desventajas en diferentes contextos.\newline

La arquitectura Kappa es más adecuada para el procesamiento de datos en tiempo real, con baja latencia, pero requiere más recursos y es más compleja de implementar.
La arquitectura Delta, por otro lado, es más escalable y eficiente en el uso de recursos, pero tiene tiempos de latencia más altos, lo que la hace menos adecuada para el procesamiento en tiempo real.\newline

En un sistema de salud integral, se podría combinar ambas arquitecturas, utilizando Delta como fuente de verdad y un subsistema basado en Kappa para el monitoreo intrahospitalario, donde la latencia es crítica.
Esta combinación permitiría aprovechar las ventajas de ambas arquitecturas y mejorar la eficiencia del sistema en su conjunto.
\newpage
\section{Trabajo Futuro}

Este trabajo puede abrir las puertas a futuras investigaciones en el área de monitoreo remoto de pacientes y análisis de datos en tiempo real.
Se pueden explorar diversas direcciones de investigación:

\begin{itemize}
    \item \textbf{Optimización de Parámetros:} Investigar la optimización de ventanas de tiempo y tasas de degradación para mejorar la validez de las mediciones
    \item \textbf{Mejora de la Evaluación de Calidad:} Desarrollar métricas de calidad más precisas y protocolos de validación para nuevos tipos de dispositivos
    \item \textbf{Validación Clínica:} Comparar la efectividad del sistema gdNEWS2 con métodos tradicionales en entornos clínicos
    \item \textbf{Optimización del Sistema:} Investigar parámetros de rendimiento y capacidades de integración para mejorar la eficiencia del sistema
    \item \textbf{Tecnologías Emergentes:} Explorar el uso de tecnologías emergentes como inteligencia artificial y aprendizaje automático para mejorar la toma de decisiones en tiempo real
    \item \textbf{Interoperabilidad:} Investigar la interoperabilidad entre diferentes sistemas y dispositivos para mejorar la integración de datos
    \item \textbf{Seguridad y Privacidad:} Evaluar la seguridad y privacidad de los datos en el contexto del monitoreo remoto de pacientes
    \item \textbf{Experiencia del Usuario:} Investigar la experiencia del usuario y la usabilidad de las interfaces de monitoreo remoto
    \item \textbf{Gobernanza de Datos:} Desarrollar políticas y prácticas para la gobernanza de datos en el contexto del monitoreo remoto de pacientes
    \item \textbf{Despliegue Contínuo: } Explorar las diferentes herramientas y sus limitaciones en el despliegue continuo de sistemas de streaming
    \item \textbf{Comparación con otras arquitecturas:} Comparar la arquitectura Delta con otras arquitecturas de procesamiento de datos o arquitecturas híbridas
    \item \textbf{Evaluación en un entorno real:} Realizar una evaluación del sistema en un entorno real, con datos reales y pacientes reales
\end{itemize}
\newpage
\section{Reflexiones Finales}

Este trabajo me ha permitido explorar y comparar las arquitecturas Kappa y Delta en el contexto del monitoreo remoto de pacientes.
Esta es un área en constante evolución y crecimiento, y la implementación de estas arquitecturas puede tener un impacto significativo en la atención médica.\newline

Además, desde el punto de vista académico, este trabajo me ha permitido profundizar en el análisis de datos en tiempo real.
Esto implicó la investigación, aprendizaje y aplicación de diferentes herramientas y tecnologías que no sólo aplican 
a este trabajo sino que podré aplicar en el ámbito profesional.\newline

Este proceso ha sido desafiante porque he tenido que aprender a utilizar herramientas y tecnologías que no conocía previamente 
y que no están tan difundidas como aquellas con las que trabajo habitualmente.\newline

Sin embargo, esto me ha permitido adquirir nuevas habilidades y conocimientos que serán valiosos en mi carrera profesional.\newline

Por último, ha sido un proceso gratificante y espero poder seguir profundizando en esta área en el futuro.