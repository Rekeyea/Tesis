\chapter{Conclusiones}
Contenido del capítulo...



% # 7. Future Research Directions

% ## 7.1 Parameter Optimization

% ### Time Windows and Degradation Rates
% - Empirical validation of freshness weight decay rates
% - Patient-specific adaptation of maximum time windows
% - Impact of circadian rhythms on measurement validity periods
% - Optimization of quality weight component ratios

% ### Alert Thresholds
% - Machine learning approaches for dynamic threshold adjustment
% - Population-based threshold optimization
% - Context-aware alert modification
% - Alert fatigue reduction strategies

% ## 7.2 Quality Assessment Enhancement

% ### Device Quality Metrics
% - Validation protocols for new device types
% - Cross-device correlation studies
% - Environmental impact on device accuracy
% - Long-term drift detection methods

% ### Measurement Conditions
% - Automated detection of measurement interference
% - Impact of patient activity on measurement quality
% - Environmental factor quantification
% - Multi-sensor fusion for condition assessment

% ## 7.3 Clinical Validation

% ### gdNEWS2 Effectiveness
% - Comparison with traditional NEWS2 in clinical settings
% - Impact on patient outcomes
% - False positive/negative rate analysis
% - Cost-benefit analysis of implementation

% ### Score Adjustment Mechanisms
% - Alternative degradation models
% - Non-linear quality weight impacts
% - Parameter interdependency effects
% - Confidence level optimization

% ## 7.4 System Optimization

% ### Performance Parameters
% - Optimal measurement frequencies
% - Data validation thresholds
% - Alert buffer periods
% - System recovery procedures

% ### Integration Capabilities
% - Multi-device synchronization methods
% - Data format standardization
% - Legacy system compatibility
% - Real-time processing optimization

% These research directions aim to enhance the system's clinical validity while maintaining its practical implementability. Each area represents an opportunity for focused studies that could lead to meaningful improvements in remote patient monitoring capabilities.