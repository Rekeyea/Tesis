\section{Criterios de Evaluación}

Para evaluar y comparar las arquitecturas Kappa y Delta en el contexto del monitoreo remoto de pacientes, se considerarán creiterios unificados y medibles.
Estos servirán como base para una evaluación objetiva de las arquitecturas Kappa y Delta en el contexto del monitoreo remoto de pacientes, permitiendo tomar criterios 
fundamentados para la elección de una u otra según el escenario.

\subsection{Preguntas de Investigación}

Este trabajo busca responder las siguientes preguntas de investigación, comparando las arquitecturas Kappa y Delta \textbf{bajo condiciones de igualdad tecnológica y metodológica}; es decir, utilizando un stack tecnológico y técnicas de programación similares, y sin llevar a las arquitecturas a su límite teórico:

\begin{enumerate}
    \item ¿Cuál de las arquitecturas (Kappa o Delta) ofrece una menor latencia y un mayor throughput cuando se implementan en un entorno de monitoreo remoto de pacientes utilizando las mismas tecnologías y patrones de desarrollo?
    \item ¿Cómo se comparan ambas arquitecturas en términos de eficiencia de almacenamiento y costos operativos, manteniendo constante el stack tecnológico y el contexto de aplicación?
    \item ¿Qué ventajas y limitaciones prácticas surgen en la integración y el análisis histórico de datos en tiempo real al aplicar cada arquitectura bajo condiciones experimentales equivalentes?
    \item ¿Qué desafíos específicos aparecen al implementar Kappa y Delta con tecnologías open source y metodologías homogéneas en un entorno simulado de monitoreo de pacientes, y cómo influyen estos desafíos en los resultados obtenidos?
\end{enumerate}

\newpage

\subsection{Latencia y Rendimiento}

Se implementarán mediciones a través de puntos de instrumentación estratégicos a lo largo de los componentes de la arquitectura.  
Estos puntos de medición incluirían timestamps en los mensajes, métricas de procesamiento en los componentes intermedios, y el tiempo de escritura/lectura en la capa final. \newline

Las métricas serán tomadas utilizando Prometheus y mostradas en un tablero de Grafana.\newline

Las métricas a considerar serán:
\begin{itemize}
    \item Tiempo de ingesta de histórico
    \item Latencia de procesamiento
\end{itemize}

\subsection{Manejo de Datos Históricos}

Se medirán especificamente la efectividad con la que el sistema permite consultar datos históricos y reprocesar toda la historia.
Las métricas serán tomadas utilizando Prometheus y mostradas en un tablero de Grafana.

\begin{itemize}
    \item Tiempo de reprocesamiento de la historia completa
    \item Uso de recursos en reprocesamiento de la historia completa
\end{itemize}

Se omitirá la implementación de un cambio en el modelo de procesamiento que implique reprocesar la historia completa, 
pero se dará una propuesta de como hacerlo y se evaluará las implicancias de su puesta en producción manteniendo los dos sistemas funcionando e integrandolos eventualmente.

\newpage

\subsection{Costos Operativos}

La implementación del sistema en contenedores, permitirá un monitoreo del uso de los recursos del sistema.
Además, se plantearán cálculos, utilizando las calculadoras de costo que proveen los sistemas de nube, que permitan dimensionar el costo operativo de estos sistemas.\newline

Las métricas serán tomadas utilizando Prometheus y mostradas en un tablero de Grafana.\newline

Las métricas a considerar serán:
\begin{itemize}
    \item Uso de memoria RAM
    \item Uso de CPU
    \item Uso de disco
    \item Uso de red
\end{itemize}

\newpage