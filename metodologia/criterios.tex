\section{Criterios de Evaluación para Arquitecturas de Streaming}

Para evaluar y comparar las arquitecturas Kappa y Delta en el contexto del monitoreo remoto de pacientes, se considerarán los siguientes criterios:

\subsection{Latencia de Procesamiento}
\begin{itemize}
    \item Tiempo de respuesta para el procesamiento de datos en tiempo real
    \item Capacidad para manejar picos de datos sin aumentar significativamente la latencia
\end{itemize}

\subsection{Escalabilidad}
\begin{itemize}
    \item Capacidad para manejar un aumento en el volumen de datos
    \item Facilidad de agregar recursos computacionales según sea necesario
    \item Rendimiento bajo diferentes cargas de trabajo
\end{itemize}

\subsection{Consistencia de Datos}
\begin{itemize}
    \item Garantía de consistencia entre datos en tiempo real y datos históricos
    \item Manejo de datos fuera de orden o retrasados
\end{itemize}

\subsection{Tolerancia a Fallos}
\begin{itemize}
    \item Capacidad de recuperación ante fallos del sistema
    \item Prevención de pérdida de datos en caso de interrupciones
\end{itemize}

\subsection{Manejo de Datos Históricos}
\begin{itemize}
    \item Eficiencia en el acceso y análisis de datos históricos
    \item Capacidad para reprocesar datos históricos cuando sea necesario
\end{itemize}

\subsection{Costo Operativo}
\begin{itemize}
    \item Requisitos de hardware y software
    \item Costos de mantenimiento y operación a largo plazo
\end{itemize}

\subsection{Seguridad y Cumplimiento Normativo}
\begin{itemize}
    \item Capacidad para cifrar datos en tránsito y en reposo
    \item Cumplimiento con regulaciones de protección de datos en salud (por ejemplo, HIPAA)
\end{itemize}

\subsection{Rendimiento en Análisis Complejos}
\begin{itemize}
    \item Capacidad para realizar análisis en tiempo real de múltiples fuentes de datos
    \item Eficiencia en la ejecución de modelos de machine learning
\end{itemize}

Estos criterios servirán como base para una evaluación exhaustiva y objetiva de las arquitecturas Kappa y Delta en el contexto del monitoreo remoto de pacientes, permitiendo una comparación detallada y fundamentada.