\section{Despliegue}

Se realizará un despliegue de ambas arquitecturas, manteniendo la misma cantidad de nodos de cada tecnología.
Para esto se utilizarán contenedores con la tecnología Docker Compose dentro de una única máquina virtual linux.
El sistema donde se desplegará el sistema cuenta con 16 núcleos, 64 GB de RAM y 2 TB de disco duro de estado sólido.
El sistema operativo utilizado es Ubuntu 22.04 LTS.

\subsection{Componentes}

Se desplegarán los siguientes componentes: 

\begin{longtable}{|p{6cm}|p{6cm}|}
    \hline
    \textbf{Componente} & \textbf{Cantidad} \\
    \hline
    \endhead
    Broker Apache Kafka & 3 \\ 
    \hline
    Zookeeper & 3 \\ 
    \hline
    Apache Flink Job Manager & 1 \\ 
    \hline
    Apache Flink Task Manager & 4 \\ 
    \hline
    Apache Doris Frontend & 1 \\ 
    \hline
    Apache Doris Backend & 3 \\ 
    \hline
    MinIO & 1 \\ 
    \hline
    Grafana & 1 \\ 
    \hline
    Prometheus & 1 \\ 
    \hline 
    \caption{Cantidad de componentes desplegados}
    \label{tab:deployed_components}
\end{longtable}

El objetivo de esta configuración es proveer de un entorno de alta disponibilidad que a su vez permita escalar horizontalmente y
se adecúe al ambiente de pruebas.

\newpage

\subsection{Configuración de los componentes}

Se utilizarán configuraciones por defecto para cada uno de los componentes desplegados.
Sin embargo, se realizán los cambios necesarios para asegurar el correcto funcionamiento de los mismos.\newline

Se deberá asegurar la mayor cantidad de recursos del sistema para cada uno de los componentes,
dándole mayor importancia a los componentes que se encargan de la ingestión y procesamiento de datos.\newline

De esta forma, se deberá asegurar el procesamiento ininterrumpido de los datos.