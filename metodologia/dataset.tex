\section{Conjunto de Datos}

\subsection{Proceso de Análisis}

El objetivo es que se pueda utilizar el sistema como plataforma para poder darle mejor seguimiento a los pacientes ambulatorios,
así como también ingresados a los centros de salud. 
Por esto, se propone tomar como referencia la implementación de los sistemas de alerta temprana para diversas áreas: 

\begin{itemize}
    \item MEWS como sistema de alerta de pacientes fuera de una instituición de salud
    \item NEWS2 como sistema de alerta de pacientes ingresados en una institución de salud
    \item qSOFA como sistema de alerta de pacientes en ingresados en una Unidad de Cuidados Intensivos
\end{itemize}

Se modularizará la lógica tanto como sea posible, de modo de reducir problemas de medición teniendo dos implementaciones diferentes. 

\subsection{Conjunto de Datos}

El conjunto de datos utilizado para evaluar el sistema será de datos sintéticos. 
Se utilizará la plataforma Node-RED para la creación de una red de nodos de dispositivos virtuales que enviarán datos a los distintos sistemas. 
Se buscará generar un fuerte volúmen de datos, del orden de los 10 GB para poder tener una buena evaluación de los criterios de evaluación de los sistemas. 

Existen dos razones principales para el uso de datos sintéticos: 

\begin{itemize}
    \item Los conjuntos de datos de sensores disponibles son heterogéneos y muy pequeños; lo que de todos modos implicaría generar datos sintéticos.
    \item No es necesario contar con datos que muestren tendencias reales, sino que se comporten como lo harían en realidad para demostrar los atributos de calidad.
\end{itemize}

Para esto, se definirá un flujo de datos para cada sensor que será interconectado pero definiendo comportamientos especificos que tengan que ver con la naturaleza de su medición.
Los signos vitales tomados para esto son: 

\begin{itemize}
    \item Frecuencia respiratoria
    \item Presión sistólica
    \item Frecuencia cardíaca
    \item Temperatura
    \item Saturación de oxígeno
    \item Nivel de conciencia en escala AVPU
    \item Nivel de conciencia en escala Glasgow
\end{itemize}