\section{Conjunto de Datos}

El conjunto de datos utilizado para evaluar el sistema será de datos sintéticos. 
Se desarrollará un script en Python que genere datos sintéticos para los sensores de signos vitales.\newline

Estos datos serán generados de acuerdo a la naturaleza de la medición de cada sensor y se interconectarán para simular un flujo de datos continuo.
El script generará datos para tres tipos de pacientes: Sanos, Sanos pero que se deterioran con el tiempo, y enfermos pero estables. 
De esta manera, se podrá evaluar el sistema en diferentes escenarios.\newline

Existen dos razones principales para el uso de datos sintéticos: 

\begin{itemize}
    \item Los conjuntos de datos de sensores disponibles son heterogéneos y muy pequeños; lo que de todos modos implicaría generar datos sintéticos.
    \item No es necesario contar con datos que muestren tendencias reales, sino que se comporten como lo harían en realidad para demostrar los atributos de calidad del sistema.
\end{itemize}

Para esto, se definirá un flujo de datos para cada sensor que será interconectado pero definiendo comportamientos especificos que tengan que ver con la naturaleza de su medición.
Los signos vitales tomados para esto son: 

\begin{itemize}
    \item Frecuencia respiratoria
    \item Presión sistólica
    \item Frecuencia cardíaca
    \item Temperatura
    \item Saturación de oxígeno
    \item Nivel de conciencia en escala APVU
\end{itemize}