\section{Stack de Tecnologías a Utilizar}

Para que la implementación de las instancias de arquitecturas de referencia sean comparables es necesario mantener criterios similares para no favorecer a una de ellas.
Estas instancias serán definidas mediante contenedores y desplegadas en Kubernetes de forma local utilizando la herramienta Minikube; 
lo que permitirá definir un ambiente Cloud-Native para la comparación de las mismas.

Se realizó un análisis exhaustivo de las tecnologías y se decidió por: 

\begin{itemize}
    \item Apache Pulsar como punto de ingestión de datos y ruteo de mensajes
    \item Apache Flink como motor de procesamiento distribuido
    \item MinIO como almacenamiento de objetos
    \item Apache Pinot como base de datos de análisis
    \item Apache Superset como tablero que consume de los sistemas implementados
    \item Apache Iceberg como formato de tabla analítica
    \item Apache Airflow como gestor de procesos de mantenimiento
    \item Prometheus para monitoreo de métricas de los sistemas
    \item Grafana para creación de tableros de métricas comparativas
    \item K6 para pruebas de carga
\end{itemize}
\newpage