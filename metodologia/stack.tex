\section{Stack de Tecnologías a Utilizar}

Para que la implementación de las instancias de arquitecturas de referencia sean comparables es necesario mantener criterios similares para no favorecer a una de ellas.
Estas instancias serán definidas mediante contenedores y desplegadas en Kubernetes de forma local utilizando la herramienta Minikube; 
lo que permitirá definir un ambiente Cloud-Native para la comparación de las mismas.

Se realizó un análisis exhaustivo de las tecnologías, de modo de utilizar aquellas que puedan ser compartidas entre ambas implementaciones y a su vez permitan
optimizar su uso. Las seleccionadas son: 

\begin{itemize}
    \item Apache Kafka como punto de ingestión de datos y ruteo de mensajes
       \begin{itemize}
           \item Estandard de la industria
           \item Ecosistema maduro con fuertes integraciones
       \end{itemize}
    \item Apache Flink como motor de procesamiento distribuido
       \begin{itemize}
           \item Su modelo de procesamiento ofrece la menor latencia posible
           \item Tiene amplia compatibilidad con los diferentes ecosistemas de tecnologías
       \end{itemize}
    \item MinIO como almacenamiento de objetos
       \begin{itemize}
           \item Compatible con API S3
           \item Alta disponibilidad
           \item Gestión de datos distribuida
       \end{itemize}
    \item Apache Iceberg como formato de tabla analítica
       \begin{itemize}
           \item Es el formato de tabla más extendido
           \item Ecosistema diverso para integración con otras tecnologías
       \end{itemize}
    \item Apache Doris como base de datos de análisis
       \begin{itemize}
           \item Consultas con baja latencia
           \item Tablas en tiempo real
           \item Integración con Apache Iceberg mediante un catálogo
           \item Tablas híbridas
       \end{itemize}
    \item Apache Superset como tablero que consume de los sistemas implementados
       \begin{itemize}
           \item Visualizaciones interactivas
           \item Múltiples fuentes de datos
       \end{itemize}
    \item Apache Airflow como gestor de procesos de mantenimiento
       \begin{itemize}
           \item Orquestación de tareas
           \item Monitoreo de flujos de trabajos
       \end{itemize}
    \item Prometheus para monitoreo de métricas de los sistemas
       \begin{itemize}
           \item Estandard de la industria
           \item Muy buena integración con diferentes ecosistemas
       \end{itemize}
    \item Grafana para creación de tableros de métricas comparativas
       \begin{itemize}
            \item Estandard de la industria
            \item Tableros personalizables
       \end{itemize}
    \item K6 para pruebas de carga
       \begin{itemize}
           \item Scripts en JavaScript
           \item Métricas detalladas
           \item Escenarios personalizables
       \end{itemize}
    \item Kubernetes para el despliegue de los componentes del sistema
       \begin{itemize}
           \item Orquestación de contenedores
           \item Auto-escalado
           \item Alta disponibilidad
       \end{itemize}
   \end{itemize}
\newpage