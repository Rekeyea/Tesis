\section{Stack de Tecnologías a Utilizar}

Con el objetivo de comparar objetivamente las arquitecturas Kappa y Delta, 
se definió un stack tecnológico común que permite mantener condiciones similares en términos de despliegue, 
monitoreo y operación.\newline 

Estas tecnologías fueron seleccionado considerando aspectos como compatibilidad, 
madurez del ecosistema, integración con otras herramientas, facilidad de operación, 
disponibilidad de documentación y eficiencia de recursos.

\subsection*{Sistema de Mensajería - Apache Kafka}

Apache Kafka fue elegido como el punto de entrada del sistema por ser el estándar de facto en mensajería distribuida en arquitecturas de datos modernas. 
Su arquitectura distribuida permite alta disponibilidad, tolerancia a fallos, retención configurable 
y orden garantizado en particiones. \newline

Kafka desacopla la ingesta del procesamiento, permitiendo escalabilidad horizontal 
y resiliencia ante picos de carga.\newline

Se hicieron pruebas con Apache Pulsar, que promete ser el sucesor de Kafka,
pero su adopción en la industria es aún limitada y su ecosistema de herramientas no es tan maduro.
A su vez, su documentación presenta inconsistencias que hacen un desafío su integración en escenarios poco convencionales. \newline

\newpage

\subsection*{Motor de Procesamiento de Flujos - Apache Flink}

Flink fue seleccionado por su baja latencia, manejo avanzado de estado y compatibilidad directa con fuentes como Kafka y formatos de tablas analíticas como Apache Paimon.\newline 

Su modelo de programación basado en flujos de datos y su integración con catálogos lo hacen ideal tanto para arquitecturas Kappa como Delta.\newline 

Además, permite operaciones con garantías \textit{exactly-once} y es capaz de reprocesar eventos desde el principio del log si fuera necesario.\newline

Aunque Apache Spark Structured Streaming podría haber sido considerado, 
Flink presenta menor latencia y mejor soporte para semánticas de estado complejas.

\subsection*{Almacenamiento de Objetos - MinIO}

MinIO fue elegido como la solución de almacenamiento distribuido compatible con la API de S3. 
Su facilidad de despliegue, rendimiento eficiente y compatibilidad con entornos locales lo posicionan como una alternativa ideal a soluciones propietarias en la nube, 
permitiendo independencia de proveedor y bajo costo operativo.

\subsection*{Formato de Tabla Analítica - Apache Paimon}

Apache Paimon fue seleccionado como formato de tabla debido a su diseño nativo para flujos de datos. 
Su integración con Flink es directa y permite operaciones de escritura eficientes en escenarios de actualización frecuente. 
A pesar de tener una comunidad pequeña, 
su especialización para el procesamiento incremental en tiempo real fue clave para esta elección.\newline

Otras alternativas como Apache Iceberg o Hudi tienen comunidades más grandes y mejor documentación, 
pero presentan mayores desafíos en integración directa con Flink 
o no funcionan también para el caso de procesamiento de streaming.

\newpage

\subsection*{Motor Analítico - Apache Doris}

Doris combina procesamiento paralelo masivo (MPP) con consultas OLAP en tiempo real. 
Su soporte para tablas híbridas y la posibilidad de realizar consultas federadas sin necesidad de duplicar los datos lo convierte en una excelente opción para analítica en tiempo real e histórica.
También la integración con Paimon mediante un catálogo lo posiciona como una solución eficiente y simple de operar.\newline

Aunque Druid o Pinot fueron considerados, 
Doris ofrece capacidades que estas otras tecnologías no tienen. En particular, las consultas federadas.

\subsection*{Orquestación de Contenedores - Docker Compose}

El uso de Docker Compose permite una gestión simple, portable y reproducible del entorno experimental. 
Facilita el aislamiento de componentes, el control de versiones de imágenes y la replicabilidad del experimento. \newline

Esta elección responde a la necesidad de contar con un entorno controlado sin la complejidad que presentan otras herramientas.

\subsection*{Monitoreo y Visualización: Prometheus + Grafana}

Prometheus permite recolección y análisis eficiente de métricas, 
mientras que Grafana proporciona una interfaz amigable y personalizable para visualizar el estado del sistema y comparar el rendimiento entre arquitecturas. 

Esta combinación es ampliamente adoptada en entornos productivos, y su integración nativa con sistemas de contenedores y servicios distribuidos la vuelve especialmente útil para este caso.

Además, se aprovecha la capacidad de Grafana para crear paneles de control personalizados 
para también visualizar los resultados de las consultas analíticas.

\newpage